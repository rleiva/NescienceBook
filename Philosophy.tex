%
% CHAPTER - Philosophy
%

\chapterimage{Philosophers.pdf}

\chapter{Philosophy of Science}
\label{chap:Philosophy}

\begin{quote}
\begin{flushright}
\emph{To go where you don't know, \\
you have to go the way you don't know.} \\
San Juan de la Cruz 
\end{flushright}
\end{quote}
\bigskip

The \emph{philosophy of science}\index{Philosophy of science} is the branch of philosophy that examines the foundations, methods, and implications of scientific inquiry. It explores how scientific knowledge is generated, the validity of scientific theories, the nature of scientific reasoning, and the role of values in science. By addressing fundamental questions about objectivity, reality, and the limits of scientific explanation, the philosophy of science helps us understand how science works and its impact on our understanding of the world.

The philosophy of science provides us a theoretical framework for analyzing the core elements that constitute our theory of nescience. It enables us to critically examine the foundations and assumptions of our theory, guiding us toward the identification of the essential questions that we must address. Moreover, philosophical inquiry compels us to rigorously push the boundaries of our analysis, ensuring that our theory is explored to its ultimate consequences, both logically and conceptually. This approach strengthens the robustness of the theory, even if it these results do not directly translate into practical applications.

In the chapter, we provide a concise overview of key elements from the philosophy of science, as well as relevant concepts from other branches of philosophy, such as metaphysics, epistemology, and ontology, that are important to the theory of nescience. Certain other topics within the philosophy of science, such as the problem of objectivity (examining whether science can be truly objective or is influenced by social and personal values) or the role of values in science, including the relationship between ethical, social, and political values and scientific practices, are not included in this review, as they are not directly relevant to a mathematical theory of nescience.

The chapter begins with a brief introduction to the field and its importance in understanding scientific inquiry. We delve into the problem of which entities can be known, examining the scope of scientific knowledge and the nature of observable and unobservable phenomena. The chapter also addresses the concept of scientific representation, discussing how models, theories, and laws reflect aspects of reality. We examine how science discovers new knowledge, the principles behind the scientific method, and the various ways in which scientists formulate and test hypotheses. Finally, we explore the limits of science, considering the boundaries of what science can explain and where its explanatory power may fall short.

%
% Section: Metaphysics
%

\section{What is Science}

\emph{Science}\index{Science} is a systematic method of investigating the world around us, aimed at generating reliable knowledge through observation and reasoning. Unlike other forms of inquiry, science is rooted in the idea that knowledge must be based on evidence that can be tested and verified. By combining theoretical thinking with empirical data, science offers a powerful way to explain, predict, and understand natural phenomena. The following is a list of the key features that make science distinct and valuable as a way of knowing.

\begin{itemize}

\item \emph{Empiricism}\index{Empiricism}: At its core, science is an empirical endeavor, meaning it relies on observation, experimentation, and measurable evidence to understand the natural world. Scientific knowledge is grounded in data that can be gathered through direct or indirect observation, ensuring that claims can be tested and verified by others.

\item \emph{Testability}\index{Testability}: A key characteristic of science is its focus on developing testable hypotheses, that is, statements or predictions that can be empirically investigated. This means that scientific claims must be falsifiable, open to potential disproof if the evidence does not support them. This distinguishes science from fields that rely on unfalsifiable or speculative claims.

\item \emph{Theoretical Frameworks}\index{Theoretical framework}: Science is not merely about collecting facts; it seeks to develop broader explanations through models, laws, and theories. Models offer simplified representations of complex systems, while theories are well-supported frameworks that explain the underlying mechanisms of phenomena. These frameworks help interpret data and guide further investigation, offering coherence to the body of scientific knowledge.

\item \emph{Self-Correction}\index{Self-correction}: A defining feature of science is its self-correcting nature. Scientific theories are not static; they evolve as new evidence comes to light. When new data contradicts a theory, science adapts, modifies, refines, or sometimes discards the theory in favor of one that better fits the evidence. This continuous process of revision ensures that scientific knowledge becomes more accurate over time.

\item \emph{Generalizability}\index{Generalizability}: Science strives to uncover universal principles that apply across various contexts, not just to isolated cases. While science begins with specific observations, its goal is to identify general laws and patterns that explain a broad range of phenomena. This pursuit of generalizable knowledge allows science to predict future occurrences and provide deeper insights into the workings of the natural world.

\end{itemize}

 \emph{Ontology}\index{Ontology}, the branch of philosophy concerned with the nature of existence, addresses the question of which entities exist in the real world. It examines the types of entities that science can study, whether they are observable, abstract, or indeterminate, and engages with the philosophical debates surrounding their existence. Ontology provides the foundation for exploring the boundaries of scientific knowledge and the limits of inquiry. \emph{Epistemology}\index{Epistemology} is the branch of philosophy concerned with knowledge itself—how we acquire it, what justifies it, and what its limits are. While ontology deals with what exists, epistemology addresses how we come to know what exists. In the context of science, epistemology examines the methods, evidence, and reasoning that underpin scientific investigation, exploring how scientific knowledge is built, how reliable it is, and what counts as a justified belief in scientific practice. Both ontology and epistemology are part of the broader field of \emph{metaphysics}\index{Metaphysics}, which deals with the fundamental nature of reality and existence. Metaphysics explores the most basic concepts and categories of being, such as time, space, causality, and possibility, as well as the relationship between mind and matter. Together, ontology, epistemology, and metaphysics provide a comprehensive philosophical foundation for understanding what science studies, how it builds knowledge, and the fundamental nature of the reality science seeks to explain.


%
% Section: What is an Entity
%

\section{What is an Entity}

In this section, we focus on the fundamental problem of determining which kinds of entities can be known or investigated by science.

\begin{figure}[t]
\centering
\begin{tikzpicture}

  % Define the ovals
  \node[draw, ellipse, minimum width=4cm, minimum height=1.1cm, align=center] (knowable) {Knowable};
  \node[draw, ellipse, minimum width=4cm, minimum height=1.1cm, right=of knowable, xshift=2+1cm] (nonknowable) {Non-Knowable};
  \node[draw, ellipse, minimum width=3.5cm, minimum height=1.1cm, below=of knowable, xshift=-2.5cm] (concrete) {Concrete};
  \node[draw, ellipse, minimum width=3.5cm, minimum height=1.1cm, below=of knowable, xshift=2.5cm] (abstract) {Abstract};
  \node[draw, ellipse, minimum width=3cm, minimum height=1.1cm, below=of concrete, xshift=-2.5cm] (observable) {Observable};
  \node[draw, ellipse, minimum width=3cm, minimum height=1.1cm, below=of concrete, xshift=2.5cm] (nonobservable) {Non-Observable};

  % Arrows indicating subset relationships
  \draw[<-] (concrete) -- (knowable) node[midway, left] {};
  \draw[<-] (abstract) -- (knowable) node[midway, right] {};
  \draw[<-] (observable) -- (concrete) node[midway, left] {};
  \draw[<-] (nonobservable) -- (concrete) node[midway, right] {};

\end{tikzpicture}
\caption{Classification of Research Entities}
\end{figure}

A \emph{knowable entity}\index{Knowable entity} is an object, phenomenon, or concept that can be investigated, understood, or described through scientific or intellectual inquiry. Knowable entities are those that, either directly or indirectly, can be observed, measured, inferred, or modeled, using available tools, methods, or theories. They are within the scope of human knowledge, and their properties can be analyzed or explained. Examples of knowable entities include the stars and plantes, animals, or computer algorithms. A \emph{non-knowable entity}\index{Non-knowable entity} refers to something that cannot be directly observed, measured, or understood using current scientific or intellectual methods. This could be due to limitations in technology, the abstract or metaphysical nature of the entity, or inherent epistemological boundaries. Non-knowable entities might remain beyond the reach of human understanding either temporarily (until methods evolve) or permanently (due to their nature). Examples of non-knowable entities include the nature of consciousness, the origin of the universe, or the existence of a deity. The distinction between knowable and non-knowable entities is not always fixed, but varies depending on the clarity and precision of the area of interest and the evolving nature of scientific understanding in each field.

Scientific research encompasses both \emph{concrete entities}\index{Concrete entities} and \emph{abstract entities}\index{Abstract entities}. Concrete entities refer to physical objects or phenomena that can be directly observed, measured, or interacted with, such as stars, cells, or chemical compounds. These entities form the basis of empirical research, where data is collected through direct observation or experimentation. In contrast, abstract entities are conceptual and do not have a physical presence, such as numbers, algorithms, or theoretical models. Abstract entities play a crucial role in scientific research, particularly in fields like mathematics and theoretical physics, where they provide the framework for understanding and modeling concrete phenomena. While abstract entities cannot be observed directly, they can be known through indirect methods. The inclusion of these abstract entities in scientific inquiry raises important philosophical questions about their ontological status: Are these abstract entities real in the same way that physical objects are, or are they simply conceptual tools? This question remains an open and debated issue in the philosophy of science.

Finally, in scientific inquiry, there is also a distinction between \emph{observable entities}\index{Observable entities}, which can be directly perceived or measured, and \emph{non-observable entities}\index{Non-observable entities}, which are inferred from empirical evidence but cannot be directly observed. Observable entities include things like trees, planets, and bacteria—objects that can be seen or detected using scientific instruments. Non-observable, but still concrete, entities include things like subatomic particles and gravitational forces. These entities are often crucial for explaining observable phenomena but exist at a level beyond direct human perception. For example, we cannot observe an electron in the same way we observe a tree, but through scientific theory and experimentation, we infer its existence.

A central debate within the scope of science is the tension between \emph{reductionism}\index{Reductionism} and \emph{holism}\index{Holism}. Reductionism is the view that complex systems can be fully understood by breaking them down into their simplest, most fundamental parts. For example, a reductionist might argue that biological processes can be explained entirely by chemistry, and chemistry by physics. This approach assumes that understanding the smallest components of a system will provide a complete explanation of the whole. In contrast, holism argues that some phenomena cannot be fully understood by reducing them to their components. Instead, the whole system exhibits properties that cannot be predicted or explained by analyzing its parts in isolation. For example, in ecology, the interactions within an ecosystem can produce emergent properties that are not reducible to the behavior of individual species.

Finally, the scope of science is shaped by the debate between \emph{scientific realism}\index{Realism} and \emph{anti-realism}\index{Anti-realism}, which addresses the question of whether the entities posited by scientific theories are real or merely useful constructs. Scientific realism holds that the entities described by scientific theories—whether observable or not—exist independently of our knowledge of them. According to this view, successful scientific theories reveal truths about the world. In contrast, anti-realism (or \emph{instrumentalism}\index{Instrumentalism}) argues that scientific theories are useful tools for predicting and organizing observations, but we should not necessarily believe that unobservable entities like electrons or gravitational waves are real. For anti-realists, the purpose of science is not to describe an independent reality, but rather to provide models that help us navigate and predict phenomena. This ontological uncertainty leads to philosophical discussions about \emph{natural kinds}\index{Natural kinds}—whether the categories used in science reflect real, essential divisions in nature, or whether they are human-made constructs imposed to bring order to a complex and often ambiguous reality. 

%
% Scientific Representation
%

\section{Scientific Representation}
\label{sec:scientific_representation}

Science helps us understand the natural world by using different kinds of representations of research entities. These representations include measurements from scientific instruments, descriptions of observations, digital images like X-rays or MRI scans, and more. Scientific practice also often considers mathematical equations, models, and theoretical constructs as valid forms of representation. The challenge of \emph{scientific representation}\index{Scientific representation} is to identify the conditions that make a representation scientific and determine what makes an effective representation. The main issues discussed in this area of philosophy include:

\emph{Scientific Representation Problem}: The scientific representation problem is about figuring out the necessary and sufficient conditions that make a representation valid in science. It explores whether these conditions are the same across all scientific fields or if they vary depending on the discipline or research context. For example, what qualifies as a valid representation in physics, which often uses mathematical models, might be different from what is used in biology, where visual and descriptive representations are common. This problem also questions whether scientific representations need to be adapted to specific research goals to be considered valid.

\emph{Representational Demarcation Problem}\index{Representational demarcation}: The representational demarcation problem looks at whether scientific representations are fundamentally different from other kinds of representations, like those found in art or everyday life. It examines what makes scientific representations unique, focusing on their purpose, accuracy, and the methods used to create them. Unlike artistic representations, which may emphasize subjective interpretation or aesthetic value, scientific representations are generally held to standards of precision, reliability, and empirical adequacy. Understanding these differences helps clarify the specific role that scientific representations play in knowledge production.

\emph{Problem of Style}\index{Problem of style}: The problem of style addresses the fact that the same entity can be represented in different ways, depending on the goals and methods of the research. Different styles of representation—such as diagrams, mathematical equations, physical models, or computer simulations—each have unique characteristics, including the intended audience, level of abstraction, and type of information conveyed. This issue also asks whether these styles are fixed or if new styles can be invented to meet emerging scientific needs. The flexibility of representation styles is crucial because it allows for new insights and different ways of understanding scientific phenomena.

\emph{Standard of Accuracy}\index{Standard of accuracy}: The standard of accuracy problem is about determining what makes a scientific representation accurate. It involves figuring out how to distinguish between accurate and inaccurate representations by considering factors like how well the representation matches empirical data, captures important features of the phenomenon, and its ability to make predictions. This issue also explores whether accuracy should be seen as an objective standard or if it depends on the specific aims and context of the research. For example, a simplified model might still be considered accurate if it effectively serves its purpose, such as making predictions or providing explanations.

\emph{Problem of Ontology}\index{Problem of ontology}: The problem of ontology in scientific representation deals with the nature of the entities that can serve as representations. It asks whether representations need to be concrete, like physical models or graphs, or if they can also be abstract, like mathematical equations or theoretical constructs. This issue also questions whether representations must be realistic or if more abstract, idealized forms can still be effective in scientific inquiry. Understanding these ontological aspects helps define the types of entities that are allowed in scientific discourse and how they relate to the real-world phenomena they represent.

There are also five conditions of adequacy that a scientific representation should satisfy to be considered effective and reliable:

\emph{Requirement of Directionality}\index{Requirement of directionality}: The requirement of directionality examines the relationship between representations and the real world. Representations are meant to describe entities in the real world, but this condition raises the question of how, if at all, real-world entities might describe their representations. It challenges us to think about the direction of influence between the representation and the entity it aims to depict.

\emph{Surrogative Reasoning}\index{Surrogative reasoning}: Surrogative reasoning addresses how scientific representations allow researchers to generate hypotheses about the entities they represent. This condition explores how using a representation as a surrogate can lead to new insights or predictions about the target, effectively using the representation to stand in for the real-world entity during reasoning and analysis.

\emph{Applicability of Mathematics}\index{Applicability of mathematics}: The applicability of mathematics condition is concerned with how mathematical models can be used to represent the real world. It questions how abstract mathematical constructs can effectively describe complex physical systems and whether the success of mathematical representation depends on any special features of the target phenomena. This condition highlights the central role of mathematics in developing and understanding scientific theories.

\emph{Possibility of Misrepresentation}\index{Possibility of misrepresentation}: The possibility of misrepresentation addresses whether representations that are not fully accurate can still be considered valid scientific representations. It considers situations where simplifications or approximations are necessary and whether these less-than-perfect representations can still contribute valuable understanding of a phenomenon. This condition is important for understanding how idealizations and abstractions function in scientific practice.

\emph{Targetless Models}\index{Targetless models}: The targetless models condition explores whether we can allow representations that do not have a direct real-world counterpart. It questions if a model that does not represent any existing entity can still be useful in scientific inquiry, perhaps as a way to explore theoretical possibilities or to understand potential scenarios. This condition emphasizes the creative and exploratory aspects of scientific modeling.

There have been multiple proposals to formally define the concept of scientific representation. Unfortunately, none of these proposals can provide a convincing answer to the questions and conditions of adequacy described above. In the rest of this section, we describe some of these proposals, identifying their advantages and drawbacks. To compare these proposals, we will present them as: "A scientific model $M$ represents a target system $T$ if, and only if ...".

\emph{Stipulative Fiat}\index{Stipulative fiat}: The stipulative fiat proposal states that "a scientific model $M$ represents a target system $T$ if, and only if, a scientist stipulates that $M$ represents $T$." The main problem with this interpretation is that, since anything can be a representation if a scientist says so, it is difficult to guarantee the surrogative reasoning condition. If any model can be deemed a representation by simple stipulation, it becomes challenging to determine which representations are genuinely useful for making scientific inferences. Proponents of this theory acknowledge that while all representations may be stipulated, some are undeniably more useful than others.

\emph{Similarity Conception}\index{Similarity conception}: The similarity conception proposes that "a scientific model $M$ represents $T$ if, and only if, $M$ and $T$ are similar." This conception addresses the surrogative reasoning condition since similarity between the model and the target allows us to derive similar properties. However, it introduces new challenges, particularly regarding the problem of style. The concept of similarity is often vague: in what sense are $M$ and $T$ similar? This vagueness can lead to issues with directionality and accuracy, as different aspects of similarity may not always align with what is relevant for scientific representation.

\emph{Structuralist Conception}\index{Structuralist conception}: The structuralist conception is based on the idea of isomorphism. According to this view, a scientific model $M$ represents a target system $T$ if the structure of $M$ is isomorphic to the structure of $T$. In other words, there is a one-to-one correspondence between the elements and relationships in both $M$ and $T$. This approach justifies surrogative reasoning because having the same structure implies that properties and relations in the model correspond to those in the target. Furthermore, since mathematics is fundamentally concerned with the study of structures, this conception also supports the applicability of mathematics in representing natural systems.

\emph{Inferential Conception}\index{Inferential conception}: The inferential conception proposes that a model $M$ is an epistemic representation of a target $T$ if, and only if, the user adopts an interpretation of $M$ in terms of $T$. This view emphasizes the role of the user in giving meaning to the model, suggesting that representation is not an inherent property of the model itself but arises through its use in making inferences about the target system. This conception underscores the importance of context and interpretation in determining whether a model effectively represents its target.

\emph{Fiction View of Models}\index{Fiction view of models}: According to the fiction view of models, $M$ represents $T$ if and only if $M$ functions as a prop in a game of make-believe that prescribes imagining certain things about $T$. This view draws an analogy between scientific modeling and storytelling, where models are treated as fictional constructs that facilitate imaginative engagement with the target system. Although this approach highlights the creative aspects of modeling, it raises questions about how such fictional constructs can be rigorously linked to real-world entities.

\emph{Representation-As}\index{Representation-as}: The representation-as approach suggests that a scientific model represents a target system as something, emphasizing that representation involves highlighting certain features of the target while downplaying others. This conception focuses on the interpretive aspect of modeling, where the modeler selects specific attributes of the target to represent, depending on the research goals. This approach allows for a flexible understanding of representation that can accommodate different styles and purposes, but it also implies that the usefulness of a representation is contingent on how well the modeler captures the relevant aspects of the target.

Each proposal has its strengths and weaknesses, highlighting the complexity of what it means for a model to effectively represent a target in scientific inquiry. Understanding these various perspectives is crucial for advancing our comprehension of the role of representation in science.


%
% Scientific Explaination
%

\section{Scientific Explaination}

The assumption often made is that there exists a single, distinct type of explanation that qualifies as "scientific." The concept of "scientific explanation" suggests at least two key distinctions: first, a contrast between explanations characteristic of science and those that are not, and second, a contrast between "explanations" and other forms of discourse, such as mere "descriptions." It is important to note that a set of claims can be true, accurate, and supported by evidence while still failing to qualify as explanatory. Scientific explanations primarily aim to clarify why things happen, whether these "things" are particular events or more general phenomena. Central to this discussion are interrelated concepts such as "explanation," "law," "cause," and "support for counterfactuals," all of which share a "modal" character. However, explaining one of these concepts using others from the same family is often considered "circular"; instead, they must be elucidated through concepts external to this modal framework. Additionally, a significant question arises regarding whether all scientific explanations are inherently causal, and if not, what precisely distinguishes causal explanations from non-causal ones.

The \emph{Deductive-Nomological model}\index{Deductive-Nomological model}, or DN model, emphasizes the importance of deductive reasoning and general laws. According to this model, a phenomenon is explained by demonstrating how it logically follows from a general law combined with specific initial conditions. This an explanation comprises two main components: the \emph{explanans}\index{explanans}, which includes the general laws and initial conditions, and the \emph{explanandum}\index{explanandum}, which is the phenomenon to be explained. For the explanation to be valid, the explanans must be true and logically entail the explanandum, meaning the phenomenon can be deduced from the laws and conditions provided. A DN model answers the question "Why did the explanandum occur?" by showing that the phenomenon resulted from specific circumstances $C_1, C_2, \ldots, C_i$, in conjunction with laws $L_1, L_2, \ldots, L_j$. For example, the motion of a pendulum can be explained by applying Newton's laws of motion (general laws) along with specific details such as the pendulum's length and initial displacement.

The \emph{Statistical-Relevance model}\index{Statistical-Relevance model}, or SR model focuses on explaining phenomena through statistical relationships rather than strict deductive reasoning. Unlike the Deductive-Nomological model, which requires logical entailment from general laws, the SR model emphasizes the identification of statistically relevant factors that significantly influence the likelihood of a phenomenon. In this model, given some class or population \( A \), an attribute \( C \) is \emph{statistically relevant}\index{Statistical relevance} to another attribute \( B \) if and only if \( P(B | A, C) \neq P(B | A) \). This means that \( C \) affects the probability of \( B \) within the context of \( A \). An explanation involves identifying such statistically relevant factors and evaluating their impact within a reference class (a group of events or entities sharing common characteristics). For example, in explaining the likelihood of developing a particular disease, an SR explanation might highlight factors such as age, genetic predisposition, or lifestyle choices, showing how these variables alter the probability of the disease occurring. By uncovering these statistical relationships, the SR model provides a method for explaining probabilistic phenomena that cannot be addressed deterministically.

The \emph{Causal-Mechanical model}\index{Causal-Mechanical model}, or CM model, of scientific explanation emphasizes understanding phenomena by uncovering the underlying causal mechanisms that produce them. This model asserts that explanations are not just about identifying laws or statistical relationships but about revealing the actual processes and interactions that link causes to effects. A causal-mechanical explanation requires tracing a continuous causal chain, often through detailed physical or biological processes, to show how an event is brought about. For example, explaining the boiling of water involves identifying the causal mechanism: heat energy transfers to the water molecules, increasing their kinetic energy until intermolecular bonds are overcome, leading to a phase change from liquid to gas. The CM model prioritizes clarity in how individual components interact and influence each other, providing a deeper understanding of the phenomenon by grounding it in observable and empirically testable mechanisms. This approach is particularly effective in fields like biology, physics, and engineering, where complex systems and their interactions play a central role in explanation.

The \emph{unificationist account}\index{Unificationist account} of scientific explanation, emphasizes the power of explanation through the unification of diverse phenomena under a single, coherent framework of principles and patterns. According to this model, the primary aim of scientific explanation is to reduce the number of independent assumptions and derive a wide range of phenomena from a minimal, consistent set of explanatory patterns. An explanation is considered successful if it contributes to this unifying framework by connecting seemingly disparate observations through common principles. For instance, Newtonian mechanics unifies the motions of celestial bodies and terrestrial objects under the same laws of motion and gravitation. The unificationist approach highlights the importance of simplicity, generality, and coherence in scientific theories, proposing that the value of an explanation lies in its ability to integrate knowledge into an organized, explanatory schema. By offering a comprehensive understanding of diverse phenomena, this account showcases the interconnectedness and systematic nature of scientific inquiry.

The \emph{pragmatic theories}\index{Pragmatic theories} of scientific explanation emphasize the context-dependent and audience-specific nature of explanations, focusing on their purpose and practical utility rather than strict formal structures. These theories argue that explanations are answers to "why" questions posed within a specific context, and their adequacy depends on how well they address the interests and background knowledge of the audience. A scientific explanation, therefore, is not inherently tied to a universal standard but varies depending on the explanatory goals, such as prediction, understanding, or control. For instance, explaining why a bridge collapsed might involve detailed structural analysis for engineers, whereas a simplified account focusing on the immediate cause, like high winds, might suffice for the general public. Pragmatic approaches recognize that explanatory demands can differ across disciplines, situations, and audiences, making the effectiveness of an explanation contingent on its relevance, clarity, and alignment with the inquirer's needs. This perspective underscores the interplay between scientific knowledge and its communication within varied practical contexts.


%
% The Scientific Method
%

\section{The Scientific Method}

The \emph{scientific method}\index{Scientific method} is understood as the systematic process by which science acquires new knowledge. In schools, the scientific method is often taught as a series of steps: first observing and describing a phenomenon, then coming up with a hypothesis to explain it, testing that hypothesis through experiments, analyzing the data, and finally making a conclusion. However, the idea of a single, universal scientific method isn’t widely accepted anymore. Instead, scientists and philosophers recognize that different scientific fields require different methods because each field faces its own challenges and complexities.

A major idea in scientific methodology is the difference between \emph{discovery}\index{Scientific discovery} and \emph{justification}\index{Scientific justification}. Discovery is about how new ideas or hypotheses come to life, which often happens through creativity, intuition, or even by accident. Justification, on the other hand, is about using evidence and logical reasoning to evaluate these ideas and decide if they are valid. Most philosophers of science have focused more on justification, looking at ways to carefully test and analyze ideas rather than on the less structured processes involved in creating them.

% Scientific Discovery
\subsection{Scientific Discovery}

Scientific discovery refers to the process through which new knowledge, ideas, or principles are uncovered within science. Unlike the systematic procedures associated with justification, discovery often involves creativity, intuition, and inspiration. While some discoveries arise from planned experiments or systematic observation, others occur unexpectedly, challenging existing paradigms or opening new fields of inquiry. The general agreement among philosophers is that the creative process of conceiving a new idea is a non-rational process that cannot be formalized as a set of steps. Understanding discovery is crucial for appreciating how science evolves and adapts, as it reveals the dynamic and often unpredictable nature of scientific progress.

The following two proposals assume that a domain-neutral logic of discovery can be formalized, offering attempts to develop such a framework.

\begin{itemize}

\item \emph{Discovery as abduction}\index{Abductive reasoning}: Abductive reasoning is a mode of discovery that begins with surprising or anomalous phenomena and seeks to generate plausible hypotheses to explain them. This process is conceptualized as follows: (i) some unexpected data, such as $p_1, p_2, \ldots, p_n$, is encountered; (ii) these data would be less surprising if a hypothesis of type $H$ were true; and (iii) therefore, there is justification to develop a hypothesis of type $H$. Two types of abduction are distinguished: \emph{selective abduction}, which involves choosing from known hypotheses, and \emph{creative abduction}, which generates entirely new hypotheses. Abduction present some limitations. First, multiple hypotheses may explain the same phenomena, making additional criteria necessary to evaluate their merit. Second, the schema of abductive reasoning does not account for the act of conceiving a hypothesis itself.

\item \emph{Heuristic programming}\index{Heuristic programming} is a computational approach designed to simulate and assist the creative aspects of human problem-solving. These programs operate as searches within a defined problem space, which includes all possible configurations for a given domain. Each configuration represents a specific state within the problem space, with two key states being the initial state, the starting point of the search, and the goal state, which represents the desired outcome. Operators define the moves that transition between states, while path constraints limit permissible moves within the problem space. Problem-solving in this context involves finding a sequence of operations that connects the initial state to the goal state. The core aim of this approach is to develop heuristics (practical rules or strategies) to efficiently navigate and solve complex problems. Heuristic programming has its limitations, scientific problem spaces are often ill-defined, and computer programs rely on experimental data, meaning that simulations frequently cover only specific aspects of scientific discovery.

\end{itemize}

Many philosophers argue that discovery is an important topic within the philosophy of science, even as they move away from the idea of a formal logic of discovery. A highly influential perspective is Thomas Kuhn's examination of how new facts and theories emerge in the so-called \emph{paradigm shifts}\index{Paradigm shift}. According to Kuhn, discovery is not a single event but rather a complex and prolonged process that often results in paradigm shifts. Paradigms consist of shared generalizations, theoretical commitments, values, and exemplars that unify a scientific community and shape its research practices. During periods of normal science, research focuses on expanding and refining the existing paradigm rather than pursuing novelty. Discovery typically begins with the recognition of anomalies—phenomena that defy the expectations established by the current paradigm. This process includes observing and conceptualizing the anomaly, followed by revising the paradigm to accommodate it. During paradigm crises, theory-driven discoveries may occur as scientists propose speculative theories, develop new expectations, and conduct experiments or observations to test these ideas. Ultimately, a new paradigm emerges, transforming the once-anomalous phenomena into standard expectations.

{\color{red} Others argue that discovery is governed by a methodology. The methodology of discovery is a legitimate topic for a philosophical analysis.}

\begin{itemize}

\item Discoverability

\item Preliminary apprisal

\item Heuristic strategies

\end{itemize}

{\color{red} Yet another response assumes that discovery is or at leas involves a creative act. Drawing on resources, from cognitive science, neurosience, computational research, and environmental and social psychology, philosophers have sought to demystify the cognitive processes involved in the gneration of new ideas. Philosophers who take this approach argue that scientific creativity is amenable to philosophical analysis.}

{\color{red} Philosophical studies of creativity [...] integrate philosophical analysis with empirical work from cognitive science, psychology, evolutionary biology and computational neuroscience [...] based on the asumption that creativity can be analyzed and that empirical research can be contribute to the analysis.}

{\color{red} General definitions of creativity highlight novelty or originality and significance or value as distinctive features of a creative act or product.}

{\color{red} Two key elements of the cognitive process involved in creative thinging that have been in the focus of philosophical analysis are analogies and mental models}

\begin{itemize}

\item Analogy

\item Mental models

\end{itemize}





% Scientific Justification
\subsection{Scientific Justification}

Thomas Kuhn introduced the concept of scientific paradigms, which revolutionized the understanding of scientific progress. A paradigm encompasses the set of practices, theoretical frameworks, methodologies, and standards that define a scientific discipline during a specific period. According to Kuhn, normal science operates within the confines of a prevailing paradigm, solving puzzles and extending the framework. However, scientific revolutions occur when anomalies accumulate—phenomena that the existing paradigm cannot adequately explain—leading to a paradigm shift. This shift replaces the old framework with a new one that better accommodates the observed data, fundamentally altering the trajectory of scientific inquiry. Kuhn's insights emphasize the sociocultural and historical dimensions of science, challenging the notion of continuous, cumulative progress.

Paul Feyerabend, in his provocative work "Against Method," argued that there is no universal scientific method that guarantees the success of science. He criticized the rigid methodological rules proposed by philosophers like Popper and Kuhn, suggesting that science has advanced precisely because of its methodological anarchy. Feyerabend contended that "anything goes" in science, meaning that historical scientific breakthroughs often occurred by defying established rules and norms. For example, Galileo's use of persuasive rhetoric and creative reasoning, rather than strict adherence to empirical observation, was crucial in advancing heliocentrism. Feyerabend's ideas challenge the assumption that scientific progress is orderly and rational, emphasizing instead the role of historical, cultural, and personal factors in shaping scientific practice. While his views remain controversial, they highlight the complexity and diversity of scientific inquiry and question the feasibility of prescribing universal methodological principles.

The hypothetico-deductive method is a widely recognized approach in the philosophy of science, describing how scientific theories and knowledge are developed and tested. It begins with the formulation of a hypothesis, which serves as a tentative explanation for a phenomenon or a set of observations. From this hypothesis, scientists deduce specific, testable predictions or consequences, often in the form of "if-then" statements. These predictions are then subjected to empirical testing through experiments or observations. If the results align with the predictions, the hypothesis is supported but not conclusively proven; if the results contradict the predictions, the hypothesis may need to be revised or rejected. This iterative process underscores the provisional nature of scientific knowledge.

Falsificationism emphasizes that scientific theories can never be conclusively proven but can be rigorously tested through attempts to falsify them. An example of this is Einstein's theory of general relativity, which predicted that light would bend near massive objects like the sun. This prediction was empirically tested during the solar eclipse of 1919, when astronomers observed starlight bending as it passed near the sun, providing a crucial opportunity to potentially falsify the theory—but instead offering strong support for it. According to this perspective, a theory is scientific only if it is falsifiable—that is, if it makes predictions that could, in principle, be shown to be false by empirical evidence. In this view, the strength of a scientific theory lies in its ability to withstand attempts at falsification, and progress in science occurs when falsified theories are replaced by better, more robust ones. This perspective shifts the focus from verification to critical testing, highlighting the dynamic and self-correcting nature of scientific inquiry.

One challenge within the hypothetico-deductive method is addressing the origin of hypotheses themselves. While the method provides a systematic way to test and refine hypotheses, it does not dictate where these initial ideas come from. In the past, the generation of hypotheses was often the direct result of careful observations of natural phenomena. However, as science has progressed, hypotheses have increasingly become products of creativity, intuition, or inspiration drawn from prior knowledge, analogies, or even serendipitous observations. This aspect highlights the interplay between the logical structure of scientific testing and the imaginative processes that fuel scientific discovery. Philosophers of science have long debated whether hypothesis generation is a purely rational process or one influenced by subjective and contextual factors, underscoring the complexity of scientific creativity.

Statistical inference has become a significant approach within the scientific method, providing a framework for deriving conclusions from data amidst uncertainty. By employing probabilistic models and statistical techniques, scientists can evaluate the likelihood that observed phenomena are consistent with a given hypothesis. Techniques such as hypothesis testing, regression analysis, and Bayesian inference allow researchers to quantify uncertainty and make data-driven decisions. This approach is particularly powerful in fields where direct experimentation is difficult or impossible, such as cosmology or epidemiology. However, reliance on statistical methods also raises important questions about the interpretation of probabilities and the potential for misuse, such as overfitting models or neglecting prior assumptions. Despite these challenges, statistical inference remains an indispensable tool for connecting empirical data to theoretical models in modern science.

Bayesian inference represents a powerful extension of statistical reasoning within the scientific method. Rooted in Bayes' theorem, this approach provides a formal framework for updating the probability of a hypothesis in light of new evidence. Bayesian inference incorporates prior knowledge or assumptions about the likelihood of a hypothesis and combines it with observed data to produce a posterior probability. This iterative process allows for a dynamic adjustment of beliefs, reflecting how evidence accumulates over time. Bayesian methods are particularly useful in contexts of uncertainty or incomplete data, such as medical diagnosis, climate modeling, and artificial intelligence. While it offers a flexible and coherent framework, critics argue that the reliance on subjective priors can introduce biases, emphasizing the need for careful justification of these assumptions. Despite this, Bayesian inference continues to shape modern scientific practices, highlighting the interplay between evidence, prior knowledge, and probabilistic reasoning.


%
% The limits of science
% 

\section{The Limits of Science}

\begin{verbatim}
   - **Reductionism vs. Holism**  
     Discussion on whether complex phenomena can be reduced to basic scientific laws or require holistic approaches.
   - **The Limits of Scientific Explanation**  
     Addressing the boundaries of scientific inquiry, such as metaphysical or moral questions.
   - **Scientific Realism vs. Anti-Realism**  
     Debate about whether science uncovers true aspects of the world or just useful models.
   - **The Problem of Objectivity**  
     Exploring whether science can be truly objective or is influenced by social and personal values.
   - ** The Demarcation Problem
     Discussing the philosophical challenge of clearly distinguishing between science and non-science (including pseudoscience), and why this boundary is often contested.

Limitations of Scientific Knowledge
Addressing areas where science may have limited access, such as consciousness, moral values, or metaphysical questions.

\end{verbatim}


%
% Section: References
%

\section*{References}

{\color{red} Add entry of Scientific Representation in the Stanford Encyclopedia of Philosophy}

{\color{red} Perhaps add entries for Lewis Carroll's Sylvie and Bruno and Borges' On Exactitude in Science}

\ref{pirsig1999zen} contains an interesting review of the concept of science, the scientific method, and the role that technology plays in our society. The author proposes that the goal of science should be quality, although the concept of quality is left undefined, and how to reconcile the rational and romantic points of view in science. The book also contains some advice about which is the right state of mind to pursue a scientific problem, and how to deal with the inevitable failures.


