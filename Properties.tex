%
% CHAPTER 8.- Properties of Nescience
%

\chapterimage{Philosophers.pdf}

\chapter{Advanced Properties}
\label{chap:Properties-Nescience}

\begin{quote}
\begin{flushright}
\emph{Invert, always invert.}\\
Carl Gustav Jacob Jacobi\\
\end{flushright}
\end{quote}
\bigskip

This chapter covers more advanced mathematical properties of the concept of nescience. This chapter can be safely skipped by those readers interested only in the applications of nescience. However, it is highly recommended to read it, since if provides a deeper understanding of what nescience is exactly.

{\color{red} TODO: Explain the abstract nature of the axioms vs. the interpretation we have provided in the previous chapters. Mention that perhaps there exists other interpretations.}

Since the time of David Hilbert, mathematics is about the study of the properties of abstract objects and their relations, without paying too much attention to what these objects are or represent. Mathematicians create (or discover) abstract frameworks of logic that can have multiple interpretations. It is up to applied scientists to provide those interpretations. In the same way, we can provide an abstract definition of the concept of nescience, and study its properties, without making any explicit reference to science nor to the scientific method. In doing so, we loose the interpretability of our theory, but, on the other side, we could apply the same results to other disciplines.

{\color{red} Extend this introduction with a very short review of the topics covered in the chapter.}

%
% Section: Axioms
%

\section{The Axioms of Science}

In this section we propose a collection of axioms that formalize the concept of science and the scientific method. Our axioms are based on first-order logic and the ZFC axioms of set theory with equality (see Appendix \ref{apx:foundations_mathematics}). We also prove some basic results and show how the axioms relate to the concepts we have introduced in the previous chapters. In Appendix \ref{apx:coq} the reader can find the axioms and proofs implemented in the \texttt{Coq} proof assistant language.

Let $\mathcal{B}^\ast$ be the set composed by all finite binary strings, and let $\Sigma \subset \mathcal{B}^\ast$ be a subset of strings. A \emph{science} is a first-order logic structure $(\Sigma \mid \lambda, \mathcal{O}, <_n, \oplus, \mid, U)$ where:

\vskip 0.25cm

\begin{enumerate}[label=(\roman*)]
\item $\lambda \in \Sigma$ is the empty string.
\item $\mathcal{O}$ is a relation called \emph{oracle},
\item $<_n$ is a relation called \emph{nescience},
\item $\oplus$ is a binary function called \emph{joint},
\item $\mid$ is a binary function called \emph{conditional}, and
\item $\mathcal{U}$ is a unary function called \emph{machine}
\end{enumerate}

\vskip 0.25cm

The oracle $\mathcal{O}$ is an equivalence relation:

\vskip 0.25cm

\begin{description}
\item[Axiom 1] $\forall t \in \Sigma \; t \mathcal{O} t$.
\item[Axiom 2] $\forall s , t \in \Sigma$ if $s \mathcal{O} t$ then $t \mathcal{O} s$.
\item[Axiom 3] $\forall r, s , t \in \Sigma$ if $r \mathcal{O} s$ and $s \mathcal{O} t$ then $r \mathcal{O} t$.
\end{description}

\vskip 0.25 cm

The oracle equivalence relation $\mathcal{O}$ partitions the set of strings $\Sigma$ into the quotient set $\Sigma / \mathcal{O}$. Each equivalence class $[e] \in \Sigma / \mathcal{O}$ refers to an entity, and it contains all possible representations and descriptions related to that entity. However, the entities themselves are not part of the axioms. The abstract oracle is the only tool we have at our disposal to match entities and their representations. The schema used internally by the oracle to encode the original entities is, in general, unknown. In practice, the best results are achieved if we take into account the recommendations of Chapter \ref{cha:Topics-and-Descriptions} about how to choose the right entities to study. Of course, we could select a collection of entities that do not make any sense, and still apply the axioms, in the same way that we could select a collection of nonsense atomic propositions and apply to them the rules of propositional logic.

For the rest of this section, when we talk about entities, we refer to the equivalence classes defined by $\Sigma / \mathcal{O}$.

The nescience relation $<_N$ is a strict partial order:

\vskip 0.25cm

\begin{description}
\item[Axiom 4] $\forall t \in \Sigma$ we have that $\lnot t <_N t$.
\item[Axiom 5] $\forall s , t \in \Sigma$ if $s <_N t$ then $\lnot t <_N s$.
\item[Axiom 6] $\forall r , s, t \in \Sigma$ if $r <_N s$ and $s <_N t$ then $r <_N t$.
\end{description}

\vskip 0.25cm

We cannot provide a quantitative measure of nescience, since numbers are not part of our axiomatic structure. Instead what we have defined is an ordering of the different elements of $\Sigma$ according to how much we do not known given those strings. The ordering has to be partial, i.e. not every pair of elements of $\Sigma$ can be compared, and strict, i.e. nescience equality is not defined. The problem of assuming a non-strict total order for nescience $\leq_N$ is that if $s \leq_N t$ and $t \leq_N s$ then $s = t$, that is, if two strings have the same nescience, they must be the same string, and this is not necessarily the case. A non-strict total order matching our intuitive notion of nescience would require to include natural numbers in the axioms, and to distinguish between the symbols $=$ for equality of strings and $=_N$ for the equality of nescience. Instead, within our axiomatic structure, nescience equality is just a notational convention.

\begin{notation}
We denote by $ s =_N t$ the case that $(s, t) \notin <_N$.
\end{notation}

We want the symbol $=_N$ to represent the case that our unknown given the strings $s$ and $t$, which may belong to the same equivalence class or to different equivalence classes, be the same. In order to do that we need additional axioms.

Every equivalence class defined by $\mathcal{O}$ has at least one minimal element with respect to the relation $<_N$:

\vskip 0.25cm

\begin{description}
\item[Axiom 7] For all $[e] \in \mathcal{O} / <_N$ there exists at least one $r \in [e]$ such that it does not exists an $s \in [e]$ such that $s <_N r$.
\end{description}

\vskip 0.25cm

A class could have more than one minimal element with respect to the nescience ordering. Those minimal elements correspond to what we have called the valid representations of an entity. All the other representations of that entity that are not minimal are misrepresentations, and so, they have a higher nescience. The axioms do not require the nescience relation to have a global minimum element for $\Sigma$, nor minimum elements for the individual equivalence classes.

{\color{red} TODO: Clarify how those minimal elements relate to descriptions.}

The concatenation function $\oplus : \Sigma \times \Sigma \rightarrow \Sigma$ is a binary function that satisfy the following properties:

\vskip 0.25cm

\begin{description}
\item[Axiom 8] $\forall s, t \in \Sigma$ we have that $\oplus(s, t) \in \Sigma$.
\item[Axiom 9] $\forall s \in \Sigma$ we have that $\oplus(s, \lambda) = \oplus( \lambda, s) = s$.
\item[Axiom 10] $\forall r, s, t \in \Sigma$ we have that $\oplus(\oplus(r, s), t) = \oplus(r, \oplus(s, t))$.
\end{description}

\vskip 0.25cm

$\Sigma$ together with the concatenation operation $\oplus$ forms a free monoid.

\begin{proposition}
We have that $\Sigma = \Sigma^\ast = \mathcal{B}^\ast$. 
\end{proposition}
\begin{proof}
{\color{red} TODO: Pending!}
\end{proof}

Since this point we will use the infix notation of the concatenation function, that is, we will write $s \oplus t$ instead of the $\oplus(s, t)$, and instead of $\Sigma$ we will talk about $\Sigma^\ast$.

Next axiom states how the operation of concatenation relates to nescience (recall that $\Sigma^{+}$ denotes $\Sigma^\ast-\{\lambda\}$):

\vskip 0.25cm

\begin{description}
\item[Axiom 11] Let $r \in \Sigma^{+}$ be a non-minimal element, then $i)$ there exist a $s \in \Sigma^{+}$ such that $r \oplus s <_N s$ or $s \oplus r <_N s$, or $ii)$ there exists a $t \in \Sigma^{+}$ in the form $r = \alpha \oplus t \oplus \beta$ with $\alpha, \beta \in \Sigma^\ast$ such that $t <_N r$, or both.
\end{description}

\vskip 0.25cm

{\color{red} TODO: Axiom 11 relates to representations.}

{\color{red} TODO: Axiom 12 relates to descriptions.}

{\color{red} TODO: Explain what happens with $\lambda$, the class it belongs to, and its role as a minimal element.}

Next proposition shows how we can reduce the nescience of the non-valid representations, either by adding more relevant symbols, or by elmininating non-relevant symbols.

\begin{proposition}
Let $r \in \Sigma^{+}$ be a non-minimal element of class $[e]$, then either:
\begin{enumerate}[label=(\roman*)]
\item there exists a $s \in [e]$ such that $r \oplus s <_N r$ or $s \oplus r <_N r$
\item $r$ = $uvw$, where $u, w \in \mathcal{B}^\ast$ and $v \in \mathcal{B}^{+}$, such that $v <_N r$.
\end{enumerate}
\end{proposition}
\begin{proof}
{\color{red} TODO: Pending}
\end{proof}

The same happens when we concatenate strings that belong to different classes.

{\color{red} TODO: Prove that the concatenation of two minimal elements of the same class is also minimal.}

{\color{red} TODO: Explain how axioms 11 and 12 relate to the concatenation of descriptions.}

The machine $\mathcal{U} : \Sigma^\ast \rightarrow \Sigma^\ast$ is an universal Turing machine, that given a string $s \in \Sigma^\ast$ encoding a Turing machine, simulates its behavior. The input of the machine is a string in the form $<TM, i>$, where $TM$ is an encoded Turing machine, and $i$ is the input to that machine, and outputs $TM(i)$. If the input $d$ is not in the form $<TM, i>$, the machine will simply output $d$. In any case, the machine has to be partial, since it might happen that for some inputs it never halts.

We have to select a particular universal Turing machine and encode its internal working using first-order logic. As we have seen (Theorem \ref{def:Invariance-theorem}) selecting different universal machines would result in different encoding lengths, and that could alter the value of nescience. However, since in this axiomatization we are only interested in the ordering of strings according to their nescience, and not in the actual values, the particular machine selected is not relevant, since that order is preserved. In particular, for the axioms we will use the shortest possible universal machine ({\color{red} Ref}), composed by two states and three tape symbols:

\vskip 0.25cm

\begin{description}
\item[Axiom 12] The machine $\mathcal{U} : \Sigma^\ast \rightarrow \Sigma^\ast$ satisfy the following properties {\color{red} TODO: a universal machine with input, that leaves the input tape unmodified in case of failure.}
\end{description}

\vskip 0.25cm

The relation between the machine $\mathcal{U}$ and the oracle $\mathcal{O}$ is given by the following axiom:

\vskip 0.25cm

\begin{description}
\item[Axiom 13] For all $s \in \Sigma^\ast$ such that $\mathcal{U}(s) \downarrow$ we have that $s \mathcal{O} U(s)$.
\end{description}

\vskip 0.25cm

That is, descriptions are also representations, and they represent the same entity.

\vskip 0.25cm

The relation between the machine $\mathcal{U}$ and the concatenation of strings is given by the following axiom:

\vskip 0.25cm

\begin{description}
\item[Axiom 14] For all $s, t \in \Sigma^\ast$ such that $U(s) \downarrow$ and $U(t) \downarrow$ we have that $U(s \oplus t) = U(s) \oplus U(t)$.
\end{description}

\vskip 0.25cm

That is, we require that descriptions have to be self-delimited, for example, by being prefix-free.

The conditional function $\mid : \Sigma^\ast \times \Sigma^\ast \rightarrow \Sigma^\ast$ is a binary function that satisfy the following properties:

\vskip 0.25cm

\begin{description}
\item[Axiom 15] $\forall s, t \in \Sigma^\ast$ we have that $\mid(s, t) \in \Sigma^\ast$.
\item[Axiom 16] $\forall s \in \Sigma^\ast$ we have that $\mid (s, \lambda) = \mid( \lambda, s) = s$.
\item[Axiom 17] $\forall r, s, t \in \Sigma^\ast$ we have that $\mid (\mid (r, s), t) = \mid (r, \mid (s, t))$.
\end{description}

Since this point we will use the infix notation of the conditional function, that is, we will write $s \mid t$ instead of the $\mid (s, t)$.

The relation between the machine $\mathcal{U}$ and conditional $\mid$ is given by the following axiom:

\vskip 0.25cm

\begin{description}
\item[Axiom 18] For all $r \in \Sigma^\ast$ such that $U(r) \uparrow$ we have that $U(r \mid s) = U(r)$.
\end{description}

\vskip 0.25cm

The relation between the machine $\mathcal{U}$ and the nescience is given by the following axiom:

\begin{description}
\item[Axiom 19] For all $s \in \Sigma^\ast$ such that $U(s) \uparrow$ and $U(s) = t$ we have that $\lnot U(s) =_N t$.
\end{description}

\vskip 0.25cm

The relation of the nescience ordering and the conditional function is given through the application of the machine:

\begin{description}
\item[Axiom 20] Let $s \in \Sigma^\ast$ such that $U(s) \uparrow$, then we have that $U(s \mid t) <_N U(s)$ for all $t \in \Sigma^\ast-\{\lambda\}$.
\end{description}

{\color{red} TODO: Pending, axiom of distributive law}

\begin{proposition}
Let $d \in \mathcal{D}$ be a description such that $N(d)=0$, then $\mathcal{O}(d)=0$.
\end{proposition}
\begin{proof}
Apply the axiom of perfect knowledge.
\end{proof}

We have assumed as axioms the behavior of the nescience function in case of having two related descriptions with the same length and one of them with a smaller inaccuracy than the other ($l(s) = l(t)$ and $\mathcal{O} (s) \leq \mathcal{O} (t)$), the case of two descriptions with the same inaccuracy and one of them shorter than the other ($l(s) \leq l(t)$ and $\mathcal{O} (s) = \mathcal{O} (t)$), and the case of equality ($l(s) = l(t)$ and $\mathcal{O} (s) = \mathcal{O} (t)$). In the next proposition we will consider the rest of the provable cases.

\begin{proposition}
\label{prop:properties_nescience}
Let $s, t \in \mathcal{D}$ such that $s \mathcal{R} t$, then we have that
\begin{enumerate}[label=(\alph*)]
\item if $l(s) < l(t)$ and $\mathcal{O}(s) < \mathcal{O}(t)$ then $N(s) < N(t)$,
\item if $l(s) < l(t)$ and $\mathcal{O}(s) = \mathcal{O}(t)$ then $N(s) < N(t)$,
\item if $l(s) = l(t)$ and $\mathcal{O}(s) < \mathcal{O}(t)$ then $N(s) < N(t)$,
\item if $l(s) = l(t)$ and $\mathcal{O}(s) > \mathcal{O}(t)$ then $N(s) > N(t)$,
\item if $l(s) > l(t)$ and $\mathcal{O}(s) = \mathcal{O}(t)$ then $N(s) > N(t)$, and
\item if $l(s) > l(t)$ and $\mathcal{O}(s) > \mathcal{O}(t)$ then $N(s) < N(t)$.
\end{enumerate}
\end{proposition}
\begin{proof}

Recall that if $a \le b$ we have that $a < b$ or $a = b$, and that $a < b$ implies $b > a$. 

\begin{enumerate}[label=(\alph*)]

\item Apply the axiom of surfeit.

\item Apply the axiom of surfeit.

\item Apply the axiom of inaccuracy. 

\item Interchange $s$ and $t$ and apply (c).

\item Interchange $s$ and $t$ and apply (b).

\item Interchange $s$ and $t$ and apply (a).

\end{enumerate}

\end{proof}

Unfortunately there is not too much we can say from the axioms about the nescience of two related descriptions $s$ and $t$ in case that $l(s) < l(t)$ and $\mathcal{O} (s) > \mathcal{O} (t)$, and in case that $l(s) > l(t)$ and $\mathcal{O} (s) < \mathcal{O} (t)$. 

Next definition formally introduces the concept of scientific methodology in the context of the theory of nescience.

\begin{definition}
A \emph{scientific methodology} is an effective procedure that produces a sequence of $t_1, t_2, \ldots, t_n, \ldots$ where $t_i \in \mathcal{D}$ such that $N(t_i) < N(t_{i+1})$.
\end{definition}

Note that we do not require that in a scientific methodology the collection of $t_i$ refer to the same entity.

% Properties of the Oracle

\subsection{Properties of the Oracle}

{\color{red} TODO: equivalence classes, quotation set, projection function, ...}

Let $S$ be the set of topics in which we are interested, and assume by the moment that $S$ is finite with $k$ elements. The oracle partitions the set $\Sigma_\star$ into $k$ sets such that:
\[
\argmin_S \sum_{i=1}^k \sum_{x \in S_i} NID(x, \mu_i)
\]
where $\mu_i$ is the average topic of class $S_i$.

{\color{red} TODO: explain what $\mu_i$ is, and how to generalize this concept to the case of having infinite topics.}

% Properties of Inaccuracy

\subsection{Properties of Inaccuracy}

{\color{red} TODO: Explain the consequences of requiring to be well-founded.}

% 

\subsection{Type Theory}


%
% Section: Scientific Method
%

\section{Scientific Method}

{\color{red} Describe our proposal of scientific method. Short story: based on a exploration/exploitation approach, where the exploration is based in the concept of joint descriptions, and the exploitation in the concept of conditional description.}

{\color{red} Compare against another proposals of scientific method (inductive-deductive, hypothetico-deductive, etc) and with other techniques for knowledge discovery and creativity (triz, etc.)}

\subsection{Science as a Language}

{\color{red} TODO: Provide an alternative definition of the set of descriptions, and prove that it is equivalent to our definition based on the description function}

Since we require computable descriptions, we would like to know if the set of all possible descriptions of a topic is computable as well.

\begin{proposition}
Study if $D_{t}$ is Turing-decidable, Turing-recognizable or none.
\end{proposition}
\begin{proof}
{\color{red} TODO}
\end{proof}

Although we know that it is not computable, we are interested in the set composed by the shortest possible description of each topic.

\begin{definition}
We define the set of \emph{perfect descriptions}, denoted by $\mathcal{D}^\star$, as:
\[
\mathcal{D}^\star = \{ d_t^\star : t \in \mathcal{T} \}
\]
\end{definition}

Since the set $\mathcal{D}$ includes all the possible descriptions of all the possible (describable) topics, we can see this set as a kind of language for science. We do not call it universal language since it depends on the initial set of entities $\mathcal{E}$ and the particular encoding used for these entities.

\begin{proposition}
The set $\mathcal{D}$ is not Turing-decidable.
\end{proposition}
\begin{proof}
{\color{red} TODO: Because it depends on the set $\mathcal{T}$ that it could be not computable}
\end{proof}

Say something here

\begin{proposition}
The set $\mathcal{D}$ is Turing-recognizable.
\end{proposition}
\begin{proof}
{\color{red} TODO: The same argument as the previous proposition}
\end{proof}

A universal language is determined by a universal Turing machine. Given Two different universal Turing machines $\delta_{a}$ and $\delta_{b}$ defines two different universal languages $\mathcal{L}_{a}$ and $\mathcal{L}_{b}$. Let $\mathcal{L}_{a=b}=\left\{ \left\langle d_{a},d_{b}\right\rangle \,,\,d_{a},d_{b}\in\mathcal{D}\mid\delta_{a}\left(d_{a}\right)=\delta_{b}\left(d_{b}\right)\right\}$.

\begin{proposition}
Study if $\mathcal{L}_{a=b}$ is Turing-decidable, Turing-recognizable or none.
\end{proposition}
\begin{proof}
{\color{red} TODO}
\end{proof}

Let $\mathcal{L}_{\nexists t}$ the laguage of valid descriptions (from a formal point of view) that do not describe any topic, that is, $\mathcal{L}_{\nexists t}=\left\{ d\in\mathcal{D}\mid\nexists t\in\mathcal{T}\,,\,\delta\left(d\right)=t\right\}$.

\begin{proposition}
Study if $\mathcal{L}_{\nexists t}$ is Turing-decidable, Turing-recognizable or none.
\end{proposition}

Let $\mathcal{L}_{\nexists d}$ the laguage topics that are not described by any description, that is, $\mathcal{L}_{\nexists d}=\left\{ t\in\mathcal{T}\mid\nexists d\in\mathcal{D}\,,\,\delta\left(d\right)=t\right\}$.

\begin{proposition}
Study if $\mathcal{L}_{\nexists d}$ is Turing-decidable, Turing-recognizable or none.
\end{proposition}

{\color{red} TODO: Extend the concept of conditional description to multiple topics.}

{\color{red} TODO: Introduce the concept of "independent topics" based on the complexity of the conditional description. Study its properties.}

{\color{red} TODO: Define a topology in $\mathcal{T}$ and $\mathcal{D}$. Study the continuity of $\delta$. Study the invariants under $\delta$.}


%
% Section: The Inaccuracy - Surfeit Trade-off
%

\section{The Inaccuracy - Surfeit Trade-off}

{\color{red} TODO: Explain that given a miscoding, there is a trade-off between inaccuracy and surfeit, in the sense that there exists an optimal point beyond it the more we decrease the inaccuracy, the more we increase the surfeit, and so, the nescience keep constant. Explain how this trade-off imposes a limit to knowledge.}

%
% Section: Science vs. Pseudoscience
%

\section{Science vs. Pseudoscience}

{\color{red} TODO: Provide a characterization of the difference between science and pseudoscience. In science, nescience decreases with time, in pseudoscience not.}

%
% Section: Graspness
%
\section{Graspness}

There are some topics whose nescience decrease with time much faster than others, even if the amount of research effort involved is similar. A possible explaination to this fact is that there are some topics that are inherently more difficult to understand.

We define the \emph{graspness} of a topic as the entropy of the set of possible descriptions of that topic. A high graspness means a difficult to understand topic. Intuitively, a research topic is difficult if it has no descriptions much shorter than the others, that is, descriptions that significantly decrease the nescience. For example, in physics there have been a sucession of theories that produced huge advances in our understanding of how nature works (Aristotelian physics, Newton physics, Einstein physics), meanwhile in case of philosophy of science, new theories (deductivism, inductivism, empirism, falsation, ...) have a similar nescience than previous ones.

\begin{definition}[Graspness]
Let $D_t$ the set of descriptions of a topic $t \in T$. The \emph{graspness} of topic $t$, denoted by $G(t)$, is defined by:
\[
G(t) = \sum_{d \in D_t} 2^{-l(d)} l(d)
\]
\end{definition}

Grapsness is a postive quantity, whose maximum is reached when all the descriptions of the topic have the same length:

\begin{proposition}
Let $D_t = \{ d_1, d_2, \ldots, d_q \}$ the set of descriptions of a topic $t \in T$, then we have that $G(t) \leq \log q$, and $G(t) = \log q$ if, and only if, $l(d_1) = l(d_2) = \ldots = l(d_q)$.
\end{proposition}
\begin{proof}
Replace $P(s_i)$ by $2^{-l(d_i)}$ in Proposition \ref{prop:maximum_entropy}.
\end{proof}

If $G(t) = \log q$ then we have that $N_t = 0$ for all $\hat{d}_t$. In that case, it does not make any sense to do research, since no knew knowledge can be adquired. This is the case of pseudosciences, like astrology, where the nescience of descriptions does not decrease with time. If fact, graspness can be used as a distintive element of what constitutes \emph{science} and what does not.

\begin{definition}
A topic $t \in T$ is \emph{scientable} if $G(t) < \log d(D_t) - \epsilon$, where $\epsilon \in \mathbb{R}$ is a constant.
\end{definition}

We can extend the concept of graspness from topics to areas, for example, by means of computing the average graspness of all the topics included in the area.

\begin{definition}
Let $A \subset T$ a research area. The \emph{graspness} of area $A$, denoted by $G(A)$, is defined by:
\[
G(A) = \frac{1}{d(A)} \sum_{t \in A} G(t)
\]
\end{definition}

%
% Section: Effort
%
\section{Effort}

{\color{red} TODO: Provide a characterization of the effort, measured in terms of number of operations, or time, to reduce the nescience. Based in the concept of computational complexity. Provide a physical interpretation in terms of information.}

%
% Section: Human Understanding
%
\section{Human Understanding}

In his landmark paper \emph{"On Computable Numbers with an Application to the Entscheidungsproblem"}, where the concept of computable function was proposed for the first time, when the author Alan M. Turing talked about computers, he was thinking about human computers, not machines. In this sense, according to Turing, everything that is computable, can be computed by a human, at least in theory.

The fist limitation is about computation time. We expect that a human is able to find a solution to a particular instance of a problem in order of seconds, maybe minutes or hours, but definitely, in less than a life time, otherwise another human have to start again from scratch to solve the problem. So, given the average computing speed of a human brain, we identify as human solvable problems only those problems with a complexity smaller than a fixed number of steps.

The second limitation is about program size. It might happen that the algorithms to solve a problem is simply too big to fit in our brains. Of course, we could argue that {\color{red} as we saw in Example XX, only X states and Y tape symbols} are sufficient to implement a universal Turing machine capable of solving any computable problem, that is, any problem for which exist a Turing machine that can solve it. However, by just following a set of instruction we do not mean that we understand a problem. We understand a problem when we have made the problem for ourselves, in the sense, that the problem is somehow stored in our brain in our own language. So this limits the problem to those whose length, including the necessary background, can be stored in our brain.

We are looking for solutions that minimize at the same time the computational complexity and the Kolmogorov complexity.

\begin{definition}
We say that a problem $P$ is \emph{human solvable} if there exist an algorithm $TM$ to solve $P$ such that $t(n)$ and $K(P) < l$.
\end{definition}

Given the above definition, we could argue that most of the problems we know are not human solvable, but in fact, they have been solved by human. We use two strategies to deal with too complex problems for our individual brains. The fist strategy is that those time-comsuming mechanical parts of the problems are left for computers, so we can reduce the computing time. The second one is that we split the problems into subproblems and each of use specializes in those subproblems.

In this section we are interested in to study the nature of these problems in which this strategy cannot be applied, and so, they are out of reach of individual, nor groups of, humans.

%
% Section: Areas in Decay
%

\section{Areas in Decay}

{\color{red} Provide a model of the decay in the interest of research areas. For example, a good explanatory variable could be the number of interesting questions: the less interesting questions, the more the area is near to its end as a interesting research area, of course, as long as its topics are not used as tools.}

%
% Section: References
%

\section*{References}

{\color{red} Mention the polemic between Hilbert and Fredge given a reference to the book of Mosterin.}


