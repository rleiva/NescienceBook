%
% CHAPTER: Inaccuracy
%

\chapterimage{Train_wreck_at_Montparnasse_1895.pdf} % Chapter heading image

\chapter{Inaccuracy}
\label{chap:Error}

\begin{quote}
\begin{flushright}
\emph{A little inaccuracy sometimes saves\\
tons of explanations.}\\
Saki
\end{flushright}
\end{quote}
\bigskip

In Section \ref{sec:descriptions_models} we defined the concept of description, or model, of an entity as a computer program that, when executed, recreates one of the representations that encode that entity. More specifically, a description $d$ of a representation $r$ for an entity $e$ is a Turing machine that, when interpreted by a universal Turing machine $\delta$, prints out the string $r$. However, since our knowledge about the entity $e$ under study is usually incomplete, the description $\delta(d)$ will transcribe a different string $r'$, which will be similar to $r$, but not equal. In this chapter we are going to study the error induced by bad models, i.e. how close the string $r'$ is to the original string $r$. We refer to this type of error as the inaccuracy of the description $d$.

Inaccuracy is the second element we will use to characterize how well we understand a research entity. The intuition is that the more accurate our model is, the better we know the entity. From a formal point of view, we compute the inaccuracy of a description $d$ as the normalized information distance between the original representation $r$ and the representation $r'$ produced by our description $d$. That is, inaccuracy is masured as the length of the shortest computer program that can fix the incorrect output of our model.

Inaccuracy compares the output of our description with the representation selected to encode the entity. However, this representation could be at the same time an incorrect one, as we have seen in the previous chapter. Inaccuracy is a concept that deals only with the description $d$, and does not take into account the fact that the representation $r$ might have a positive miscoding. Furthermore, despite not requiring the use of the oracle, inaccuracy is a quantity that it is no computable for the general case, so it must be approximated in practice as we will see in Part III of this book.

In this chapter we will introduce formally the concept of inaccuracy and we will study its properties. We will also review how inaccuracy behaves when we use a conditional description of a representation compared to the unconditional one. And finally, we will extend the concept of inaccuracy from individual entities to research areas.

%
% Section: Inaccuracy
%

\section{Inaccuracy}
\label{sec:inaccuracy:inaccuracy}

When studying an entity $e \in \mathcal{E}$ through a representation $r \in \mathcal{R}_e$, it might happen that our candidate description $d$ is not a valid description for $r$, that is, $d \notin \mathcal{D}_r$ (see Definition \ref{def:descriptions_model}). In that case, the universal Turing machine $\delta$, when given as input $d$, will print out a string $r'$ different from the expected string $r$. Intuitively, we can say that $d$ is an inaccurate description of the entity $e$. However, since descriptions describe entities indirectly though representations, our formal definition of the concept of inaccuracy has to be given with respect to the representations in use, not with respect to the original entities, and we should take into account that representations might be themselves wrong (something that has been already addressed with the concept of miscoding). Given the above considerations, we propose the following definition of the concept of inaccurate description.

\begin{definition}
Let $r \in \mathcal{B}^\ast$ be a representation, and $d \in \mathcal{D}$ a description, with $ d = \langle TM, a \rangle$. If $TM(a) = r'$, such that $r \neq r'$, we say that $d$ is an \emph{inaccurate} description for $r$.
\end{definition}

Our candidate description $d$ might not belong to the set of valid descriptions $\mathcal{D}_r$ of $r$ (positive inaccuracy), and the representation $r$ might not belong to the set of valid $\mathcal{R}^\star_e$ representations of $e$ (positive miscoding).

If our description is inaccurate, we would like to have a quantitative measure of how far we are from the right description. In terms of machines, a natural way to define this measure would be by means of computing how difficult is to transform the wrong representation $r'$ produced by our description into the original representation $r$, that is, to compute the normalized information distance between $r'$ and $r$ .

\begin{definition} [Inaccuracy]
\label{def:inaccuracy:inaccuracy:inaccuracy}
Let $r \in \mathcal{B}^\ast$ be a representation, and $d \in \mathcal{D}$ a description, with $d = \langle TM, a \rangle$. We define the \emph{inaccuracy} of the description $d$ for the representation $r$, denoted by $\iota(d, r)$, as:
\[
\iota(d, r) = \frac{ \max\{ K \left(r \mid \delta(d) \right), K \left( \delta(d) \mid r \right) \} } { \max\{ K(r), K \left(\delta(d) \right) \} }
\]
\end{definition}

Having a relative measure of inaccuracy instead of an absolute one allow us to compare the inaccuracy of different descriptions for the same representation, and the inaccuracy of different descriptions for different representations.

Inaccuracy, as it was the case of miscoding (see Definition \ref{def:miscoding}), is computed using a two-way approach: we compute the length of the shortest computer program that can print the correct representation $r$ given the wrong one $r'$, and the other way around, that is, to compute the length of the shortest computer program that can print $r'$ given the string $r$. That is, the representation generated by a valid description has to include all the information required to reconstruct an entity, but it cannot include wrong, or irrelevant, information. 

\begin{example}
Inaccuracy is about how difficult is to fix the output of a description, i.e. the output of computable model, not how difficult is to fix the description itself. If we have a dataset produced by system that can be perfectly described by a quadratic function, and we use as description a linear function, inaccuracy will compare the original quadratic dataset with the linear dataset predicted. Inaccuracy is not about how difficult is to tranform the wrong linear model into the right quadratic one. In this sense, if the orginal dataset has 10 points, a ten degrees perfectly fitted polinomial would have also an inaccuracy of zero. Which model is the best between the zero inaccuracy quadratic and zero accuracy ten degrees polynomial is the subject of the surfeit metric (see Chapter \ref{chap:Redundancy}).
\end{example}

Being based in the concept of Kolmogorov complexity, inaccuracy is a quantity that cannot be computed in practice for the general case, and so, it must be approximated. How to approximate the concept of inaccuracy is something that depends on the characteristics of the entities under study and their representations.

The inaccuracy of a description is, conveniently, a number between $0$ and $1$, as next proposition shows.

\begin{proposition}
\label{prop:inaccuracy:inaccuracy:range}
We have that $0 \leq \iota(d, r) \leq 1$ for all representations $r \in \mathcal{B}^\ast$ and all the descriptions $d \in \mathcal{D}$.
\end{proposition}
\begin{proof}
Given that $0 \leq \frac{ \max\{ K(x \mid y), K(y \mid x) \} } { \max\{ K(x), K(y) \} } \leq 1$ for all $x, y \in \mathcal{B}^\ast$ according to Proposition \ref{prop:ncd_between_zero_and_one}.
\end{proof}

The above proposition holds for all possible descriptions $d$ and all possible representations $r$, even in the case that a description $d$ is not intended as a model for the representation $r$, in which case the inaccuracy would be close to one.

Inaccuracy is equal to zero if, and only if, the description $d$ is one of the possible valid descriptions of the representation $r$.

\begin{proposition}\label{prop:perfect_description}
Let $d \in \mathcal{D}$ be a description for a representation $r \in \mathcal{B}^\ast$, we have that $\iota(d, r) = 0$ if, and only if, $d \in \mathcal{D}_r$.
\end{proposition}
\begin{proof}
If $d \in \mathcal{D}_r$ we have that $K \left( r \mid \delta(d) \right) = K \left( \delta(d) \mid r \right) = 0$ and that $\iota(d, r) = 0$. If $\iota(d, r) = 0$ we have that $\max\{ K \left( r \mid \delta(d) \right) = K \left( \delta(d) \mid r \right) \} = 0$, which implies that $K \left( r \mid \delta(d) \right) = K \left( \delta(d) \mid r \right) = 0$ and that $d \in \mathcal{D}_r$.
\end{proof}

%
% Section: Inaccuracy of Joint Representations
%

\section{Inaccuracy of Joint Representations}

Given two representations $r$ and $s$, we want to know the inaccuracy of the model $d$ when describing the joint representation $rs$. Since we require that $rs$ must be a valid representation, the formalization of the concept of inaccuracy applied to joint representation is straightforward, and it does not require a new definition:
\[
\iota(d, rs) = \frac{ \max\{ K \left(rs \mid \delta(d) \right), K \left( \delta(d) \mid rs \right) \} } { \max\{ K(rs), K \left(\Gamma(d) \right) \} }
\]
As a direct consequence of Proposition \ref{prop:range_miscoding}, if $r, s \in \mathcal{B}^\ast$ are two arbitrary representations and $d \in \mathcal{D}$ is a description, we have that $0 \leq \iota(d, rs) \leq 1$.

{\color{red} TODO: Recall the properties of NID, and particularize for the case of inaccuracy of joint representations.}

%
% Section: Inaccuracy of Conditional Descriptions
%

\section{Conditional Inaccuracy}

In this section we are going to extend the concept of inaccuracy from descriptions to conditional descriptions, that is, the inaccuracy of a description assuming the existence of some background knowledge, what we call conditional inaccuracy.

We have to start by defining what we mean when we say that a conditional description is inaccurate.

\begin{definition}
Let $r \in \mathcal{B}^\ast$ be a representation, and $d_{r \mid s} = \langle d, s \rangle$ a conditional description of $r$ given the string $s \in \mathcal{B}^\ast$, with $ d = \langle TM, a \rangle$. If $TM \left(\langle a, s \rangle \right) = r'$, such that $r \neq r'$, we say that $d_{r \mid s}$ is an \emph{inaccurate conditional description} for $r$.
\end{definition}

In the same way we defined the concept of inaccuracy of a description in Definition \ref{def:inaccuracy:inaccuracy:inaccuracy}, we can define the concept of conditional inaccuracy to characterize the error made when using an inaccurate conditional description.

\begin{definition}
Let $r \in \mathcal{B}^\ast$ be a representation, $s \in \mathcal{B}^\ast$ a string, and $d_{r \mid s} = \langle d, s \rangle$ a inaccurate condtional description. We define the \emph{conditional inaccuracy} of the description $d_{r \mid s}$ for the representation $r$ given the string $s$, denoted by $\iota(d_{r \mid s})$, as:
\[
\iota(d_{r \mid s}) = \frac{ \max\{ K \left(r \mid \delta(\langle d, s \rangle) \right), K \left( \delta(\langle d, s \rangle) \mid r \right) \} } { \max\{ K(r), K \left(\delta(\langle d, s \rangle) \right) \} }
\]
\end{definition}

The conditional inaccuracy of a description is a number between $0$ and $1$.

\begin{proposition}
\label{prop:range_conditional_inaccuracy}
Let $r \in \mathcal{B}^\ast$ be a representation, $s \in \mathcal{B}^\ast$ a string, and $d_{r \mid s} = \langle d, s \rangle$ a inaccurate condtional description. We have that $0 \leq \iota(d_{r \mid s}) \leq 1$.
\end{proposition}
\begin{proof}
Given that $0 \leq \frac{ \max\{ K(x \mid y), K(y \mid x) \} } { \max\{ K(x), K(y) \} } \leq 1$ for all $x, y \in \mathcal{B}^\ast$ according to Proposition \ref{prop:ncd_between_zero_and_one}.
\end{proof}

Conditional inaccuracy is equal to zero if, and only if, the description $d$ is one of the possible valid descriptions of the representation $r$.

\begin{proposition}\label{prop:perfect_description}
Let $r \in \mathcal{B}^\ast$ be a representation, $s \in \mathcal{B}^\ast$ a string, and $d_{r \mid s} = \langle d, s \rangle$ a condtional description, with $ d = \langle TM, a \rangle$. We have that $\iota(d_{r \mid s}) = 0$ if, and only if, $TM \left(\langle a, s \rangle \right) = r$.
\end{proposition}
\begin{proof}
If $TM \left(\langle a, s \rangle \right) = r$ we have that $K \left( r \mid \delta(d_{r \mid s}) \right) = K \left( \delta(d_{r \mid s}) \mid r \right) = 0$ and that $\iota(d_{r \mid s}) = 0$. If $\iota(d_{r \mid s}) = 0$ we have that $\max\{ K \left( r \mid \delta(d_{r \mid s}) \right) = K \left( \delta(d_{r \mid s}) \mid r \right) \} = 0$, which implies that $K \left( r \mid \delta(d_{r \mid s}) \right) = K \left( \delta(d_{r \mid s}) \mid r \right) = 0$ and that $TM \left(\langle a, s \rangle \right) = r$.
\end{proof}

Given two representations $r$ and $s$, we want to know the inaccuracy of a conditional model $d$ geniven $t$ when describing the joint representation $rs$. Since we require that $rs$ must be a valid representation, the formalization of the concept of contional inaccuracy applied to joint representation is straightforward, and it does not require a new definition:
\[
\iota(d_{rs \mid t}) = \frac{ \max\{ K \left(r \mid \delta(\langle d, t \rangle) \right), K \left( \delta(\langle d, t \rangle) \mid r \right) \} } { \max\{ K(rs), K \left(\delta(\langle d, t \rangle) \right) \} }
\]
As a direct consequence of Proposition \ref{prop:range_conditional_inaccuracy}, if $r, s \in \mathcal{B}^\ast$ are two arbitrary representations, $d_{r \mid s} = \langle d, s \rangle$ is a condtional description and $t \in \mathcal{B}^\ast$ an arbitrary string, we have that $0 \leq \iota(d_{rs \mid t}) \leq 1$.

%
% Section: Inaccuracy of Areas
%

\section{Inaccuracy of Areas}

The concept of conditional inaccuracy can be extended to research areas in order to quantitative measure the amount of effort required to fix an inaccurate description of the area assuming some already existing background knowledge.

{\color{red} TODO: review notation}

\begin{definition}
Let $\mathcal{A} \subset \mathcal{E}$ be an area with known subset $\hat{\mathcal{A}} = \{r_1, r_2, \ldots, r_n\}$, $s \in \mathcal{B}^\ast$ a string, and $d_{\hat{\mathcal{A}} \mid s}$ a condtional description. We define the \emph{inaccuracy of the area} $d_{\hat{\mathcal{A}}}$ as:
\[
\iota(d_{\hat{\mathcal{A}} \mid s}) = \frac{ \max\{ K \left( \langle r_1, r_2, \ldots, r_n \rangle \mid \delta(\langle d, t \rangle) \right), K \left( \delta(\langle d, t \rangle) \mid \langle r_1, r_2, \ldots, r_n \rangle \right) \} } { \max\{ K(\langle r_1, r_2, \ldots, r_n \rangle), K \left(\delta(\langle d, t \rangle) \right) \} }
\]
\end{definition}

{\color{red} TODO: Recall the properties of areas, and particularize for the case of inaccuracy.}

%
% Section: References
%

\section*{References}

A good introduction to the study of uncertaintines (error analysis in models) in science, and in particular in physics, chemistry, and engineering, is the best-selling text \cite{taylor2022introduction}, which also features the same image of a crashed train than in the introduction to this chapter.

{\color{red} TODO: Add more references.}

