%
% CHAPTER: Miscoding
%

\chapterimage{Escher.pdf} % Chapter heading image

\chapter{Miscoding}
\label{chap:Miscoding}

\begin{quote}
    \begin{flushright}
        \emph{All great work is the fruit of patience and perseverance,\\
            combined with tenacious concentration on a subject\\
            over a period of months or years.}\\
        Santiago Ramón y Cajal
    \end{flushright}
\end{quote}
\bigskip

In many areas of science, the things we want to study - the set of entities or $\mathcal{E}$ - often include abstract ideas or complicated things that are tough to investigate directly. For example, these could be concepts or phenomena that are difficult to describe or model. To understand such entities, we need to use a kind of roundabout method. As we have seen in the previous chapter, the theory of nescience proposes we use representations (sequences of symbols) instead of directly interacting with the actual entities. But this comes with its own issues: because our understanding of the elements of $\mathcal{E}$ is incomplete (otherwise we would not need to keep researching), the representations we use are usually imperfect - they are not as comprehensive or as precise as we would ideally want them to be. These imperfections can cause big problems because inaccuracies in how we represent something can lead to major errors in the model we use to describe it, which can then mess up our interpretation and understanding. It is really important to be aware of this potential source of error and get a handle on its implications.

To help with this, we introduce \emph{miscoding}, a measure we use to size up the errors that come from misrepresenting or inaccurately encoding these things. We define miscoding based on the length of the shortest computer program that can correct a wrong representation to make it right. Basically, miscoding measures the 'effort' (measured by the length of the program) needed to fix a wrong representation. But, there is a catch: putting this idea into practice is not straightforward, because the perfect representations of things are unknown, so we can't measure the gap or difference between our current representation and a perfect one. Even so, we suggest a theoretical approach based on the oracle machine (an abstract idea) that is used to define the set $\mathcal{E}$. This oracle machine can recognize valid representations for all entities. But, we need to be aware of some limitations about the questions we can ask the oracle. For example, we can't ask the oracle about a specific entity for which we do not have a valid representation.

Getting a better grasp on scientific representation and the challenges in reducing miscoding could really push forward scientific research, helping us create better, more thorough models of the natural world. In this chapter, we are going to properly introduce the idea of miscoding and look into its various characteristics. We will examine how miscoding behaves when it comes to combined representations and discuss methods to reduce miscoding. We will delve into the idea that there are strings that do not represent any entity. And we will dig into why miscoding is important in different research areas and talk about its potential to open up new lines of inquiry. 

%
% Section: Miscoding
%
\section{Miscoding}
\label{sec:miscoding}

In an ideal situation, we would ask the oracle to measure how far is a particular string $r$ from perfectly encoding an entity $e$. This creates a complication because we can only inform the oracle about entity $e$ through the use of a valid representation from the set $\mathcal{R}^\star_e$. However, we typically do not have access to these perfect representations, and thus, cannot use them in our query. To address this impediment, we propose an alternative approach. Instead of asking the oracle about the discrepancy between our representation $r$ and a valid representation of the target entity $e$, we propose to ask the oracle about the smallest distance between our current representation $r$ and all the valid representations of all entities encompassed within the set $\mathcal{E}$, that is, the elements of the set $\mathcal{R}^\star_\mathcal{E}$ . As we have discussed in detail in Section \ref{sec:invalid_representations}, this alternative form of query is one that the oracle can theoretically handle and provide answers to. 

\begin{definition} [Miscoding]\index{Miscoding}
\label{def:miscoding}
Let $r \in \mathcal{B}^\ast$ be a representation. The \emph{miscoding} of $r$, denoted by $\mu(r)$, is defined as:
\[
\mu(r) = \overset{o}{ \underset{ r^\star_e \in \mathcal{R}^\star_\mathcal{E} } \min} \frac{ \max\{ K \left( r \mid r^\star_e \right), K \left( r^\star_e \mid r \right) \} } { \max\{ K \left( r \right), K \left( r^\star_e \right) \} }
\]
where $\overset{o}{\min}$ denotes that the minimum has to be computed by an oracle.
\end{definition}

Intuitively, our ignorance regarding an entity corresponds to a higher miscoding value of our current representation. Conversely, gaining a deeper understanding of an entity should allow for an encoding that is closer to a valid representation.

Miscoding is computed through a bidirectional approach: we require the oracle to compute the length of the shortest computer program capable of generating the string $r^\star_e$ given our representation $r$, and vice versa – that is, to compute the length of the shortest computer program capable of generating $r$ given the string $r^\star_e$. As expected, a high-quality representation should contain all the necessary information to reconstruct an entity, devoid of any erroneous or superfluous data. Miscoding involves the inclusion of only relevant symbols, while surfeit (as discussed in Chapter \ref{chap:Redundancy}) pertains to the inclusion of only necessary symbols.

Our definition of miscoding employs a relative measure as opposed to an absolute one. This choice allows us to compare the miscoding of various encodings for the same entity, as well as to compare the miscoding across different entities.

The miscoding value of a representation $r$ consistently falls between $0$ and $1$.

\begin{proposition}
\label{prop:range_miscoding}
For all $r \in \mathcal{B}^\ast$, it holds that $0 \leq \mu(r) \leq 1$.
\end{proposition}
\begin{proof}
This is ensured by $0 \leq \frac{ \max\{ K(x \mid y), K(y \mid x) \} } { \max\{ K(x), K(y) \} } \leq 1$ for all $x, y \in \mathcal{B}^\ast$ as stated in Proposition \ref{prop:ncd_between_zero_and_one}.
\end{proof}

Miscoding is zero if, and only if, the representation $r$ is a valid representation of a particular entity $e$.

\begin{proposition}
\label{prop:perfect_encoding}
Given a representation $r \in \mathcal{B}^\ast$, we have $\mu(r) = 0$ if, and only if, $r \in \mathcal{R}^\star_\mathcal{E}$.
\end{proposition}
\begin{proof}
If $r \in \mathcal{R}^\star_\mathcal{E}$, there exists an entity $e \in \mathcal{E}$ such that $r = r^\star_e$, thus, $K \left( r \mid r^\star_e \right) = 0$ and $\mu(r) = 0$. If $\mu(r) = 0$, there must exist an $r^\star_e \in \mathcal{R}^\star_\mathcal{E}$ such that $K \left( r \mid r^\star_e \right) = 0$, implying that $r = r^\star_e$ and $r \in \mathcal{R}^\star_\mathcal{E}$.
\end{proof}

According to Proposition \ref{prop:perfect_encoding}, if the miscoding of $r$ equals 0, it can be concluded that $r$ perfectly encodes an entity $e$. The challenge lies in determining the exact entity encoded by $r$. Although scientific intuition may guide guesses regarding the encoded entity, it is not mathematically possible to substantiate these conjectures. Furthermore, with the progress of research, perceptions of the encoded entity's nature may evolve over time (see Example \ref{ex:polywater}).

It has been observed that an entity $e \in \mathcal{E}$ may have multiple valid representations, as denoted by the set $\mathcal{R}^\star_e$. Fortuitously, miscoding is a value that remains unaffected by the chosen representation (refer to the issue of style in Section \ref{sec:scientific_representation}).

\begin{proposition}
For any valid representation $r^\star_e \in \mathcal{R}^\star_\mathcal{E}$, it holds that $\mu\left( r^\star_e \right) \leq \mu\left( r \right)$ for all $r \in \mathcal{B}^\star$.
\end{proposition}
\begin{proof}
This follows from $\mu\left( r^\star_e \right) = 0$ and $\mu\left( r \right) \geq 0$ for all $r \in \mathcal{B}^\star$.
\end{proof}

For a given entity $e$, all valid representations encompassed by $\mathcal{R}^\star_e$ exhibit an equivalent degree of adequacy in terms of miscoding, given that each maintains a miscoding value of $0$. Pragmatically, the representation that facilitates the most efficient accumulation of new knowledge about the entity in question should be selected, specifically the one that enables the derivation of models explicating the entity.

%
% Section: Miscoding of Joint Representations
%

\section{Joint Miscoding}
\label{sec:joint_miscoding}

Section \ref{sec:descriptions_joint_topic} introduced the notion of 'joint representation'\index{Joint representation}, which arises when two representations, $s, t \in \mathcal{B}^\ast$, are concatenated to form a new string $st$. This section delves into the properties of miscoding intrinsic to these joint representations.

Given that the joint representation $st$ qualifies as a representation, it is not necessary to introduce a new definition of miscoding to capture this concept. The miscoding of a joint representation is given by:
\[
\mu(st) = \overset{o}{ \underset{ r^\star_e \in \mathcal{R}^\star_\mathcal{E} } \min} \frac{ \max\{ K \left( st \mid r^\star_e \right), K \left( r^\star_e \mid st \right) \} } { \max\{ K \left( st \right), K \left( r^\star_e \right) \} }
\]
In alignment with Proposition \ref{prop:range_miscoding}, the miscoding of a joint representation is a value between 0 and 1, expressed as $0 \leq \mu(st) \leq 1$.

It is worth noting that the process of supplementing an incomplete representation — a representation with a positive miscoding — with additional symbols does not unequivocally guarantee a reduction in miscoding. This is because the appended symbols may introduce errors. Conversely, it is also not certain that the miscoding will increase, as the integration of pertinent symbols can potentially decrease the miscoding of an incomplete representation. From a formal point of view, given two arbitrary representations $s, t \in \mathcal{B}^\ast$, there is no guarantee that any of the following relationships will consistently hold: $\mu(ts) \geq \mu(t)$, $\mu(ts) \leq \mu(t)$, $\mu(ts) \geq \mu(s)$, nor $\mu(ts) \leq \mu(s)$.

\begin{example}
Consider the text $r$ of a biology research article that succinctly and accurately describes a newly discovered specimen, resulting in a low miscoding value $\mu(r)$. However, when additional text $s$ from a completely unrelated article on a second specimen from a different species is appended, the new concatenated representation $rs$ muddles the original focus. This inclusion of unrelated content increases the miscoding value $\mu(ts) = 0.6$, proving the case where $\mu(ts) \geq \mu(t)$. The added text, though scientific, dilutes the clarity and precision of the original article, causing a rise in miscoding.

Conversely, in a second scenario, another biological research article $r'$ has a relatively high miscoding value of $\mu(r')$, stemming from an incomplete presentation of the specimen. However, the concatenation of an additional text $s'$, rich with the previously omitted essential details, ameliorates the article. The concatenated version $ts$ thus sees a significant reduction in miscoding. This exemplifies the case of $\mu(ts) \leq \mu(t)$ and underscores the premise that the enrichment of an incomplete representation with pertinent, focused information can lead to a more accurate, reliable, and lower miscoding value.
\end{example}

Furthermore, it should not be assumed that the miscoding of a joint representation is less than the sum of the miscoding of the individual representations, i.e., $\mu( st ) \leq \mu(s) + \mu(t)$. This holds true even when both representations $t$ and $s$ encode the same entity, as the joint representation $ts$ might encode an entirely distinct entity compared to $t$ and $s$.

\begin{example}
In a medical research context, representation $r$ describes a new drug compound $A$ for treating a specific type of cancer, while representation $s$ outlines a genetic mutation $B$ associated with that cancer type. Both $r$ and $s$ have low miscoding values, denoting accurate, focused content. However, when concatenated, the combined representation $rs$ unintentionally suggests a third entity $C$, implying a causal relationship between $A$ and $B$. This unintended implication stems from the mixture of distinct information, leading to a higher miscoding value. In this instance, each of $r$, $s$, and $rs$ essentially represents different research entities. $r$ and $s$ are clear in their individual contexts, but $rs$ is ambiguous, exemplifying a case where concatenated information can inadvertently create a representation of an entirely different, unintended entity, thus increasing miscoding.
\end{example}

Given the non-commutative nature of joining two representations, it is not guaranteed that $\mu(ts) = \mu(st)$. This is because the composite strings $ts$ and $st$ could potentially encode two different entities. This property hold true even when we restrict our focus to the valid representations of a specific entity $e$. For instance, if $r^\star_e, s^\star_e \in \mathcal{R}^\star_{e}$ denote two valid representations of the entity $e$, there is no guarantee that the concatenated string $r^\star_e s^\star_e$ will maintain its status as a representation of $e$.

\begin{example}
In environmental science, representation $r$ thoroughly examines microplastics, their nature, and detrimental impacts on marine ecosystems, while representation $s$ outlines specific technologies and methodologies for detecting these pollutants in the water. The $rs$ concatenation offers a logical narrative from the identification of microplastics and their effects to the methodologies of detection, forming a coherent entity that serves as a comprehensive resource on the pollutant, its impacts, and detection methods, and which is not fully understood. Conversely, $sr$ starts with the technical and methodological aspects of detecting microplastics and then moves to describe the pollutants and their effects, forming another clear entity that can serve as a methodological guide with a detailed context of application, with low miscoding. In this particular case we have that $\mu(rs) > \mu(sr)$
\end{example}

The concept of miscoding can be generalized to apply to joint representations of any finite collection of representations. If $r_1, r_2, \ldots, r_n \in \mathcal{B}^\ast$ comprise a finite collection of representations, the miscoding of the joint representation $r_1 r_2, \ldots r_n$, can be expressed as:
\[
\mu(r_1 r_2 \ldots r_n) = \overset{o}{ \underset{ r^\star_e \in \mathcal{R}^\star_\mathcal{E} } \min} \frac{ \max\{ K \left( r_1 r_2 \ldots r_n \mid r^\star_e \right), K \left( r^\star_e \mid r_1 r_2 \ldots r_n \right) \} } { \max\{ K \left( r_1 r_2 \ldots r_n \right), K \left( r^\star_e \right) \} }
\]
Consistent with the case of joining two representations, we maintain that $0 \leq \mu(r_1, r_2, \ldots, r_n) \leq 1$. It cannot be assured that $\mu(r_1, r_2, \ldots, r_n) \leq \mu(r_i)$ or $\mu(r_1, r_2, \ldots, r_n) \geq \mu(r_i)$ for $i = 1, \ldots, n$. Additionally, it is not necessarily true that $\mu(r_1, r_2, \ldots, r_n) \leq \mu(r_1) + \ldots + \mu(r_n)$. Lastly, it is not possible to make any assertions regarding the miscoding of a permutation of the representations incorporated in the joint representation compared to the original miscoding.

%
% Section: Reducing Miscoding
%

\section{Decreasing Miscoding}

A valid representation of an entity refers to a string that encapsulates all the vital information needed by the oracle to reproduce the given entity, and solely that information. Conversely, an invalid representation—signified by a miscoding exceeding zero—may stem from either a lack of critical information, the presence of incorrect symbols, or the inclusion of irrelevant symbols. To decrease the miscoding of a representation, one can supplement missing information or eliminate incorrect or non-pertinent symbols. Determining in advance whether any information is absent or if certain symbols require removal is not possible. Nonetheless, as the ensuing theorem demonstrates, one of these scenarios must inevitably be true.

\begin{theorem}
\label{th:reduce_miscoding}
Let $r \in \mathcal{B}^\ast$ be a representation such that $\mu(r) >0$, then one or both of the following conditions must hold:
\begin{enumerate}[label=(\roman*)]
\item There exists a $s \in \mathcal{B}^\ast$ such that $\mu(rs) < \mu(r)$ or $\mu(sr) < \mu(r)$,
\item There exists an $s \in \mathcal{B}^\ast$ in the form $r = \alpha s \beta$ with $\alpha, \beta \in \mathcal{B}^\ast$ such that $\mu(s) < \mu(r)$.
\end{enumerate}
\end{theorem}
\begin{proof}
{\color{red} To be completed}
Assuming that $\mu(r) >0$, we find that
\[
\overset{o}{ \underset{ r^\star_e \in \mathcal{R}^\star_\mathcal{E} } \min} \frac{ \max\{ K \left( r \mid r^\star_e \right), K \left( r^\star_e \mid r \right) \} } { \max\{ K \left( r \right), K \left( r^\star_e \right) \} } > 0
\]
Let $r^\star_e = \text{arg}\,\min \left( \mu(r) \right)$. Hence, $\max\{ K \left( r \mid r^\star_e \right), K \left( r^\star_e \mid r \right) \} > 0$. If $K \left( r \mid r^\star_e \right) > 0$, it indicates that $r$ includes non-relevant symbols. Conversely, if $K \left( r^\star_e \mid r \right) > 0$, it implies that $r$ omits some relevant symbols.
\end{proof}

\begin{example}
\label{ex:lung_cancer}
Suppose that the research entity $e$ in which we are interested is the causes of lung cancer. In order to understand this entity, we have measured a collection of risk factors in a random sample of the population (smooking, exercise, diet, age, etc.). However, due to a problem with the sampling procedure, all the samples correspond to a subset of the population, for example, males. This dataset would be a representation $s$ for our entity $e$, but a very bad one, since it is strongly biased. If we have a second representation, corresponding the sample data of females $t$, the joint representation $st$ will be a better one that any of them, $s$ or $t$, isolated.
\end{example}

Next example describes a case in which removing some symbols from a representation can decrease its miscoding.

\begin{example}
A historian compiles representation $r$, aiming to narrate the events surrounding the signing of a pivotal treaty. However, the inclusion of speculative statements and personal interpretations about the involved parties’ intentions, not backed by primary sources, leads to a high miscoding value. The document, while rich in information, is clouded by content that is erroneous, making it an invalid representation of the historical event. By removing these speculative and unverified segments, a refined version $r’$ emerges, offering a concise, factual account based solely on verifiable data and primary sources. This refinement leads to a decrease in the miscoding value, transforming $r’$ into a more valid representation that accurately reflects the historical event without the noise of uncorroborated interpretations.
\end{example}

%
% Section: Targetless Representations
%

\section{Targetless Representations}

In the most extreme scenario, the miscoding value could escalate to its maximum limit of 1. This situation transpires when our prevailing representation $r$ does not contain a single symbol that corresponds to the encoding of any entity within the set of entities $\mathcal{E}$. In this case, $r$ would be a targetless representation\index{Targetless representation}, an abstract construct without a concrete or meaningful interpretation. We could assign ramdomly the targetless representations to an entity, but that would violate the highly desirable property of subgetive reasoning.

Next proposition formalices this idea.

\begin{proposition}
\label{prop:miscoding_upper_bound}
Given a representation $r \in \mathcal{B}^\ast$, the miscoding $\mu(r)$ reaches its maximum value of 1 if and only if there is no symbol in $r$ that contributes to the encoding of any of the entities of $\mathcal{E}$.
\end{proposition}
\begin{proof}
Assume that $r$ does not contain any symbol that contributes to the encoding of $e$. Let $r^\star_e$ be any valid representation of $e$. Then the shortest program capable of generating $r^\star_e$ from $r$ would need to append every symbol in $r^\star_e$ to $r$, effectively resulting in $r^\star_e$. Hence, $K(r^\star_e | r) = K(r^\star_e)$, indicating no symbolic overlap between $r$ and $r^\star_e$.

Following the definition of miscoding, we have $\mu(r) = \max{ K(r | r^\star_e), K(r^\star_e | r) } / \max{K(r), K(r^\star_e)}$. Because $K(r | r^\star_e) = K(r)$ and $K(r^\star_e | r) = K(r^\star_e)$, we get $\mu(r) = 1$, assuming that $K(r) \neq 0$.

The converse also holds. If $\mu(r) = 1$, this means that there's no information in $r$ that assists in generating $r^\star_e$ or vice versa, which implies that $r$ does not contain any symbol that contributes to the encoding of $e$.
\end{proof}

For every finite or countable infinite set of entities $\mathcal{E}$, there exists a infinite number of targetless represention. We will prove this property using a constructive method. That is, we will prove that there exists at least one targetless representation and define a method to construct more targetless representations given the original one.

\begin{proposition}
For every finite or countably infinite set of entities $\mathcal{E}$, there exists an uncountably infinite set of targetless representations.
\end{proposition}
\begin{proof}
Let $\mathcal{R}$ denote the set of all possible representations for entities in $\mathcal{E}$, such that for each entity $e \in \mathcal{E}$, there exists a unique representation $r \in \mathcal{R}$ and vice versa. Since $\mathcal{E}$ is finite or countably infinite, so is $\mathcal{R}$.

Consider the set $\mathcal{B}^\ast$ of all possible binary strings of finite length. This set is uncountably infinite, because it includes all possible sequences of two symbols ('0' and '1') of finite length.

Since $\mathcal{R}$ is countable (either finite or countably infinite), and $\mathcal{B}^\ast$ is uncountably infinite, there must exist some strings in $\mathcal{B}^\ast$ that are not in $\mathcal{R}$. We can select any such string and call it $r_0$. By definition, $r_0$ is a targetless representation, as it does not correspond to any entity in $\mathcal{E}$.

Given a targetless representation $r_i$, we can construct a new targetless representation $r_{i+1}$ by appending either a '0' or a '1' to $r_i$. Since $r_i$ does not correspond to any entity in $\mathcal{E}$, $r_{i+1}$, being a lengthened version of $r_i$, also cannot correspond to any entity in $\mathcal{E}$. Therefore, $r_{i+1}$ is also a targetless representation.

By repeating this process, we can construct an infinite sequence of targetless representations $\{r_0, r_1, r_2, \ldots\}$, demonstrating that there exists an uncountably infinite set of targetless representations for any finite or countably infinite set of entities $\mathcal{E}$. 
\end{proof}

What it is a targetless representation for an oracle is not necesarily a target representation for another oracle, as next example shows.

\begin{example}
Imagine two machines, A and B, each controlling a robotic arm adept at crafting nuts and bolts. Machine A operates on a low-level assembly language, while Machine B utilizes a more sophisticated high-level programming language. Consequently, a particular set of instructions that proves inadequate and fails to yield a finished product with Machine A could be perfectly executed by Machine B, resulting in a completed bolt.
\end{example}

The normalized compression distance between a targetless representation and the closest non-targetless representation might be less that one, meaning that the target-less representation contains some information about the non-targetless one. However, this is not sufficient to guarantee scientific progress, given the size of the Turing machine needed to benefit from this knowledge.

%
% Section: Miscoding of Areas
%
\section{Miscoding of Areas}
\label{sec:miscoding_areas}

The concept of miscoding can be extended to research areas in order to quantitative measure the amount of effort required to fix an inaccurate representation of an area. Unfortunately, as we have discussed in Section \ref{sec:areas}, there is no way to check if the strings included in a $n$-fold representation correspond to entities of that area, or to avoid the fact there may be instances where some of these strings represent the same entity.

Let's consider the strings $r_1, r_2, \ldots, r_n$, where each $r_i \in \mathcal{B}^\ast$ for $i=1, 2, \ldots, n$. Recall that when we write $r_1 r_2 \ldots r_n$, we refer to the concatenation of these strings. This operation could potentially meld the individual strings in such a way that they cannot be separated back into their original components. Conversely, by $\langle r_1, r_2, \ldots, r_n \rangle$ we denote a re-encoding of the individual strings $r_i$ into a single string, but with the provision that it retains the ability to decompose into the original strings. Moreover, the joint representation $r_1 r_2 \ldots r_n$ encodes a single entity, while the $n$-folded representation $\langle r_1, r_2, \ldots, r_n \rangle$ might encode to up to $n$ distinct entities.

The following defintion extend the concept of miscoding to $n$-fold representations.

\begin{definition}
Let $R = \left( r_1, r_2, \ldots, r_n \right) \in \mathcal{B}^\ast \times \mathcal{B}^\ast \times \overset{n} \ldots \times \mathcal{B}^\ast$  be an $n$-fold representation. We define the \emph{miscoding} of $R$, denoted by $\mu(R)$, as:
\[
\mu(R) = \overset{o}{ \underset{ (r^\star_{e_1}, \ldots, r^\star_{e_n}) \in \mathcal{R}^\star_\mathcal{E} \times \overset{n} \ldots \times \mathcal{R}^\star_\mathcal{E} } \min} \frac{ \max\{ K \left( \langle r_1, \ldots, r_n \rangle \mid \langle r^\star_{e_1}, \ldots, r^\star_{e_n} \rangle \right), K \left( \langle  r^\star_{e_1}, \ldots, r^\star_{e_n} \rangle \mid \langle r_1, \ldots, r_n \rangle \right) \} } { \max\{ K \left( \langle r_1, \ldots, r_n \rangle \right), K \left( \langle r^\star_{e_1}, \ldots, r^\star_{e_n} \rangle \right) \} }
\]
where $\overset{o}{\min}$ denotes that the minimum has to be computed by an oracle.
\end{definition}

The miscoding of the representation of an area falls within the range of $0$ and $1$, as illustrated by the following proposition.

\begin{proposition}
\label{prop:inaccuracy:inaccuracy:range}
For all known subsets $R = \left( r_1, r_2, \ldots, r_n \right) \in \mathcal{B}^\ast \times \mathcal{B}^\ast \times \overset{n} \ldots \times \mathcal{B}^\ast$, we have that $0 \leq \mu(R) \leq 1$.
\end{proposition}
\begin{proof}
Given that $\langle r_1, r_2, \ldots, r_n \rangle$ is a string and Proposition \ref{prop:ncd_between_zero_and_one}.
\end{proof}

By extending the concept of miscoding to encompass areas, we can quantitatively evaluate the quality of representations for specific subsets of entities. This mathematical framework offers a robust tool to assess and rectify inaccuracies in both individual entities and broader research areas.


%
% Section: References
%

\section*{References}

Misrepresentation or inaccuracies in scientific representation carry substantial implications for scientific discovery, technological advancement, policymaking, among others. However, no book or paper explicitly addresses the subject of "incorrect representations in science," from the perspective adopted in this book, in which we dissect the issue of scientific representation into two complementary subproblems: representation of entities and description of representations. Despite the absence of a direct focus, a variety of researchers have ventured into this domain indirectly. Their discussions often touch on concerns such as scientific fraud, the replication crisis, and the usage of incorrect models, to name a few. 

