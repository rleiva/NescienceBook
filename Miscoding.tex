%
% CHAPTER: Miscoding
%

\chapterimage{Escher.pdf} % Chapter heading image

\chapter{Miscoding}
\label{chap:Miscoding}

\begin{quote}
    \begin{flushright}
        \emph{All great work is the fruit of patience and perseverance,\\
            combined with tenacious concentration on a subject\\
            over a period of months or years.}\\
        Santiago Ramón y Cajal
    \end{flushright}
\end{quote}
\bigskip

In most scientific disciplines, the set $\mathcal{E}$ of entities under consideration will be composed by abstract elements, or other kind of complex objects, that cannot be studied directly. If we want to understand them we have to use an indirect method. As we have seen in the previous chapter, the approach proposed by the theory of nescience is to work with representations (i.e. strings of symbols) instead of using the original entities. Unfortunately, proceeding in this way introduce new problems: since we do not fully understand the elements of $\mathcal{E}$, otherwise we will not be doing research, the representations used probably will not be as complete and accurate as they should be. This limitation has serious implications, since an error in the representation of an entity will induce an error in the model we use to describe that entity. It is of utmost importance to characterize this type error, and to understand its implications.

Miscoding is a quantity that measures the error due to the use of bad encodings for entities. We propose a definition of miscoding based on the length of the shortest computer program that can print a correct representation given an incorrect one. Intuitively, miscoding quantifies the effort (measured as the length of a program) required to fix an incorrect representation. In practice, since the ideal representations of the entities are unknown, otherwise we would be using them, we cannot compute how far our current representation is from a perfect one. From a theoretical point of view we can take advantage of the oracle machine used to characterize the set $\mathcal{E}$, since that (abstract) machine knows the valid representations for all the entities. However, there are some limitations with respect to the kind of questions we can ask to the oracle that we have to take into account. For example, we cannot query the oracle about a particular entity for which we do not have a valid representation.

In this chapter we will formally introduce the concept of miscoding and study its properties. We will also see how miscoding behaves when dealing with joint representations, and what we can do to drecrease the miscoding. Finally, we will study how miscoding relates to research areas, and how it can be used to discover new research topics.

%
% Section: Miscoding
%
\section{Miscoding}
\label{sec:miscoding}

It would be ideal to query the oracle how far a particular string $r$ is from perfectly encoding the entity $e$ of interest. Regrettably, this is not feasible as informing the oracle about the target entity requires the use of a valid representation $r^\star_e$, which are unknown. To circumvent this issue, we suggest asking the oracle about the shortest distance between $r$ and all the entities of $\mathcal{E}$.  As discussed in Section \ref{sec:invalid_representations}, this is a query that the oracle can (theoretically) answer.

\begin{definition} [Miscoding]
\label{def:miscoding}
Let $r \in \mathcal{B}^\ast$ be a representation. We define the \emph{miscoding} of $r$, denoted by $\mu(r)$, as:
\[
\mu(r) = \overset{o}{ \underset{ r^\star_e \in \mathcal{R}^\star_\mathcal{E} } \min} \frac{ \max\{ K \left( r \mid r^\star_e \right), K \left( r^\star_e \mid r \right) \} } { \max\{ K \left( r \right), K \left( r^\star_e \right) \} }
\]
\end{definition}

In the above definition, the minimum $\overset{o}{ \underset{ r^\star_e \in \mathcal{R}^\star_\mathcal{E} } \min}$ has to be computed by the oracle. Intuitively, the more ignorant we are about an entity, the greater will be the miscoding of our current representation. A deeper understanding of the entity should allow for an encoding closer to a perfect one.

Miscoding is computed using a two-way approach: we require the oracle to compute the length of the shortest computer program that can print the string $r^\star_e$ given our representation $r$, and the other way around, that is, to compute the length of the shortest computer program that can print $r$ given the string $r^\star_e$. As expected, a good representation will include all the information required to reconstruct an entity, and it will not include wrong, nor irrelevant, information. Miscoding is about including only relevant symbols, meanwhile surfeit (see Chapter \ref{chap:Redundancy}) is about including only those symbols that are needed.

In our definition of miscoding we have used a relative measure, instead of the absolute one, because besides to compare the miscoding of different encodings for the same entity, we are also interested in comparing the miscoding of different entities.

The miscoding of a representation $r$ is always a number between $0$ and $1$.

\begin{proposition}
\label{prop:range_miscoding}
We have that $0 \leq \mu(r) \leq 1$ for all $r \in \mathcal{B}^\ast$.
\end{proposition}
\begin{proof}
Given that $0 \leq \frac{ \max\{ K(x \mid y), K(y \mid x) \} } { \max\{ K(x), K(y) \} } \leq 1$ for all $x, y \in \mathcal{B}^\ast$ according to Proposition \ref{prop:ncd_between_zero_and_one}.
\end{proof}

Miscoding is equal to zero if, and only if, the representation $r$ is one of the possible valid representations of an entity $e$.

\begin{proposition}\label{prop:perfect_encoding}
Let $r \in \mathcal{B}^\ast$ be a representation. We have that $\mu(r) = 0$ if, and only if, $r \in \mathcal{R}^\star_\mathcal{E}$.
\end{proposition}
\begin{proof}
If $r \in \mathcal{R}^\star_\mathcal{E}$ then there exists an entity $e \in \mathcal{E}$ such that $r = r^\star_e$, and so, we have that $K \left( r \mid r^\star_e \right) = 0$ and $\mu(r) = 0$. If $\mu(r) = 0$ then there must exist an $r^\star_e \in \mathcal{R}^\star_\mathcal{E}$ such that $K \left( r \mid r^\star_e \right) = 0$, which implies that $r = r^\star_e$ and $r \in \mathcal{R}^\star_\mathcal{E}$.
\end{proof}

According to Proposition \ref{prop:perfect_encoding}, if the miscoding of $r$ is equal to 0 we can conclude that $r$ perfectly encodes an entity $e$. The problem is that there is no way to know which one is the entity encoded by $r$. Of course, given our scientific intuition, we could guess the entity encoded, but from a mathematical point of view, we can not prove we are right. Moreover, with time and more research, it might happen that we change our mind about the nature of the encoded entity (see Example \ref{ex:polywater}).

As we have seen, an entity $e \in \mathcal{E}$ can have multiple valid representations, given by the set $\mathcal{R}^\star_e$. Fortunately, miscoding is a quantity that does not depend on the representation selected (see the problem of style in Section \ref{sec:scientific_representation}).

\begin{proposition}
Let $r^\star_e \in \mathcal{R}^\star_\mathcal{E}$ be a valid representation, then we have that $\mu\left( r^\star_e \right) \leq \mu\left( r \right)$ for all $r \in \mathcal{B}^\star$.
\end{proposition}
\begin{proof}
Given that $\mu\left( r^\star_e \right) = 0$ and $\mu\left( r \right) \leq 0$ for all $r \in \mathcal{B}^\star$.
\end{proof}

Given an entity $e$, all the valid representations that belong to $\mathcal{R}^\star_e$ are equally good form the point of view of miscoding, since all of them have a miscoding of $0$. In practice, we should select the representation that makes it easier to gather new knowledge about the original entity, that is, to derive models that explain the entity.

%
% Section: Miscoding of Joint Representations
%

\section{Miscoding of Joint Representations}
\label{sec:joint_miscoding}

As we have seen in Section \ref{sec:descriptions_joint_topic} if $s, t \in \mathcal{B}^\ast$ are two representations, the concatenation $st$ of those strings is also a representation, called joint representation. In this section we are interested in to study the miscoding of joint representations and their properties.

Given that the joint representation $st$ is also a representation, its miscoding will be given by:
\[
\mu(st) = \overset{o}{ \underset{ r^\star_e \in \mathcal{R}^\star_\mathcal{E} } \min} \frac{ \max\{ K \left( st \mid r^\star_e \right), K \left( r^\star_e \mid st \right) \} } { \max\{ K \left( st \right), K \left( r^\star_e \right) \} }
\]

The miscoding of a joint representation is a number between 0 and 1, $0 \leq \mu(st) \leq 1$, as a direct consequence of Proposition \ref{prop:range_miscoding}.

Adding more symbols to an incomplete representation, i.e. a representation with positive miscoging, is not a guaratee that the misconding will decrease, since the added symbols migth be wrong. In the same way, it is not always the case that the miscoding will increase, because adding relevant symbols will decrease miscoding of an incomplete representation. Formally, given two arbitrary representations $s, t \in \mathcal{B}^\ast$ there is no guarantee that one of the following relations will hold: $\mu(ts) \geq \mu(t)$, $\mu(ts) \leq \mu(t)$, $\mu(ts) \geq \mu(s)$ nor $\mu(ts) \leq \mu(s)$.

It is not either the case that the miscoding of a joint representation is smaller than the sum of the miscoding of the individual representations, that is, $\mu( st ) \leq \mu(s) + \mu(t)$. Not even in the case that both representations $t$ and $s$ encode the same entity, since it might happen that the joint presentation $ts$ encodes a totally different entity than $t$ and $s$.

Finally, since the operation of joining two representations is not conmutative, it is not guaranteed that $\mu(ts) = \mu(st)$. Again, it might happen that the strings $ts$ and $st$ encode two different entities.

We can extend the concept of miscoding of joint representations to any finite collection of representations. If $r_1, r_2, \ldots, r_n \in \mathcal{B}^\ast$ is a finite collection of representations, the miscoding of the joint representation $r_1 r_2, \ldots r_n$, will be given as:
\[
\mu(r_1 r_2 \ldots r_n) = \overset{o}{ \underset{ r^\star_e \in \mathcal{R}^\star_\mathcal{E} } \min} \frac{ \max\{ K \left( r_1 r_2 \ldots r_n \mid r^\star_e \right), K \left( r^\star_e \mid r_1 r_2 \ldots r_n \right) \} } { \max\{ K \left( r_1 r_2 \ldots r_n \right), K \left( r^\star_e \right) \} }
\]
As it was the case of joining two representatoins, we have that $0 \leq \mu(r_1, r_2, \ldots, r_n) \leq 1$. There is no guarantee that $\mu(r_1, r_2, \ldots, r_n) \leq \mu(r_i)$ nor $\mu(r_1, r_2, \ldots, r_n) \geq \mu(r_i)$ for $i = 1, \ldots, n$. Also, it is not always the case that $\mu(r_1, r_2, \ldots, r_n) \leq \mu(r_1) + \ldots + \mu(r_n)$. Finally, we cannot say anything about the miscoding of a permutation of the representations included in the joint representation with respect to the original miscoding.

%
% Section: Reducing Miscoding
%

\section{Reducing Miscoding}

A valid representation of an entity is a string that contains all the necessary information required by the oracle to reconstruct that entity and only that information. If the representation is non-valid, meaning its miscoding exceeds zero, it may be because some crucial information is missing, some symbols are incorrect, or it contains irrelevant symbols. To reduce the miscoding of a representation, we can add the missing information, or remove the incorrect or irrelevant symbols. We cannot know in advance if any information is missing or some symbols need to be removed. Fortunately, as the next theorem illustrates, it must be one of these two cases.

\begin{theorem}
\label{th:reduce_miscoding}
Let $r \in \mathcal{B}^\ast$ be a representation such that $\mu(r) >0$, then at least one of the following cases is true:
\begin{enumerate}[label=(\roman*)]
\item there exist a $s \in \mathcal{B}^\ast$ such that $\mu(rs) < \mu(r)$ or $\mu(sr) < \mu(r)$,
\item there exists a $s \in \mathcal{B}^\ast$ in the form $r = \alpha s \beta$ with $\alpha, \beta \in \mathcal{B}^\ast$ such that $\mu(r) < \mu(s)$.
\end{enumerate}
\end{theorem}
\begin{proof}
{\color{red} Finish}
Assume that $\mu(r) >0$. We have that
\[
\overset{o}{ \underset{ r^\star_e \in \mathcal{R}^\star_\mathcal{E} } \min} \frac{ \max\{ K \left( r \mid r^\star_e \right), K \left( r^\star_e \mid r \right) \} } { \max\{ K \left( r \right), K \left( r^\star_e \right) \} } > 0
\]
Let $r^\star_e = arg\,min \left( \mu(r) \right)$. We have that $\max\{ K \left( r \mid r^\star_e \right), K \left( r^\star_e \mid r \right) \} > 0$. If $K \left( r \mid r^\star_e \right) > 0$ we have that $r$ contains non-relevant symbols. If $K \left( r^\star_e \mid r \right) > 0$ we have that $r$ is missing some relevant symbols.
\end{proof}

\begin{example}
{\color{red} TODO: Provide a practical example.}
\end{example}

%
% Section: Miscoding of Areas
%
\section{Miscoding of Areas}
\label{sec:miscoding_areas}

{\color{red} TODO: review and rewrite this section.}

The concept of miscoding can be extended to research areas in order to quantitative measure the amount of effort required to fix an inaccurate representation of the area.

\begin{definition}
Let $A \subset \mathcal{E}$ be an area with known subset $\hat{A} = \{e_1, e_2, \ldots, e_n\}$. We define the \emph{miscoding of the area} given the known subset $\hat{A}$ as:
\[
\mu(\hat{A}) = \min_{(r^\star_{e_1}, r^\star_{e_2}, \ldots, r^\star_{e_n}) \in \mathcal{R}^\star_\mathcal{E}}  \frac{K \left( \langle t_{e_1}, t_{e_2}, \ldots, t_{e_n} \rangle \mid \langle t_1, t_2, \ldots, t_n \rangle \right) }{K \left( \langle t_{e_1}, t_{e_2}, \ldots, t_{e_n} \rangle \right)}
\]
\end{definition}

{\color{red} TODO: Say something about the concept of intradisciplinary and interdisciplinary in the context of miscoding and provide a reference to the chapter of computational creativity.}

%
% Section: References
%

\section*{References}

{\color{red} TODO: Pending.}
