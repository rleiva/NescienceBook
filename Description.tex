%
% CHAPTER: Entities, Represenations and Descriptions
%

\chapterimage{owl.pdf} % Chapter heading image

\chapter{Fundamental Elements}
\label{chap:Topics-and-Descriptions}

\begin{quote}
\begin{flushright}
\emph{We are all agreed that your theory is crazy. \\
The question which divides us is whether it is crazy enough.} \\
Niels Bohr
\end{flushright}
\end{quote}
\bigskip

The first step in quantifying our lack of knowledge involves the precise identification of the entities under investigation. The elements of this collection depend largely on the specific application of the theory of nescience. Each application requires a distinct set of entities, whether mathematical objects, living organisms, human needs, or otherwise. Fortunately, the process of quantifying what we do not know remains consistent across all such domains.

The next step is to devise a procedure for representing the identified entities as strings of symbols. Accurately encoding a research entity is a complex and unresolved epistemological challenge. The solution proposed within the theory of nescience is based on the concept of an oracle Turing machine. The practical feasibility of this solution depends largely on the level of abstraction of the entities being studied. For example, encoding abstract mathematical objects poses significant difficulties and often requires an approximation. In contrast, encoding computer programs is relatively straightforward, as they are already expressed as strings.

Once a suitable method for encoding the original entities as string-based representations has been established, the final step is to produce a description. This description should be both accurate and concise, reflecting our current understanding of the representation. In the theory of nescience, descriptions are required to be computable, meaning that a computer must be able to fully reconstruct the original representation from its description. However, descriptions refer to representations (that is, to the way entities are encoded) not to the entities themselves. As a result, how well a description captures knowledge about an entity depends largely on the quality of the representation used.

In this chapter, we will formalize these core concepts: entities, representations, and descriptions, among others. We will also explore what constitutes a perfect description, how to compute the combined representation of multiple entities, and how to describe a representation given some prior background knowledge. Additionally, we will examine the concept of a research area and its associated properties.

%
% Section: Entities
%

\section{Entities}
\label{sec:descriptions_entities}

Defining the nature of a research entity is a complex and unresolved philosophical problem. We approach this complexity from a fundamentally pragmatic perspective, rather than a philosophical one. Our theory starts from the premise that there exists a non-empty set of \emph{entities}\index{Entities} that we seek to understand.

\begin{notation}
We represent the set of research entities under consideration as $\mathcal{E}$.
\end{notation}

The elements of $\mathcal{E}$ are defined relative to a particular area of knowledge in which the theory of nescience is applied, such as mathematics, physics, engineering, or the social sciences, or may be further restricted to a specific subarea within any of these disciplines. From a formal perspective, $\mathcal{E}$ should be regarded as a fixed but generally inaccessible collection: although its elements are determined by the underlying domain of inquiry, there is in general no effective (computable) procedure for accessing, deciding membership in, or enumerating its elements. 

The set $\mathcal{E}$ contains both \emph{knowable}\index{Knowable entity} and \emph{unknowable}\index{Unknowable entity} entities (see Section~\ref{sec:what-is-an-entity} for further discussion of the notion of an entity and Section~\ref{sec:perfect_representations} for a formal characterizations of each of these types of entities). Knowable entities are those for which a correct or perfect representation may exist, and they may be either concrete, such as physical particles or biological species, or abstract, such as mathematical structures or algorithms. The set $\mathcal{E}$ also includes unknowable entities, whether real, such as infinite systems or idealized structures, or fictitious, such as hypothetical physical entities later shown not to exist. 

Unknowable fictitious entities are included in $\mathcal{E}$ because they play a genuine role in scientific research: even when they do not correspond to any real entity, they guide experimentation, shape theories, and contribute to the dynamics of discovery, error, and revision. An entity belongs to $\mathcal{E}$ insofar as it is the intended target of representation, measurement, explanation, or modeling, independently of whether it ultimately admits a correct or perfect representation.

In the theory of nescience, the possibility of using universal sets is excluded; that is, the existence of a set $\xi$ containing all possible objects cannot be assumed. The problem with universal sets is that their existence is incompatible with Cantor's theorem (see Example \ref{cantor_theorem}). Cantor's theorem proves that the power set $\mathcal{P}(\xi)$, consisting of all possible subsets of $\xi$, has more elements than the original set $\xi$. This contradicts the assumption that $\xi$ includes everything. The set $\mathcal{E}$ must be explicitly fixed and domain-scoped; that is, it must denote the set of entities under consideration in a given research area, rather than an ambiguous or open-ended collection.

\begin{example}[Cantor's theorem]\index{Cantor's theorem}
\label{cantor_theorem}
Cantor's theorem proves that for any set $A$, $d(A) < d\left(\mathcal{P}(A)\right)$\footnote{Here $d(\cdot)$ denotes cardinality, see Appendix \ref{chap:Discrete_Mathematics}.}. Consider $f: A \rightarrow \mathcal{P}(A)$, a function that maps each element $x \in A$ to the set $\{x\} \in \mathcal{P}(A)$; evidently, $f$ is injective, implying $d(A) \leq d\left(\mathcal{P}(A)\right)$. To substantiate that the inequality is strict, let's assume there exists a surjective function $g: A \rightarrow \mathcal{P}(A)$ and consider the set $B = \{ x \in A : x \notin g(x) \}$. As $g$ is surjective, there must exist a $y \in A$ such that $g(y) = B$. This, however, raises a contradiction, $y \in B \Leftrightarrow y \notin g(y) = B$. Consequently, the function $g$ cannot exist, therefore $d(A) < d\left(\mathcal{P}(A)\right)$.
\end{example}

In the theory of nescience, not all conceivable sets are acceptable, as some may give rise to paradoxes. Take, for example, Russell's paradox, which proposes a set $R$ consisting of all sets that are not members of themselves. The paradox arises when we try to discern whether $R$ is a member of itself (see the Example \ref{ex:russell_paradox}). To avoid such problems, the theory of nescience is based on the Zermelo-Fraenkel set of axioms\index{Zermelo-Fraenkel set of axioms}, along with the Axiom of Choice\index{Axiom of Choice} (or ZFC). The Axiom of Separation\index{Axiom of Separation} (if $P$ is a property with parameter $p$, then for any set $x$ and parameter $p$ there exists a set $y=\{u \in x : P(u) \}$ that includes all those sets $u \in x$ that have property $P$) allows the use of this notation only to construct sets that are subsets of already existing sets. A more extensive Axiom of Comprehension\index{Axiom of Comprehension} (if $P$ is a property, then there exists a set $y=\{u : P(u) \}$) would be required to allow sets like the one proposed by Russell's paradox. Russell's paradox arises from the use of an unrestricted comprehension principle. In the axioms of ZFC, and in the theory of nescience, the axiom of comprehension is considered false.

\begin{example}[Russell's paradox]\index{Russell's paradox}
\label{ex:russell_paradox}
Suppose $R$ is the set of all sets not members of themselves, such that $R = \{ x : x \notin x \}$. The contradiction arises when querying if $R$ is a member of itself. If $R$ is not a member of itself, by its own definition, it should be; conversely, if $R$ is declared to be a member of itself, its definition dictates it should not be. Symbolically, this can be written as $R \in R \Leftrightarrow R \notin R$.
\end{example}

In this book we do not address the classic problems of ontology\index{Ontology}, that is, the classification of entities that exist in the world and can be known. Furthermore, we do not attempt to resolve epistemological\index{Epistemology} questions, such as how scientific knowledge is validated by evidence, or what the nature of that evidence is.

Once a set $\mathcal{E}$ of entities has been selected, the next step is to uniquely encode them as strings of symbols, which will make them easier to describe. A method for performing this encoding efficiently is described in the following section.

%
% Section: Representation Oracle
%

\section{Representation Oracle}
\label{sec:representations}

The representation of the entities composing the set $\mathcal{E}$, which may be abstract or only informally specified, raises significant challenges (see Section~\ref{sec:scientific_representation}\index{Scientific representation} for an overview of proposed solutions in the scientific literature, and their limitations). This book presents a particular approach to this problem by dividing scientific representation into two closely related subproblems: the encoding of entities by means of representations, and the modeling of these representations by means of descriptions. From this perspective, scientific models do not explain entities directly, but rather explain them through their representations (see Figure~\ref{fig:representationProblem} in the Introduction reproduced here for ease of reference). In this section we focus on encoding; the role of descriptions is addressed in Section~\ref{sec:descriptions_models}.

% A4
% \begin{figure}[t]
% \centering
% \begin{tikzpicture}
%     \node[draw, ellipse, minimum width=3cm] (entities) at (0,0) {Entities};
%     \node[draw, ellipse, minimum width=3cm] (representations) at (5,0) {Representations};
%     \node[draw, ellipse, minimum width=3cm] (descriptions) at (10,0) {Descriptions};
% 
%     \draw[-{Latex[length=3mm]}] (representations) -- node[above,midway] {encode} (entities);
%     \draw[-{Latex[length=3mm]}] (descriptions) -- node[above,midway] {model} (representations);
%     \draw[dashed,-{Latex[length=3mm]}] (descriptions) to [bend right=45] node[above,midway] {explain?} (entities);
% \end{tikzpicture}
% \caption{\label{fig:representationProblem}The Problem of Understanding}
% \end{figure}

% Octavo
\begin{figure}[t]
\centering
\begin{tikzpicture}
    \node[draw, ellipse, minimum width=1.5cm] (entities) at (0,0) {Entities};
    \node[draw, ellipse, minimum width=1.5cm] (representations) at (4.25,0) {Representations};
    \node[draw, ellipse, minimum width=1.5cm] (descriptions) at (9,0) {Descriptions};

    \draw[-{Latex[length=3mm]}] (representations) -- node[above,midway] {encode} (entities);
    \draw[-{Latex[length=3mm]}] (descriptions) -- node[above,midway] {model} (representations);
    \draw[dashed,-{Latex[length=3mm]}] (descriptions) to [bend right=45] node[above,midway] {explain?} (entities);
\end{tikzpicture}
\caption{The Problem of understanding}
\end{figure}

In Kolmogorov complexity\index{Kolmogorov complexity}, the representation problem is usually approached by assuming that the set $\mathcal{E}$ is well defined and countable, and that a total encoding function $f : \mathcal{E} \rightarrow \mathbb{N}$ exists, analogous to a Gödel numbering\index{Gödel numbering}. The theory of nescience similarly adopts the idea that entities can be encoded as numbers, or equivalently as finite strings. Our approach, however, departs from this tradition by dropping the requirement that $\mathcal{E}$ be explicitly definable or enumerable.

Rather than postulating an encoding function with domain $\mathcal{E}$, we take as primitive the well-defined set of finite strings and assume the existence of an abstract oracle that internally associates certain strings with entities. More precisely, we introduce an oracle $\mathcal{O}_\mathcal{E}$ that acts as a scientific authority. Given a finite string $r$, the oracle determines whether $r$ contains sufficient information to reconstruct a unique entity without additional input. If so, the oracle internally associates $r$ with that entity; otherwise, the string is deemed non-representational.

The oracle $\mathcal{O}_{\mathcal{E}}$ is an idealized construct. For most collections of entities, it cannot be realized in practice, and no explicit decoding procedure associated with it can be provided. The role of $\mathcal{O}_{\mathcal{E}}$ is purely formal: it serves as a reference device for delimiting what representations of entities are in principle obtainable within the framework. Accordingly, $\mathcal{O}_{\mathcal{E}}$ is treated as a black box and is defined only through its input-output behavior. In this respect, $\mathcal{O}_{\mathcal{E}}$ is analogous to the oracle in an oracle Turing machine\index{Oracle Turing machine}\index{Oracle Turing machine} (see Definition~\ref{def:Oracle-Turing-Machine}).

For the purposes of this book, and without loss of generality, we restrict attention to binary strings as the medium of representation.

\begin{definition}
\label{def:descriptions_topic}
Given a set of entities \(\mathcal{E}\), a \emph{representation oracle}\index{Representation oracle}
is a total decision oracle
\[
\mathcal{O}_{\mathcal{E}} : \mathcal{B}^\ast \longrightarrow \{\mathrm{true},\mathrm{false}\}
\]
with the following semantics: there exists an oracle-internal partial \emph{representation function}\index{Representation function}

\[
\delta_{\mathcal{E}} : \mathcal{B}^\ast \rightharpoonup \mathcal{E}
\]
such that, for every \(r \in \mathcal{B}^\ast\),
\[
\mathcal{O}_{\mathcal{E}}(r)=\mathrm{true}
\quad\Longleftrightarrow\quad
\delta_{\mathcal{E}}(r)\ \text{is defined}.
\]
\end{definition}

\noindent
When defined, $\delta_{\mathcal{E}}(r)$ yields a unique entity of $\mathcal{E}$ reconstructed from $r$ without any additional information. The function $\delta_{\mathcal{E}}$ is not assumed to be computable or explicitly characterizable, and serves only to fix the semantic interpretation of the oracle.

{\color{red}TODO: Introduce the concept of "decodable" representation as those strings for which the Oracle answer true.}

Within this framework, only elements of $\mathcal{B}^\ast$, that is, finite binary strings, are eligible to serve as representations of entities (see the discussion of the ontology problem\index{Problem of ontology} in Section~\ref{sec:scientific_representation}). Other forms of representation (such as physical models, diagrams, or verbal descriptions) are admissible only insofar as they can be transcribed into binary strings. No distinction is drawn between scientific representations and other types of representations (see the discussion of representational demarcation\index{Representational demarcation} in Section~\ref{sec:scientific_representation}). Finally, it is essential to note that $\delta_{\mathcal{E}}$ is a partial function: not every binary string represents an entity.

In general, the internal oracle need not be injective, surjective, or unique for a given collection of entities, reflecting the plurality and ambiguity inherent in real-world scientific representation.

\begin{example}
\label{ex:not_unique_oracle}
For a given set of entities $\mathcal{E}$, the existence of a representation internal function $\delta_{\mathcal{E}}$ does not imply its uniqueness. For instance, a binary negation oracle (which transforms the zeros of a binary string into ones, and the ones into zeros), assigning to each string $x \in \mathcal{B}^\ast$ the entity $\delta_{\mathcal{E}} \left( \neg x \right)$ would also qualify as a representation function.
\end{example}

Rather than deploying individual oracles for each set $\mathcal{E}$, we could have employed a universal oracle machine. This is a machine $\mathcal{U}_\mathcal{O}$ that, given the encoding of an oracle $\mathcal{O}_\mathcal{E}$ and a string $r$ as input, computes $\mathcal{U}_\mathcal{O} \left( \langle \mathcal{O}_\mathcal{E}, r \rangle \right) = \mathcal{O}_\mathcal{E} \left( r \right)$. Universal machines are particularly applicable to the universal set $\xi$, encompassing all entities. However, the theory of nescience is explicitly domain-relative: the choice of $\mathcal{E}$ determines which entities are under investigation and fixes the admissible representations in which the notion of nescience is defined. We work with entity sets $\mathcal{E}$ corresponding to different areas of knowledge, each associated with a single oracle $\mathcal{O}_\mathcal{E}$, the one most suitable to our current knowledge and practical needs.

\begin{example}
\label{ex:description_dna}
Consider the case when the subjects of study are animals. Initially, one might use detailed physical descriptions of animals as encodings. In this scenario, the internal representation function would be a hypothetical machine capable of reconstructing the original animal from its description. As our understanding of biology advances, we might instead adopt an alternative encoding based on the animals' DNA\index{DNA}. Both of these encodings serve as valid representations of the entities.
\end{example}

As illustrated in Example \ref{ex:not_unique_oracle} and Example \ref{ex:description_dna}, the entities in $\mathcal{E}$ can be encoded in multiple ways (see the problem of style\index{Problem of style} in Section \ref{sec:scientific_representation}). Different oracles accommodate different encoding schemes. In practical terms, some strings provide more adecuate representations of an entity than others (see the standard of accuracy\index{Standard of accuracy} in Section \ref{sec:scientific_representation}). The optimal representation depends on the type of questions we aim to answer. Representations must not only correspond to entities, but also they must be relevant to them (see the requirement of directionality\index{Requirement of directionality} in Section \ref{sec:scientific_representation} and Chapter \ref{chap:Miscoding}).

We employ an oracle function rather than an oracle relation $\mathcal{R}_\mathcal{O} \subset \mathcal{B}^\ast \times \mathcal{E}$, not merely to associate strings with their corresponding entities, but to reconstruct an entity from its representation. The internal oracle's ability to recover original entities underpins our capacity to make hypotheses about entities based on their representations (see surrogative reasoning\index{Surrogative reasoning} in Section \ref{sec:scientific_representation}). According to the theory of nescience, scientific inquiry involves not only learning how to encode entities properly but also understanding the mechanisms by which internal representaion functions decode them.

The purpose of encoding entities in the theory of nescience differs fundamentally from that in Shannon's information theory (see Appendix \ref{chap:Coding}), as illustrated in Example \ref{ex:shannon_encoding}.

\begin{example}
\label{ex:shannon_encoding}
Take, for instance, a set $\mathcal{E}$ consisting of two books: "The Ingenious Gentleman Don Quixote of La Mancha" and "The Tragedy of Romeo and Juliet". We might encode the first book with the string "0" and the second with the string "1". While these strings allow us to uniquely identify each book within the set, they do not qualify as valid encodings within the the theory of nescience. In information theory, the goal is to uniquely identify an object based on a reference, assuming mutual agreement between the sender and the receiver about the mapping from references to objects. In contrast, the theory of nescience seeks representations that preserve the richness and detail of the original entities. For example, it would be impossible to hypothesize about Cervantes'\index{Cervantes} influence on Shakespeare\index{Shakespeare} using only the strings "0" and "1".
\end{example}

One possible response to the limitation discussed in Example \ref{ex:shannon_encoding}, where entities are encoded using overly simplistic strings that fail to capture their internal structure, would be to require the set $\mathcal{E}$ to be infinite, as is done in Kolmogorov complexity. However, while requiring $\mathcal{E}$ to be infinite prevents all entities from being encoded using uniformly short strings, it does not prevent the use of representations that, although longer, remain arbitrary and devoid of semantic content. That is, they do not allow us to make meaningful inferences about the original entities based solely on their representations. To enable such reasoning, the encoding must preserve essential structural and semantic features of the entities, not just their identity.

\begin{figure}[t]
\centering
\begin{tikzpicture}[line width=1pt, >=Stealth]

%--- left outer ellipse -------------------------------------------
\draw[fill=gray!25] (-3,0) ellipse (1.5 and 2);          % 𝔅* (grey background)
\node at (-4.2,-2.4) {$\mathcal{B}^{*}$};

%--- left inner region R_{e1} -------------------------------------
\draw (-3,1) ellipse (1 and 0.7);                      % 𝑅_{e1}
\node[anchor=east] at (-2.8,1) {$r_{1}$};
\fill (-2.7,1) circle (5pt);                           % black dot r1
% \node at (-3.2,0) {$\mathcal{R}_{e_1}$};

%--- right outer ellipse ------------------------------------------
\draw (3,0) ellipse (1.5 and 2);                         % ℰ (white background)
\node at (4.4,-2.4) {$\mathcal{E}$};

%--- right inner grey region --------------------------------------
\draw[fill=gray!25] (3,0.25) ellipse (1.3 and 1.5);

%--- entities e1 and e2 -------------------------------------------
\fill (3.1,1.3) circle (5pt);                          % e1 dot
\node[anchor=west] at (3.3,1.3) {$e_{1}$};

\fill (3.2,-1.5) circle (5pt);                         % e2 dot
\node[anchor=west] at (3.4,-1.5) {$e_{2}$};

%--- oracle barrier -----------------------------------------------
\draw (0,-2.4) -- (0,2.4);
\node[below=2pt] at (0,-2.4) {oracle};

%--- curved two-piece arrow (solid then dashed) -------------------
\path
  (-2.2,1)                                     % start at r1
  .. controls (-1,2.2) and (-0.4,2) .. (0,2)   % solid part up to barrier
  node[pos=0.55]{};                            % (dummy to split the path)

\draw[line width=1pt] (-2.7,1) .. controls (-1.5,2) and (-0.4,2) .. (0,2);

\draw[line width=1pt,dashed,-Stealth]
      (0,2) .. controls (1.3,2.1) and (2.4,1.9) .. (3.05,1.3);

\end{tikzpicture}
\caption{\label{fig:entities_topics_1}Encodings and Entities}
\end{figure}

% \begin{figure}[h]
% \centering\includegraphics[scale=0.5]{entities_topics_1}
% \caption{\label{fig:entities_topics_1}Encodings and Entities.}
% \end{figure}

\begin{remark}
One of the fundamental challenges in science, and in human intellectual activity more broadly, is the tendency to confuse symbols with the things they represent. The theory of nescience has been carefully designed to avoid this issue by clearly distinguishing between research entities and their representations. However, making this distinction explicit at every step would render the book unnecessarily difficult to read. We have aimed to strike a balance between clarity of exposition and rigor of definition. Occasionally, especially when introducing new ideas, we use the term \emph{topic}\index{Topic} to refer broadly to an entity, a representation, or both. Nevertheless, in all mathematical definitions and propositions, this distinction is made unambiguous. In the case of any uncertainty, the formal definitions should be taken as the definitive reference.
\end{remark}

% Redundant Representations

\subsubsection*{Redundant Representations}

Within the theory of nescience, representations are not required to be minimal. Their primary role is to be correct, in the sense that they contain sufficient information for the oracle to reconstruct the corresponding entities. Representations themselves are not intended to support scientific reasoning directly, and therefore need not be concise. The requirement of minimality becomes relevant only at a later stage, when we introduce descriptions, or models (see Section \ref{sec:descriptions_models}). Descriptions aim to capture essential aspects of representations in a more compact form and are explicitly designed to support reasoning, explanation, and prediction. In that context, assumptions about background knowledge and shared conventions are not only acceptable but often necessary to achieve brevity and explanatory power.

\begin{definition}
\label{def:nonredundant_description}
Let $\mathcal{O}$ be an oracle with representation function $\delta$, and let $r \in \mathcal{B}^\ast$ a decodable representation. We say that $r$ is a \emph{non-redundant}\index{Non-redundant representation} representation if there is no a decodable representation $s\in\mathcal{B}^\ast$ such that
\[
l(s) < l(r) \wedge\ \delta(s) = \delta(r)
\]
\end{definition}

\begin{notation}
We denote by $\bar{\mathcal{R}}_{\mathcal{O}_\mathcal{E}}$ the set of all non-redundant decodable representations.
\end{notation}

\begin{proposition}[Existence of non-redundant descriptions]\index{Non-redundant description!existence}
\label{prop:existence_nonredundant}
Fix an oracle \(\mathcal{O}\) and an entity \(e\in\delta(\mathcal{R}^\star_{\mathcal{O}})\).
Then there exists at least one non-redundant description \(r\in\mathcal{R}^{\star,0}_{\mathcal{O}}\) such that \(\delta(r)=e\).
\end{proposition}

\begin{proof}
Let \(A_e:=\{\,|r| : r\in\mathcal{R}^\star_{\mathcal{O}},\ \delta(r)=e\,\}\subseteq\mathbb{N}\).
By assumption \(e\in\delta(\mathcal{R}^\star_{\mathcal{O}})\), so \(A_e\neq\varnothing\).
Since \(\mathbb{N}\) is well-ordered, \(A_e\) has a minimum \(m\).
Choose \(r\in\mathcal{R}^\star_{\mathcal{O}}\) with \(\delta(r)=e\) and \(|r|=m\).
By construction, there is no shorter \(s\in\mathcal{R}^\star_{\mathcal{O}}\) with \(\delta(s)=e\), hence \(r\in\mathcal{R}^{\star,0}_{\mathcal{O}}\).
\end{proof}

\begin{example}[Redundant expansions and non-redundancy]
\label{ex:nonredundant_padding}
Let \(\mathcal{B}=\{0,1\}\) and let \(\sigma\) be a style in which appending any number of trailing zeros is redundant:
\[
\mathrm{Red}_\sigma(r)=\{\, r0^k : k\in\mathbb{N}\,\}.
\]
Consider an oracle \(\mathcal{O}\) for which \(\delta(101)=e\) for some entity \(e\), and assume redundancy closure.
Then \(1010, 10100, 101000,\ldots\) are all in \(\mathcal{R}^\star_{\mathcal{O}}\) and decode to the same entity \(e\).
Among these, \(101\) is non-redundant (there is no shorter representation of \(e\) in this family),
whereas \(10100\) is redundant relative to \(\mathcal{O}\) because it is longer than \(101\) and carries no essential additional information about \(e\).
\end{example}

\begin{proposition}[Redundancy does not affect the core]\index{Non-redundant description!core}
\label{prop:redundancy_core}
Assume redundancy closure (Assumption~\ref{ass:red_closure}).
For any oracle \(\mathcal{O}\), the set \(\mathcal{R}^{\star,0}_{\mathcal{O}}\) contains, for each entity \(e\in\delta(\mathcal{R}^\star_{\mathcal{O}})\), at least one shortest representation of \(e\) accepted by \(\mathcal{O}\).
In particular, adding redundant expansions to \(\mathcal{R}^\star_{\mathcal{O}}\) does not create new elements of \(\mathcal{R}^{\star,0}_{\mathcal{O}}\).
\end{proposition}

\begin{proof}
The first claim is Proposition~\ref{prop:existence_nonredundant}.
For the second, redundant expansions (by definition) are longer strings constructed from an already accepted representation.
They therefore cannot be shorter than an existing shortest representation of the same entity, hence they cannot enter \(\mathcal{R}^{\star,0}_{\mathcal{O}}\).
\end{proof}

\begin{definition}[Redundancy closure]\index{Redundancy closure}
\label{ass:red_closure}
For every oracle \(\mathcal{O}\) under style \(\sigma\), for every \(r\in\mathcal{R}^\star_{\mathcal{O}}\), and every \(r'\in\mathrm{Red}_\sigma(r)\),
we have \(r'\in\mathcal{R}^\star_{\mathcal{O}}\).
\end{definition}


\begin{definition}
Let $\mathcal{O}_A$ and $\mathcal{O}_B$ be two oracles defined over the set of entities $\mathcal{E}$. We say are said that the oracles $\mathcal{O}_A$ and $\mathcal{O}_B$ are \emph{comparable}\index{Comparable oracle} if
\[
\mathcal{O}^\top_A = \mathcal{O}^\top_B.
\]
\end{definition}


% Minimal Oracles
\subsubsection*{Minimal Oracles}
\label{sec:minimal_oracles}

Ideally, representation oracles should be minimal in size, meaning that all information needed to reconstruct an entity is carried by the representation string rather than being hard-wired into the oracle. A non-minimal oracle\index{Non-minimal oracle} can simplify the representation problem by offloading complexity to its internal representation function, but it also forces researchers to understand not only the representation itself but the oracle's internal reconstruction process, something that, given the abstract nature of oracles, is generally not accessible in practice.

When an oracle contains hard-wired information, it can reconstruct an entity from a representation that leaves part of that information implicit, representations that an oracle without that hard-wired knowledge would be unable to decode. This property provides a natural basis for ordering oracles: we compare the sets of distinct non-redundant representations they accept, since an oracle with more hard-wired information admits a larger set of non-redundant decodable representations.

\begin{definition}
Let $\mathcal{O}_A$ and $\mathcal{O}_B$ two comparable oracles. We say that $\mathcal{O}_A$ \emph{reduces}\index{Oracle reduction} to $\mathcal{O}_B$, denoted by $\mathcal{O}_A \triangleright \mathcal{O}_B$, if
\[
\bar{\mathcal{R}}_{\mathcal{O}_B} \subsetneq \bar{\mathcal{R}}_{\mathcal{O}_A}.
\]
\end{definition}

$\mathcal{O}_B$ is a simplification of $\mathcal{O}_A$, in the sense that it decodes strictly fewer non-redundant representations while still decoding the same entity subset of entities of $\mathcal{E}$.

We now formalize the idea that one oracle may be a \emph{simplification} of another, obtained by removing hard-wired information.

This is consistent with the intuition that hard-wired information increases decodability (and therefore tends to enlarge the set of non-redundant perfect representations by enabling representations with missing information).

\begin{example}
Consider representations of chemical compounds. A non-minimal oracle may incorporate extensive background chemical knowledge (e.g., periodic table, standard valences, and common structures). Under such an oracle, a compound could be represented by a short string such as its common name (e.g., \emph{water}, \emph{ethanol}), from which the oracle reconstructs the full molecular structure. In this case, most of the information required for reconstruction is implicit in the oracle rather than explicit in the representation. By contrast, a more minimal oracle requires the representation to encode the relevant structure directly, for instance by specifying atoms and bonds. The representation is then longer but makes explicit what is being encoded and reduces reliance on background assumptions embedded in the oracle.
\end{example}



\begin{example}[Hard-wiring and reduction]
\label{ex:hardwire_reduction}
Let \(\mathcal{E}=\{e_0,e_1\}\), \(\mathcal{B}=\{0,1\}\), and fix a style \(\sigma\) in which redundant expansions are obtained by appending trailing zeros, as in Example~\ref{ex:nonredundant_padding}.
Consider two comparable oracles for \((\sigma,S)\) with \(S=\{e_0,e_1\}\):
\begin{itemize}
\item Oracle \(\mathcal{O}_B\) decodes \(e_0\) from the representation \(0\) and decodes \(e_1\) from \(1\), and nothing shorter.
Then \(\mathcal{R}^{\star,0}_{\mathcal{O}_B}=\{0,1\}\).
\item Oracle \(\mathcal{O}_A\) contains one extra hard-wired bit of information allowing it to decode \(e_0\) also from the empty string \(\epsilon\) (an information-omitted representation).
Then \(\mathcal{R}^{\star,0}_{\mathcal{O}_A}=\{\epsilon,1\}\).
\end{itemize}
Here \(\mathcal{R}^{\star,0}_{\mathcal{O}_B}\not\subseteq \mathcal{R}^{\star,0}_{\mathcal{O}_A}\) and \(\mathcal{R}^{\star,0}_{\mathcal{O}_A}\not\subseteq \mathcal{R}^{\star,0}_{\mathcal{O}_B}\), so these two oracles are incomparable under \(\triangleright\).
Now modify \(\mathcal{O}_A\) so that it decodes all representations that \(\mathcal{O}_B\) decodes and, in addition, decodes \(\epsilon\) as \(e_0\).
Then \(\mathcal{R}^{\star,0}_{\mathcal{O}_A}=\{\epsilon,0,1\}\), and therefore \(\mathcal{O}_A\triangleright\mathcal{O}_B\).
\end{example}

Basic properties of reduction.

\begin{proposition}
Let $S \subset \mathcal{E}$ a subset of entities, and $\mathfrak{O}_{S}$ the set of comparable oracles over $S$ The reduction relation $\triangleright$ on  is:
\begin{enumerate}
\item \emph{irreflexive}: $\neg(\mathcal{O} \triangleright \mathcal{O})$ for all $\mathcal{O} \in \mathfrak{O}_{S}$,
\item \emph{transitive}: if $\mathcal{O}_A \triangleright \mathcal{O}_B$ and $\mathcal{O}_B \triangleright \mathcal{O}_C$, then $\mathcal{O}_A \triangleright \mathcal{O}_C$, for all $\mathcal{O}_A, \mathcal{O}_B, \mathcal{O}_C \in \mathfrak{O}_{S}$
\item \emph{not total}: there may exist $\mathcal{O}_A , \mathcal{O}_B \in \mathfrak{O}_{S}$ such that neither $\mathcal{O}_A \triangleright \mathcal{O}_B$ nor $\mathcal{O}_B \triangleright \mathcal{O}_A$ holds.
\end{enumerate}
\end{proposition}

\begin{proof}
Irreflexivity follows from the fact that no set is a strict subset of itself.
Transitivity follows from transitivity of strict set inclusion.
Non-totality is witnessed by any pair of incomparable sets under \(\subsetneq\), as illustrated in Example~\ref{ex:hardwire_reduction}.
\end{proof}

An oracle is minimal if it cannot be reduced to any other comparable oracle.

\begin{definition}
\label{def:minimal_oracle}
Let $\mathfrak{O}$ be a set of comparable oracles. We say that $\mathcal{O} \in \mathfrak{O}$ is \emph{minimal}\index{Minimal oracle} if
\[
\nexists \mathcal{O}' \in \mathfrak{O} : \mathcal{O} \triangleright \mathcal{O}'.
\]
\end{definition}

\noindent

Equivalently, \(\mathcal{O}\) is minimal if there is no \(\mathcal{O}'\in\mathfrak{O}_{\sigma,S}\) with
\[
\mathcal{R}^{\star,0}_{\mathcal{O}'} \subsetneq \mathcal{R}^{\star,0}_{\mathcal{O}}.
\]

{\color{red} TODO: Minimal oracles are those who have no wired information.}

\begin{example}[A minimal oracle]
\label{ex:minimal_oracle}
Let \(\mathcal{E}=\{e_0,e_1,\dots,e_7\}\) and let \(\mathcal{B}=\{0,1\}\).
Fix a style \(\sigma\) in which redundant expansions may append trailing zeros (as before), and let \(S=\mathcal{E}\).
Consider an oracle \(\mathcal{O}_{\mathrm{bin}}\) whose non-redundant descriptions of each entity are exactly the length-\(3\) binary strings,
\[
\mathcal{R}^{\star,0}_{\mathcal{O}_{\mathrm{bin}}}=\{000,001,010,011,100,101,110,111\},
\]
with \(\delta(abc)=e_k\) where \(k\) is the integer represented by \(abc\).
Any oracle that hard-wires additional information (while still decoding the same \(S\) under \(\sigma\)) can be constructed so that it also decodes some entities from shorter strings (e.g., from a 2-bit prefix, or from the empty string), thereby introducing additional non-redundant descriptions and strictly enlarging \(\mathcal{R}^{\star,0}\).
Within the class \(\mathfrak{O}_{\sigma,S}\) that requires decoding exactly \(S\) under the binary style \(\sigma\), \(\mathcal{O}_{\mathrm{bin}}\) is a natural candidate for minimality because it does not decode any entity from fewer than \(3\) bits.
\end{example}

\begin{proposition}[Characterization of minimality]\index{Oracle!minimal!characterization}
\label{prop:characterization_minimality}
Fix \((\sigma,S)\).
An oracle \(\mathcal{O}\in\mathfrak{O}_{\sigma,S}\) is minimal if and only if
\(\mathcal{R}^{\star,0}_{\mathcal{O}}\) is a minimal element (under set inclusion) among the family
\(\{\mathcal{R}^{\star,0}_{\mathcal{O}'} : \mathcal{O}'\in\mathfrak{O}_{\sigma,S}\}\).
\end{proposition}
\begin{proof}
This is immediate from Definition~\ref{def:minimal_oracle} and Definition~\ref{def:oracle_reduction}:
\(\mathcal{O}\) is minimal precisely when there is no \(\mathcal{O}'\) with a strictly smaller non-redundant core.
\end{proof}


{\color{red} TODO: Minimality is a local property.}

\begin{proposition}[Non-uniqueness and non-existence]\index{Oracle!minimal!existence}
\label{prop:minimal_nonunique_nonexist}
Fix \((\sigma,S)\).
\begin{enumerate}
\item Minimal oracles need not be unique: there may exist distinct minimal \(\mathcal{O}_1,\mathcal{O}_2\in\mathfrak{O}_{\sigma,S}\) with incomparable sets \(\mathcal{R}^{\star,0}_{\mathcal{O}_1}\) and \(\mathcal{R}^{\star,0}_{\mathcal{O}_2}\).
\item A minimal oracle need not exist: the family \(\{\mathcal{R}^{\star,0}_{\mathcal{O}} : \mathcal{O}\in\mathfrak{O}_{\sigma,S}\}\) may contain infinite strictly descending chains under \(\subsetneq\) with no minimal element.
\end{enumerate}
\end{proposition}

\begin{proof}
(1) Incomparability of \(\mathcal{R}^{\star,0}\)-sets yields distinct minimal elements in a partial order.
(2) Standard examples of set systems admit strictly descending chains with empty intersection not attained within the system; the same phenomenon can occur here depending on how \(\mathfrak{O}_{\sigma,S}\) is specified.
\end{proof}

%
% Section: Representations
%
\section{Representations}
\label{sec:perfect_representations}

In Section~\ref{sec:representations} we addressed the scientific representation problem by introducing an abstract oracle $\mathcal{O}_{\mathcal{E}}$, which evaluates finite binary strings and indicates whether each string contains sufficient information, according to an idealized standard, to reconstruct a unique entity in $\mathcal{E}$. Those strings constitute the set of perfect representations \footnote{Recall that \(s \sqsubset r\) denotes a substring of \(r\), and \(r\setminus s\) denotes the string obtained from \(r\) by removing the all the occurrences of \(s\) (see Section \ref{sec:strings}).}.

\begin{definition}
\label{def:set_perfect_represenatations}
Let \(\mathcal{E}\) be a collection of entities, and let $\mathcal{O}_{\mathcal{E}}$ be a representation oracle with oracle-internal reconstruction function
\(\delta_{\mathcal{E}} : \mathcal{B}^\ast \rightharpoonup \mathcal{E}\).
We define the set of \emph{perfect representations}\index{Perfect representation} for $\mathcal{E}$ with respect to $\mathcal{O}_{\mathcal{E}}$ as:
\begin{equation*}
\begin{split}
\mathcal{R}^\star_{\mathcal{O}_{\mathcal{E}}}
\;=\;
\Bigl\{\, & r \in \mathcal{B}^\ast \ \Big|\ 
\delta_{\mathcal{E}}(r)\ \downarrow
\ \land\
\neg\exists s \sqsubset r : \\
& \bigl( K(s \mid r \setminus s) > O(1)\ \land\ \delta_{\mathcal{E}}(r \setminus s)=\delta_{\mathcal{E}}(r) \bigr)
\,\Bigr\}.
\end{split}
\end{equation*}
\end{definition}

The set $\mathcal{R}^\star_{\mathcal{O}_{\mathcal{E}}}$ is generally unknown, since its definition depends on the behavior of an abstract and typically inaccessible oracle. Intuitively, a representation is considered perfect if it contains all the information required for the oracle to reconstruct the corresponding entity without relying on any additional information. A perfect representation must neither include incorrect information nor omit information essential for reconstruction; otherwise, the oracle would fail to recognize it as valid.

Definition \ref{def:set_perfect_represenatations} allows \emph{redundant}\index{Redundant string} information within a perfect representation, but excludes \emph{irrelevant} information. Intuitively, redundancy corresponds to information that is already implicit in the remaining part of the representation: if a substring \(s\) can be reconstructed from the rest of the string \(r\setminus s\) by a program of constant size, that is, if \(K(s\mid r\setminus s)=O(1)\), then \(s\) introduces no essentially new information about the entity beyond what is already contained in \(r\setminus s\). Such redundant content does not threaten perfectness and may legitimately remain in the representation. In fact, redundancy is often desirable in scientific and engineering practice, because it can facilitate the construction of usable descriptions or models: for instance, training modern machine learning systems (such as neural networks) typically requires very large datasets, which contain extensive redundancy in the form of repeated patterns, or multiple observations that jointly support the same underlying structure.

By contrast, an \emph{irrelevant}\index{Irrelevant string} substring is one that is both dispensable for reconstruction of the entity (i.e., removing it does not change the reconstructed entity, \(\delta_{\mathcal{E}}(r\setminus s)=\delta_{\mathcal{E}}(r)\)) and non-redundant (i.e., it is not determined by the rest of the representation, \(K(s\mid r\setminus s)>O(1)\)). Such a substring contains independent information that is not related to the entity being represented, and therefore must be removed. Otherwise, any description or model derived from the representation would be forced to account for this extraneous content, increasing complexity without improving our ability to explain or predict properties of the original entity. As a result, irrelevant information can obscure the structure that truly matters and make scientific inference more difficult.

The notion of irrelevance employed here should be understood as irrelevance relative to both the entity and the representation style. A representation is assumed to belong to a particular representational style given by the oracle (for instance, a dataset, an image, an equation, or a text description), and perfectness requires that the representation does not contain independent information that falls outside that style while leaving the reconstructed entity unchanged. For example, consider an entity \(e\) representing the data-generating mechanism behind a controlled experiment, and suppose that a dataset \(D\) of measured variables is already sufficient for the oracle to reconstruct \(e\). If we concatenate to \(D\) a photograph taken in the laboratory, the resulting string still contains valid information about the experiment, but this additional content is not part of the dataset-style encoding of the entity and can be removed without changing \(\delta_{\mathcal{E}}\). Since the imange is not determined by the remaining dataset, it has high conditional complexity and is classified as irrelevant by the definition.

Throughout this work, we assume that a specific representation oracle $\mathcal{O}_{\mathcal{E}}$ has been fixed for the collection $\mathcal{E}$ under study.

\begin{notation}
When the choice of oracle is clear from context, we denote the set of perfect representations simply by $\mathcal{R}^\star_{\mathcal{E}}$. If the set of entities is also clear from the context, we denote the set of perfect representations by $\mathcal{R}^\star$.
\end{notation}

{\color{red}TODO: review the rest of the section based on the new definition of perfect representation.}

An entity may admit more than one perfect representation, even when the oracle is restricted to a particular representational style.

\begin{definition}
\label{def:Re}
Given an entity $e \in \mathcal{E}$, we define the set of \emph{perfect representations of $e$} as
\[
\mathcal{R}^\star_e = \{  r \in \mathcal{R}^\star_{\mathcal{O}_{\mathcal{E}}} | \delta_{\mathcal{E}}(r) = e \}.
\]
\end{definition}

The set $\mathcal{R}^\star_e$ is oracle-relative, generally unknown, and need not be decidable.

\begin{example}
Given a DNA-based\index{DNA} encoding scheme (see Example~\ref{ex:description_dna}), an individual animal may admit multiple valid representations. Owing to the degeneracy of the genetic code, distinct DNA sequences can encode the same proteins through synonymous codons. These alternative sequences differ at the nucleotide level but preserve the information required to reconstruct the same biological entity, as they give rise to identical functional outcomes.
\end{example}

\begin{notation}
We write $r^\star_e \in \mathcal{R}^\star_e$ to indicate that the string $r^\star_e$ is a perfect representation of the entity $e$, whenever such a string is known.
\end{notation}

The sets of perfect representations associated with different entities must be disjoint.

\begin{proposition}
For any two distinct entities \(e_1,e_2 \in \mathcal{E}\) with \(e_1 \neq e_2\) we have that $\mathcal{R}^\star_{e_1} \cap \mathcal{R}^\star_{e_2} = \varnothing$.
\end{proposition}
\begin{proof}
Suppose, for contradiction, that there exists a string \(r \in \mathcal{R}^\star_{e_1} \cap \mathcal{R}^\star_{e_2}\).
By definition of \(\mathcal{R}^\star_{e_1}\), we have \(\delta_{\mathcal{E}}(r)=e_1\). Likewise, by definition of
\(\mathcal{R}^\star_{e_2}\), we have \(\delta_{\mathcal{E}}(r)=e_2\). Since \(\delta_{\mathcal{E}}\) is a function on its domain, it assigns a unique value to each \(r \in \mathcal{R}^\star_{\mathcal{O}_{\mathcal{E}}}\), and therefore \(e_1=e_2\), contradicting the assumption that \(e_1 \neq e_2\). Hence no such \(r\) exists and
\(\mathcal{R}^\star_{e_1} \cap \mathcal{R}^\star_{e_2} = \varnothing\).
\end{proof}

We distinguish between \emph{knowable} and \emph{unknowable} entities (see the gray areas in Figure~\ref{fig:entities_topics_1}; for example, the entity $e_2$ is not encoded by any string). Knowable entities are those that can, in principle, be understood through scientific inquiry (see Section~\ref{sec:what-is-an-entity}), whereas unknowable entities lie beyond the reach of human comprehension, either because they are inherently inaccessible to observation and reasoning or they exceed the cognitive or methodological limits of science.

\begin{definition}
\label{def:knowable_entity}
We say that an entity $e \in \mathcal{E}$ is \emph{knowable}\index{Knowable entity} if there exists at least one $r \in \mathcal{B}^\ast$ such that $\mathcal{O}_\mathcal{E}(r) = true$ and $\delta_\mathcal{E}(r) = e$. An entity $e \in \mathcal{E}$ is \emph{unknowable}\index{Unknowable entity} if it is not knowable.
\end{definition}

A priori, it is not possible to determine whether an entity $e \in \mathcal{E}$ is knowable or unknowable. Identifying suitable knowable entities for study is a matter of trial and error. In this book, we say nothing further about unknowable entities, except to note that their unknowability cannot, in general, be established in advance.

Since our framework restricts representations to finite strings, uncountable sets of entities cannot be injectively encoded using finite binary strings. This situation reflects the fact that, in certain domains of knowledge, the space of possible problems or entities may exceed the space of representations or solutions available to us.

\begin{example}
If the collection of entities under study consists of real numbers, then there exist numbers that cannot be encoded using finite binary strings. This is because the set $\mathbb{R}$ has the cardinality of the continuum, whereas the set $\mathcal{B}^\ast$ is countable.
\end{example}

% Subsection: Extended Set of Representations

\subsection*{Extended Set of Representations}

Since the oracle defines an ideal notion of representation, the set of perfect representations $\mathcal{R}^\star_e$ of a given entity $e \in \mathcal{E}$ can be regarded as the long-term goal of scientific inquiry about that entity. In practice, however, scientific inquiry does not proceed by direct access to elements of $\mathcal{R}^\star_e$, but rather as a discovery process in which increasingly informative approximate representations of $e$ are gradually developed. These approximate representations may omit information required for complete reconstruction, include incorrect or misleading content, or contain symbols that are irrelevant to the representation of $e$.

We say that a string represents a particular entity if it contains a non-trivial amount of algorithmic information about it. This is captured by requiring that conditioning on the string yields a reduction in the Kolmogorov complexity of a perfect representation of the entity.

\begin{definition}
\label{def:representations_of_entity}
Let $e \in \mathcal{E}$ be an entity. We define the set of \emph{representations}\index{Representation} of $e$ as
\[
\mathcal{R}_e =
\left\{
r' \in \mathcal{B}^\ast \;\middle|\;
\exists r^\star \in \mathcal{R}^\star_e \text{ such that }
K(r^\star \mid r') \le K(r^\star) - O(1)
\right\}.
\]
\end{definition}

\noindent
A representation $r'$ is associated with an entity whenever the reduction in the conditional Kolmogorov complexity of a perfect representation $r$ given $r'$, relative to the unconditional complexity of $r$, cannot be attributed solely to technical artifacts such as encoding conventions or machine-dependent implementation details.

\begin{example}
The astronomical records used in the time of Ptolemy\index{Ptolemy} to describe the positions of celestial bodies throughout the year constituted an imperfect encoding of the entity "positions of celestial bodies," due to systematic observational inaccuracies. More accurate encodings were later obtained through the improved observations of Tycho Brahe, and modern astronomy provides even higher-precision representations.
\end{example}

We accept as representations strings that may be extremely poor encodings of the entities under study, and may not support even rudimentary inferences. This reflects early stages of scientific inquiry, where representations are often fragmentary, speculative, or weakly informative, yet still guide exploration and motivate further investigation. The theory of nescience accommodates the progressive refinement of such representations, from highly imperfect encodings to increasingly accurate ones as knowledge advances (see Chapter~\ref{chap:Miscoding}).

Unlike the case of perfect representations, the sets \(\mathcal{R}_e\) associated with different entities need not be disjoint, since a single imperfect representation may contain minimal but genuine information about more than one entity.

\begin{example}
\label{ex:infectious_agents}
Consider the early study of infectious agents, before the distinction between viruses and bacteria was clearly established. At that stage, representations such as "a microscopic agent that causes disease and reproduces inside a host" captured minimal but genuine information about both entities. Such a representation reduces the descriptive complexity of perfect representations of viruses as well as of bacteria, without providing enough information to discriminate between them. This string would therefore be accepted as a representation of both entities, and would belong simultaneously to the sets $\mathcal{R}_{\text{virus}}$ and $\mathcal{R}_{\text{bacterium}}$.
\end{example}

We are interested in the set composed of all possible representations of all knowable entities in $\mathcal{E}$.

\begin{definition}
Let $\mathcal{E}$ be a collection of entities, and for each $e \in \mathcal{E}$ let $\mathcal{R}_e \subseteq \mathcal{B}^\ast$ be the set of representations of $e$. We define the set of \emph{representations}\index{Representation} for $\mathcal{E}$ as
\[
\mathcal{R}_{\mathcal{E}} = \bigcup_{e \in \mathcal{E}} \mathcal{R}_e.
\]
\end{definition}

As in the case of the set of perfect representations $\mathcal{R}^\star_{\mathcal{E}}$, the set $\mathcal{R}_{\mathcal{E}}$ is generally unknown in practice. Although its definition is explicit, it depends on the true structure of the entities under investigation and on the existence of perfect representations, both of which are inaccessible in real scientific contexts, and so, it can't be enumerated or computed in practice.

\begin{notation}
When the set of entities is clear from the context, we denote the set of representations simply by $\mathcal{R}$.
\end{notation}

Strings that do not belong to $\mathcal{R}_{\mathcal{E}}$, that is, strings that do not contain information about any entity in $\mathcal{E}$, correspond to targetless representations\index{Targetless representations}, which are explicitly allowed within the theory of nescience (see Sections~\ref{sec:targetless_representations} and~\ref{sec:scientific_representation}).

% subsubsection: Effective Representations

\subsection*{Effective Representations}

The set of representations $\mathcal{R}$ is not accessible in practice, since it depends both on the true structure of the entities under investigation and on the unknown internal representation function. For this reason, scientific work cannot proceed directly with $\mathcal{R}$, but instead relies on a further level of approximation. We therefore introduce the set $\hat{\mathcal{R}} \subseteq \mathcal{B}^\ast$, consisting of candidate strings that constitute the representations currently available within scientific practice. The elements of $\hat{\mathcal{R}}$ are called \emph{effective representations}: finite descriptions that can be formulated, communicated, and manipulated within existing scientific frameworks, without any guarantee of correctness or completeness. Scientific progress can then be understood as the ongoing evolution of the set of effective representations $\hat{\mathcal{R}}$, which progressively approximates $\mathcal{R}$ and, in an idealized sense, approaches toward $\mathcal{R}^\star$.

\begin{definition}
Let $\mathcal{E}$ be a set of entities. For each entity $e \in \mathcal{E}$, we denote by
\[
\hat{\mathcal{R}}_{e} \subseteq \mathcal{B}^\ast
\]
the set of \emph{effective representations}\index{Effective representation} of $e$, that is, the finite binary strings that are currently taken, within a given scientific context, to represent or approximate the entity $e$.
\end{definition}

The notion of effective representations allows us to distinguish between known and unknown entities, in a representational or effective sense.

\begin{definition}
\label{def:unknown_entity}
An entity $e \in \mathcal{E}$ is said to be \emph{effectively known}\index{Effectively known} if $\hat{\mathcal{R}}_e \ne \varnothing$, and \emph{effectively unknown}\index{Effectively unknown} otherwise.
\end{definition}

In the theory of nescience, the notions of \emph{effective knowledge} and \emph{knowability} refer to two distinct and independent concepts. An entity is said to be \emph{effectively known} if at least one effective representation is available, that is, if $\hat{\mathcal{R}}_e \neq \varnothing$, whereas it is said to be \emph{knowable} if a perfect representation exists, i.e., if $\mathcal{R}^\star_e \neq \varnothing$ (see Definition~\ref{def:knowable_entity}). Effective knowledge therefore concerns what is currently represented and manipulated within scientific practice, while knowability concerns what can, in principle, be fully and correctly represented.

The typical situation in science is one in which an entity is knowable but still effectively unknown. In such cases, a perfect representation exists in principle, yet no effective representation has been discovered or constructed so far. This situation is not exceptional; on the contrary, it characterizes the overwhelming majority of entities in most scientific domains and reflects the open-ended, exploratory nature of scientific inquiry. By contrast, a far more problematic situation arises when an entity is unknowable, yet effectively known. In this case, scientists develop and refine representations of something that, in practice, does not correspond to any real or well-defined entity, as the following example illustrates.

\begin{example}
\label{ex:luminiferous_ether}
The luminiferous ether\index{Luminiferous ether} was a hypothetical medium postulated to explain the propagation of light as a wave. The Michelson--Morley experiment\index{Michelson-Morley experiment} provided strong empirical evidence against the existence of such a medium, and Einstein's theory of special relativity\index{Theory of relativity} later rendered the ether concept unnecessary by postulating the invariance of the speed of light in vacuum. As a result, the ether was abandoned as a physical entity.
\end{example}

The luminiferous ether is an example of effectively known entity that is unknowable, because no perfect representation corresponds to it. In such cases, scientists may develop and refine representations of an entity that ultimately does not exist. Scientific effort is then directed toward something that cannot, even in principle, be reconstructed by the oracle. Unfortunately, there is no practical way to identify such situations in advance using only finite and imperfect representations.

\begin{definition}
Let $\mathcal{E}$ be a set of entities. We define the set of \emph{effectively known representations} for $\mathcal{E}$ as
\[
\hat{\mathcal{R}}_{\mathcal{E}} = \bigcup_{e \in \mathcal{E}} \hat{\mathcal{R}}_{e}.
\]
\end{definition}

\noindent
The set $\mathcal{R}_{\mathcal{E}}$ characterizes the space of all representations that are possible in principle for the entities in $\mathcal{E}$, independently of whether they are currently available or known. By contrast, the set $\hat{\mathcal{R}}_{\mathcal{E}}$ consists only of those effective representations that have actually been discovered, constructed, or adopted within scientific practice at a given time. Thus, $\hat{\mathcal{R}}_{\mathcal{E}}$ should be understood as a time-dependent, practically accessible subset of $\mathcal{R}_{\mathcal{E}}$, whereas $\mathcal{R}_{\mathcal{E}}$ itself remains a conceptual and generally inaccessible object.

\begin{notation}
When the set of entities is clear from the context, we denote the set of effective representations simply by $\hat{\mathcal{R}}$.
\end{notation}

We can also define the set of effectively known entities, that is, the set of those entities for which at least one effective representation exists.

\begin{definition}
Given a set of entities $\mathcal{E}$ and the family of sets of effective representations $\{\hat{\mathcal{R}}_e\}_{e \in \mathcal{E}}$, we define the set of \emph{effectively known entities} as
\[
\hat{\mathcal{E}} = \{ e \in \mathcal{E} \mid \hat{\mathcal{R}}_e \neq \varnothing \}.
\]
\end{definition}

The set $\hat{\mathcal{E}}$ corresponds to entities for which representational activity exists or has existed within scientific practice. This includes entities that are currently under active investigation (e.g., artificial intelligence), entities that were investigated in the past but are no longer central to ongoing research (e.g., the aerodynamics of zeppelins), and entities that were once investigated but later rejected as non-existent or false (e.g., the luminiferous ether).

A further consequence of working with approximate, or effective, representations is that some candidate strings may, unbeknownst to us, fail to encode the entity we intend to study; membership in $\hat{\mathcal{R}}_e$ reflects contextual assignment rather than guaranteed correctness.

\begin{example}
In the late Eighteenth Century, chemist Joseph Priestley believed he was studying a substance called "phlogiston"\index{Phlogiston}, which was thought to be a fire-like element released during combustion. All of his experiments and representations were constructed around this idea. However, in reality, Priestley was observing the properties of a completely different entity: oxygen. Though his descriptions were coherent and reproducible, they were ultimately anchored to the wrong entity.
\end{example}

Effective representations are monotonic over time, in the sense that the set of representations associated with an entity grows as new representations are discovered; representations that are later discarded, rejected, or superseded are not removed from the set.

\begin{notation}
We denote by $\hat{\mathcal{R}}_e^t$ the set of effective representations for an entity $e \in \mathcal{E}$ available at time $t$, and by $\hat{\mathcal{E}}^t$ the set of effectively known entities at time $t$.
\end{notation}

Next proposition proves the monotonicity property of effective representations over time.

\begin{proposition}
Let $\hat{\mathcal{R}}_e^{t_1}$ and $\hat{\mathcal{R}}_e^{t_2}$ denote the set of effective representations associated with an entity $e \in \mathcal{E}$ at times $t_1$ and $t_2$ respectively. Then
\[
\hat{\mathcal{R}}_e^{t_1} \subseteq \hat{\mathcal{R}}_e^{t_2}
\quad \text{for all} \quad t_1 < t_2.
\]
\end{proposition}
\begin{proof}
By assumption, representations discovered at time $t_1$ are retained at all later times, even if they are discarded or superseded, and additional representations may be introduced. Therefore, any effective representation available at time $t_1$ is also available at time $t_2 > t_1$, which implies $\hat{\mathcal{R}}_e^{t_1} \subseteq \hat{\mathcal{R}}_e^{t_2}$.
\end{proof}

As a direct consequence, the set of effectively known entities is itself monotonic over time.

\begin{proposition}
Let $\hat{\mathcal{E}}^{t_1}$ and $\hat{\mathcal{E}}^{t_2}$ denote the set of effectively known entities at times $t_1$ and $t_2$ respectively. Then
\[
\hat{\mathcal{E}}^{t_1} \subseteq \hat{\mathcal{E}}^{t_2}
\quad \text{for all} \quad t_1 < t_2.
\]
\end{proposition}
\begin{proof}
By definition $\hat{\mathcal{E}}^t = \{ e \in \mathcal{E} \mid \hat{\mathcal{R}}_e^t \neq \varnothing \}$. If $e \in \hat{\mathcal{E}}^{t_1}$, then $\hat{\mathcal{R}}_e^{t_1} \neq \varnothing$. By the previous proposition, this implies $\hat{\mathcal{R}}_e^{t_2} \neq \varnothing$ for all $t_2 > t_1$, and therefore $e \in \hat{\mathcal{E}}^{t_2}$.
\end{proof}

The set of effectively known entities $\hat{\mathcal{E}}$ is thus not fixed by $\mathcal{E}$, but by the interaction between entities, representational practices, and available inferential tools. In this sense, scientific progress can be understood as the progressive expansion and refinement of the known subset $\hat{\mathcal{E}}$, together with a gradual reduction of representational ambiguity.

Finally, as in the case of representations themselves, the sets of effective representations associated with different entities need not be disjoint. In particular, for two distinct entities $e_1, e_2 \in \hat{\mathcal{E}}$, it is possible that $\hat{\mathcal{R}}_{e_1} \cap \hat{\mathcal{R}}_{e_2} \neq \varnothing$, reflecting the ambiguity of some effective representations (see Example~\ref{ex:infectious_agents}).

%
% Section: Joint Representations
%

\section{Joint Representations}
\label{sec:joint_representations}


Ideally, scientific work would proceed by using a perfect representation of the entity $e$ under study, that is, an element of the set of perfect representations $\mathcal{R}_{e}^{\star}$. However, since such perfect representations are not accessible in practice, we must instead rely on the approximate representations in $\mathcal{R}_e$. Certain representations are of high quality, in the sense that they encode substantial information about $e$. Yet, $\mathcal{R}_e$ also includes low-quality representations, which convey only limited information about $e$. Moreover, $\mathcal{R}_e$ may include representations that contain incorrect or misleading symbols: fragments that do not correspond to the entity and do not contribute to obtaining a perfect representation. Such errors may mislead us during inquiry if we treat them as valid information.

If we want to increase our knowledge about an entity, we must use representations that are as complete and correct as possible, that is, as close as possible to a perfect representation. One way to achieve this is to try different strings until we discover a high-quality representation. However, this method can be extremely time-consuming and impractical. A more efficient approach is to improve a poor representation by adding missing symbols, removing unnecessary symbols, or combining known representations, each of which contains partial information.

\subsection*{Extended Representations}

A natural way to improve a representation consists of adding symbols that encode the information that it is missing from the representation. This operation, which we call \emph{extension}, plays a fundamental role in the gradual accumulation of knowledge.

\begin{definition}
Let $r, s, t \in \mathcal{B}^*$ be arbitrary binary strings and $e \in \mathcal{E}$ be an entity, and suppose that the concatenated string $rt$ is a representation of $e$, that is, $rt \in \mathcal{R}_{e}$. We call the concatenation $rst$ an \emph{extension}\index{Extensions of represenations} of the representation $rt$, and \emph{extension string}\index{Extension string} to the string $s$.
\end{definition}

An extension may occur at any position within a representation, and both $r$ and $t$ are allowed to be empty. In particular, the usual case of appending symbols at the end of a representation corresponds to $t = \lambda$.

The extension of a representation is a representation itself.

\begin{proposition}
Let $r, s, t \in \mathcal{B}^*$ be arbitrary binary strings such that $rt \in \mathcal{R}_{e}$. Then we have that $rst \in \mathcal{R}_{e}$.
\end{proposition}

\begin{proof}
Assume $rt \in \mathcal{R}_{e}$. By definition of $\mathcal{R}_{e}$, there exists a perfect representation
$r^{\star}\in\mathcal{R}_{e}^{\star}$ such that
\begin{equation}
\label{eq:rt_rep}
K(r^{\star}\mid rt)\le K(r^{\star})-O(1).
\end{equation}

Let $g:\mathcal{B}^*\to \mathcal{B}^*$ be the computable function that, on input the (effectively decodable) concatenation
$rst$, outputs $rt$ (i.e., it discards the middle component $s$ and concatenates the first and third components).\footnote{%
Formally, we may regard $rst$ as a standard computable encoding of the triple $(r,s,t)$ into a single string (e.g., via a
self-delimiting pairing/tupling function). Under any such convention, extracting $(r,s,t)$ and outputting $rt$ is computable.}

A standard monotonicity property of conditional Kolmogorov complexity states that for any computable function $g$,
\begin{equation}
\label{eq:mono}
K(x\mid y)\le K(x\mid g(y)) + O(1),
\end{equation}
because a program that computes $x$ from $g(y)$ can be preceded by a constant-length preprocessor that computes $g(y)$ from $y$.

Applying \eqref{eq:mono} with $x=r^{\star}$ and $y=rst$ gives
\[
K(r^{\star}\mid rst)\le K(r^{\star}\mid g(rst)) + O(1)=K(r^{\star}\mid rt)+O(1).
\]
Combining this with \eqref{eq:rt_rep} yields
\[
K(r^{\star}\mid rst)\le \bigl(K(r^{\star})-O(1)\bigr) + O(1) \;=\; K(r^{\star})-O(1),
\]
where the $O(1)$ term remains a constant independent of the particular strings.

Therefore, there exists $r^{\star}\in\mathcal{R}_{e}^{\star}$ such that
$K(r^{\star}\mid rst)\le K(r^{\star})-O(1)$, and hence $rst\in\mathcal{R}_{e}$ by definition.
\end{proof}

The previous proposition formalizes the intended role of extension: adding correct and relevant information refines a representation without altering its referent.

As a direct consequence of the previous propostion we have the following corollary.

\begin{corollary}
Let $rt \in \mathcal{R}_{\mathcal{E}}$ be a representation of an entity $e \in \mathcal{E}$, and let $rst$ be an extended representation of $rt$. Then $rst \in \mathcal{R}_{\mathcal{E}}$.
\end{corollary}
\begin{proof}
Since $rt \in \mathcal{R}_{e}$ we have that $rst \in \mathcal{R}_{e} \subset \mathcal{R}_{\mathcal{E}}$.
\end{proof}

The extension of a representation does not guarantee uniqueness of reference, as next example shows.

\begin{example}
Consider an entity $e$ corresponding to a physical system described by classical mechanics, and let $rt$ be a representation encoding its mass, position, and velocity. Suppose that the extension string $s$ adds information about relativistic corrections. The extended representation $rst$ may still represent the original classical system under certain approximations, but it may also correspond, for the oracle, to a different entity: a relativistic physical model. In this case, $rst$ simultaneously belongs to $\mathcal{R}_e$ and to $\mathcal{R}_{e'}$ for some distinct entity $e'$.
\end{example}

This example illustrates an important subtlety: while extension preserves representationality and is intended to preserve entity identity, it may introduce enough additional structure for the oracle to associate the resulting string with another entity as well. Extended representations, therefore, may increase expressive power at the cost of potential ambiguity in reference.

\subsection*{Reduced Representations}

An alternative strategy for improving a representation consists of simplifying it by removing symbols that are believed to be incorrect or irrelevant.

\begin{definition}
Let $r, s, t \in \mathcal{B}^*$ be arbitrary binary stringsand $e \in \mathcal{E}$ be an entity, and suppose that the concatenated string $rst$  is a representation of $e$, that is, $rst \in \mathcal{R}_{e}$. We call the string $rt$, obtained by removing the substring $s$, a \emph{reduced representation}\index{Reduced representation} of $rst$.
\end{definition}

As with extension, this definition is intentionally generic: reduction may remove symbols from any position within a representation, and the removed substring $s$ may encode information of any kind.

The goal of a reduced representation is to produce a more accurate representation by eliminating symbols that are misleading, superfluous, or incorrectly encoded. However, unlike extension, reduction may eliminate symbols that are essential for the oracle to reconstruct the entity. Reduced representations are not always representations.

\begin{example}
Let $r$ and $t$ be the empty strings $\lambda$ and $s \in \mathcal{B}^*$ a string such that $rst \in \mathcal{R}_{e}$. Then, the reduced representation $rt$ is the empty string, and so, it is not a representation, since $K(r^{\star}\mid \lambda) = K(r^{\star})$ for all $r^{\star} \in \mathcal{R}^\ast$.
\end{example}

Even when a reduced string remains a valid representation, reduction does not guarantee preservation of the represented entity.

\begin{example}
Consider an entity $e$ corresponding to a specific chemical compound, and suppose that $rst$ encodes its full molecular structure, including stereochemical information. Let the substring $s$ encode the spatial configuration of a chiral center. The reduced representation $rt$, which omits this information, may still correspond to a valid chemical compound, but one that is chemically distinct: a different stereoisomer. In this case, $rt$ represents a different entity $e'$, even though it was obtained by reducing a representation of $e$.
\end{example}

The risk of reduction becomes particularly evident when the removed symbols are not known in advance to be redundant. While extension adds potentially useful information that the oracle may ignore, reduction removes information that the oracle may require. As a result, reduction does not preserve representationality and does not preserve entity identity in general.

\subsection*{Joint Representations}

A third, and particularly powerful, operation on representations consists of combining two existing representations into a single one. This operation, which we call \emph{joint representation}, plays a central role in the theory of nescience, as it provides a systematic mechanism for both refining existing knowledge and discovering new entities.

\begin{definition}
Let $r,s \in \mathcal{R}_{\mathcal{e}}$ be two representations of an entity $e \in \mathcal{E}$. We define the  \emph{joint representation}\index{Joint representation} of $r$ and $s$ as the concatenated string $rs$.
\end{definition}

Joint representations serve two distinct goals. When $r$ and $s$ represent the same entity, their joint representation may yield a more complete or less biased representation of that entity by aggregating complementary information. When $r$ and $s$ represent different entities, their joint representation may correspond to a new entity not previously represented.

\begin{example}
\label{ex:lung_cancer_joint}
Let $e$ be the entity corresponding to the set of causes of lung cancer. Suppose that $r$ is a dataset obtained from a population sample consisting exclusively of male subjects, and that $s$ is a dataset obtained from a population sample consisting exclusively of female subjects. Both $r$ and $s$ belong to $\mathcal{R}_e$, but each constitutes a biased and incomplete representation of the entity. The joint representation $rs$ combines both datasets and yields a more informative and less biased representation of $e$.
\end{example}

With the following propostions we formally study the properties of the concept of joint representations.

\begin{proposition}
Joint representation is associative. That is, for all $r,s,t \in \mathcal{R}_{\mathcal{E}}$, we have
\[
(rs)t = r(st).
\]
\end{proposition}
\begin{proof}
Joint representation is defined as string concatenation. Since string concatenation is associative, the equality $(rs)t = r(st)$ holds for all binary strings, and therefore for all representations.
\end{proof}

Despite its associativity, joint representation is not commutative. A concrete counterexample arises when $r$ encodes a theoretical model and $s$ encodes experimental constraints. The order in which these components are presented may affect how the oracle reconstructs the entity, leading to different outcomes.

Joint representation also satisfies an idempotence property.

\begin{proposition}
For any representation $r \in \mathcal{R}_{\mathcal{E}}$, the joint representation $rr$ represents the same entity as $r$.
\end{proposition}
\begin{proof}
The concatenated string $rr$ contains no information that is not already present in $r$. While redundancy may increase, the oracle can ignore duplicated symbols during reconstruction. Consequently, $rr \in \mathcal{R}_{\mathcal{E}}$ and $\mathcal{O}_{\mathcal{E}}(rr) = \mathcal{O}_{\mathcal{E}}(r)$.
\end{proof}

We now turn to closure properties.

\begin{proposition}
If $r,s \in \mathcal{R}_{\mathcal{E}}$ and $rs \in \mathcal{R}_{\mathcal{E}}$, then $rs$ is a representation.
\end{proposition}
\begin{proof}
This follows directly from the definition of $\mathcal{R}_{\mathcal{E}}$ as the domain of the oracle. Whenever $rs$ belongs to $\mathcal{R}_{\mathcal{E}}$, it is, by definition, a valid representation.
\end{proof}

Although syntactically well defined, joint representation is not guaranteed to preserve entity identity.

\begin{proposition}
Let $r,s \in \mathcal{R}_e$ be representations of the same entity $e$. Then the joint representation $rs$ does not necessarily belong to $\mathcal{R}_e$.
\end{proposition}

\begin{proof}
The joint representation $rs$ may introduce interactions between symbols in $r$ and $s$ that the oracle interprets as defining a different entity. Even if $r$ and $s$ individually encode correct and complementary information about $e$, their combination may cross a representational threshold that leads the oracle to reconstruct a distinct entity $e' \neq e$. Hence, $rs \in \mathcal{R}_{e'}$.
\end{proof}

This phenomenon is not a defect but a defining feature of joint representations.

\begin{example}
Consider a domain in which entities are chemical compounds and representations encode molecular descriptors. Let $r$ represent a compound with anti-inflammatory properties and let $s$ represent a structurally related compound with antiviral properties. While both $r$ and $s$ individually correspond to known compounds, the joint representation $rs$ may encode a novel molecular structure. If the oracle associates $rs$ with a previously uncharacterized compound, then $rs$ represents a new entity.
\end{example}

The concept of joint representation can be extended to any arbitrary, but finite, collection of representations. This allows us to incorporate multiple partial representations into our research or to use them in the process of discovering new entities.

\begin{definition}
Let $r_1, r_2, \ldots, r_n \in \mathcal{R}_\mathcal{E}$ be a finite collection of representations. The \emph{joint representation} of $r_1, r_2, \ldots, r_n$ is defined as the concatenated string $r_1 r_2 \ldots r_n$.
\end{definition}

The possibility that joint representations of known entities give rise to representations of new entities is a critical element of the theory of nescience. It provides a formal mechanism for the discovery of previously unknown entities through the systematic combination of existing knowledge, without requiring access to representations that lie outside the current epistemic horizon.

%
% Section: Descriptions
%

\section{Descriptions}
\label{sec:descriptions_models}

So far, our goal in working with strings from $\mathcal{B}^\ast$ has been to construct encodings, or representations, that are as complete and detailed as possible for the entities in $\mathcal{E}$, regardless of their length. However, as stated in Chapter \ref{chap:Introduction} of this book, human understanding requires the formulation of concise models of these entities, since human reasoning cannot operate effectively on lengthy representations (see Section \ref{sec:intro_descriptions}).

\begin{example}
In Example \ref{ex:lung_cancer}, we showed that a good representation of the entity "lung cancer" could be a dataset in which various risk factors are measured. However, smokers do not decide to quit smoking because they have studied and understood this extensive dataset. Rather, they do so because they understand the much simpler derived model: "smoking increases the risk of lung cancer."
\end{example}

A description or model\footnote{In the theory of nescience, the terms "description" and "model" are used interchangeably.} is a finite binary string that is mapped to a representation of an entity (see Figure \ref{fig:entities_topics_models} in Chapter \ref{chap:Introduction}). Importantly, descriptions do not directly model the entities themselves (i.e., the target systems); instead, they operate on representations of those entities (encodings in the form of strings) serving as approximations of the original entities through those representations.

In the theory of nescience, we require that descriptions be computable, so that the original representations can be fully and effectively reconstructed from them. This requirement of computability allows us to clearly define the limits of the concept of a "description." For example, paradoxes involving self-reference, such as the Berry paradox\index{Berry paradox} (i.e., "the smallest positive integer not definable in less than twelve words," see Section \ref{sec:intro_descriptions}), can be addressed within the framework of computability.

\begin{definition} [Model]
\label{def:description_model}
Let $d \in \mathcal{B}^\ast$ be a binary string of the form $d = \langle TM, a \rangle$, where $TM$ is the encoding of a prefix-free Turing machine and $a$ is the input string to that machine. If $TM(a) \downarrow$, then $d$ is called a \emph{description}\index{Description}.
\end{definition}

Intuitively, a description consists of two parts: a Turing machine that captures and compresses the regularities present in the representation, and a string that contains what remains, that is, the incompressible or random part.

\begin{definition}
\label{def:descriptions_model}
We define the \emph{set of descriptions}\index{Set of descriptions}, denoted by $\mathcal{D}$, as:
\[
\mathcal{D} = \{ d \in \mathcal{B}^\ast : d = \langle TM,a \rangle \wedge TM(a) \downarrow \}.
\]
Let $r \in \mathcal{R}$ be a representation. We define the set of \emph{descriptions for $r$}, denoted by $\mathcal{D}_r$, as:
\[
\mathcal{D}_r = \{ d \in \mathcal{D} : TM(a) = r \}.
\]
Finally, given an entity $e \in \mathcal{E}$, we define the set of \emph{descriptions for $e$}, denoted by $\mathcal{D}_e$, as:
\[
\mathcal{D}_e = \{ d \in \mathcal{D} : \exists r \in \mathcal{R}_e,\, TM(a) = r \}.
\]
\end{definition}

From an ontological point of view, descriptions are string-based objects that satisfy the additional requirement of being computable. In some cases, a description may also qualify as a representation, namely when it is accepted by the oracle as encoding an entity. In this sense, there can exist descriptions that describe other descriptions. However, in practice, it is not advisable to use descriptions as representations of entities, since what we seek in a good representation is the inclusion of as many details as possible about the original entities, rather than a concise encoding. Using descriptions in place of representations would make the task of scientific discovery considerably more difficult.

Since each description corresponds to one, and only one, representation, we can define a function that maps descriptions to representations. Given that descriptions are encoded Turing machines, it is natural to define this mapping using a universal Turing machine. As a result, not only are individual descriptions of representations computable, but the function that maps descriptions to representations is also computable.

\begin{definition}
We call a \emph{description function}\index{Description function}, denoted by $\delta$, a universal Turing machine $\delta : \mathcal{D} \rightarrow \mathcal{B}^\ast$ (which halts on all inputs in $\mathcal{D}$) that maps descriptions to their corresponding representations.
\end{definition}

If $d = \langle TM, a \rangle$ is a description of the representation $r$, then we have that $\delta \left( d \right) = \delta \left( \langle TM, a \rangle \right) = TM(a) = r$.

Inspired by the principle of Occam's razor\index{Occam's razor principle}\footnote{The Occam's razor principle refers to the number of assumptions in an explanation, not to the length of the explanation itself.}, if two explanations are equivalent, we should prefer the shorter one. Accordingly, the limit of what can be known, or understood, about a representation, that is, its perfect model, is given by the shortest description that allows us to reconstruct that representation.

\begin{definition}
\label{def:descriptions_perfect_model}
Let $\mathcal{D}_r$ be the set of descriptions of a representation $r \in \mathcal{R}$, and let $d \in \mathcal{D}_r$ be a description of $r$. We say that $d$ is a \emph{perfect description}\index{Perfect description} of the representation $r$ if there is no other description $d' \in\mathcal{D}_r$ such that $l(d') < l(d)$.
\end{definition}

Recall that what we know about an entity $e$ depends on the quality of the representation $r$ used. If the representation $r$ is incorrect, we cannot achieve perfect knowledge of $e$, even if we have found the perfect description $d$ for $r$.

\begin{notation}
We denote by $d_r^{\star}$ the fact that the description $d$ is a perfect description of the representation $r$.
\end{notation}

The perfect description of a representation may not be unique; that is, there could be multiple optimal ways to compute $r$.

\begin{definition}
\label{def:set_descriptions_perfect_model}
Let $\mathcal{D}_r$ be the set of descriptions of a representation $r \in \mathcal{R}_\mathcal{E}$. We define the \emph{set of perfect descriptions} for $r$, denoted by $\mathcal{D}^\star_r$, as the subset of $\mathcal{D}_r$ consisting of all perfect descriptions of $r$.
\end{definition}

Unfortunately, the set of perfect descriptions of a representation is generally unknown, and as Proposition \ref{prop:nescience-kolmogorov} shows, there exists no algorithm to compute it. In practice, we must rely on approximations to estimate how far our current best description is from a perfect one, that is, to quantify how much we do not know about a particular representation of an entity (see Chapter \ref{chap:Redundancy}).

\begin{proposition}
\label{prop:nescience-kolmogorov}
Let $r \in \mathcal{B}^\ast$ be a representation and let $d_r^{\star}$ be a perfect description of $r$. Then we have $l \left( d_r^{\star} \right) = K\left( r \right)$.
\end{proposition}
\begin{proof}
Apply Definition \ref{def:Kolmogorov-Complexity} and note that the Turing machines $TM$ used in descriptions of the form $\langle TM, a \rangle$ are required to be prefix-free.
\end{proof}

The actual length of a description $l(d)$ for a representation $r$ depends on the specific encoding of Turing machines used. This encoding method is determined by the chosen description function $\delta$. Fortunately, if we replace our description function with a different one, the length of perfect descriptions remains essentially unchanged, up to an additive constant that does not depend on the representation itself.

\begin{corollary}
Let $r \in \mathcal{R}$ be a representation, and let $\delta$ and $\dot{\delta}$ be two different description functions. Let $d_r^{\star}$ be a perfect description of $r$ under $\delta$, and $\dot{d}_r^{\star}$ a perfect description under $\dot{\delta}$. Then $\mid l \left( d_r^{\star} \right) - l \left( \dot{d}_r^{\star} \right) \mid \leq c
$, where $c$ is a constant that does not depend on $r$.
\end{corollary}
\begin{proof}
Apply Proposition \ref{prop:nescience-kolmogorov} and Theorem \ref{def:Invariance-theorem}.
\end{proof}

In practice, within the theory of nescience, we are often not interested in computing the exact numerical value of the nescience associated with an entity for a given description–representation pair. Rather, our goal is to determine a weaker but sufficient notion: the relative ordering of different pairs of descriptions and representations according to their nescience. From this perspective, the specific details of the universal Turing machine used to compute description lengths are not essential\footnote{Do not confuse the internal workings of the universal Turing machine that maps descriptions to representations, which are not of interest here, with the internal workings of the universal oracle Turing machine that maps representations to entities, which are of interest, as understanding this mechanism is crucial to understanding how things work.}. Accordingly, for the remainder of this book we assume that the description function $\delta$ is fixed to a reference universal Turing machine. Alternatively, the reader may interpret all theorems in this book involving the length of the shortest models as holding up to an additive constant that does not depend on the topics themselves.

A remarkable consequence of Proposition \ref{prop:nescience-kolmogorov} is that perfect descriptions must be incompressible; that is, \emph{perfect knowledge implies randomness} (see Section \ref{sec:incompressibility_randomness}).

\begin{corollary}
Let $d_r^{\star}$ be a perfect description of a representation $r$. Then $d_r^{\star}$ is incompressible, in the sense that
\[
K\!\left(d_r^{\star}\right) \geq l\!\left(d_r^{\star}\right) - c,
\]
for some constant $c$ independent of $r$.
\end{corollary}
\begin{proof}
By Proposition \ref{prop:nescience-kolmogorov}, the length of a perfect description satisfies
\[
l\!\left(d_r^{\star}\right) = K(r),
\]
up to an additive constant fixed by the choice of universal Turing machine. Suppose that $d_r^{\star}$ were compressible, that is, that there existed a description of $d_r^{\star}$ of length strictly smaller than $l(d_r^{\star}) - c$, for arbitrarily large $c$. Then this shorter description could be used to construct a description of $r$ shorter than $d_r^{\star}$, contradicting the minimality of $d_r^{\star}$ as a perfect description. Hence, $d_r^{\star}$ must be incompressible.
\end{proof}

The converse does not generally hold: a description can be random without being the shortest possible one (incompressibility does not imply minimality). That is, we may have a description $d$ of a representation $r$ such that $l(d) = K(d)$, yet $l(d_r^{\star}) < l(d)$.

\begin{example}
\label{ex:description_neural}
Consider a deep neural network\index{Neural network} with an input layer of one thousand nodes, ten hidden layers of fifty thousand nodes each, and an output layer of one thousand nodes. Suppose the network is trained to output a fixed string of one thousand 1's for any given input. The Kolmogorov complexity of a string encoding of the trained neural network is much greater than that of the output string itself, which consists of one thousand identical bits.
\end{example}

The concept of a perfect description can be generalized from individual representations to entire entities. This generalization allows us to study the nature and properties of the entities themselves.

\begin{definition}
\label{def:entities_perfect_model}
Let $\mathcal{D}_e$ be the set of descriptions of an entity $e \in \mathcal{E}$. We define the \emph{set of perfect descriptions} of the entity $e$, denoted by $\mathcal{D}^\star_e$, as the subset of $\mathcal{D}_e$ consisting of perfect descriptions. The elements of $\mathcal{D}^\star_e$ are denoted by $d^\star_e$.
\end{definition}

If $d^\star_e \in \mathcal{D}^\star_e$ there must exist a representation $r \in \mathcal{R}^\star_e$ such that $d^\star_e \in \mathcal{D}^\star_r$.

Technically speaking, we could have descriptions that are longer than the representations they describe, that is, descriptions that do not compress the representations. However, they are epistemically uninformative, since they fail to achieve any compression.

\begin{definition}
\label{def:trivial_model}
Let $r \in \mathcal{B}^\ast$ be a representation, and $d \in \mathcal{D}_r$ one of its descriptions. If $l(d) \geq l(r)$, we say that $d$ is a \emph{pleonastic description}\index{Pleonastic description} of the representation $r$.
\end{definition}

\begin{example}
\label{ex:topics_models_graph}
Consider the set of all possible finite graphs\index{Graph}. Since graphs are abstract mathematical objects, we must represent them as strings, for instance, using a binary encoding of their adjacency matrices (see Section \ref{sec:Graphs} for an introduction to graphs). The description $d = \langle TM, r \rangle$, where $r$ is the representation of a graph and $TM$ is a Turing machine that simply halts, belongs to $\mathcal{D}_r$ because $TM(r) = r$. However, this description is of limited interest, as it is likely not the shortest possible description of $r$.
\end{example}

It may happen that there is no shorter possible description of a representation than the representation itself. This occurs when the representation is an incompressible string. As discussed in Section \ref{sec:incompressibility_randomness}, the overwhelming majority of strings are incompressible. Conducting research on incompressible representations is unproductive, as it is not possible to find shorter models for such representations.

An interesting case arises when all the descriptions in $\mathcal{D}_e$ are pleonastic, that is, there exist no models shorter than the representation for any of the possible representations of the entity. This situation would occur if all representations of the entity $e$ are random strings. In such a case, scientific research would be fundamentally limited, as it would be impossible to find a suitable model for $e$. Our ability to understand and make predictions about $e$ would then be constrained by the length of its incompressible representations.

%
% Section: Models for Joint Representations
%

\section{Descriptions for Joint Representations}
\label{sec:description_joint_represenation}

In Section \ref{sec:descriptions_joint_topic}, we introduced the concept of a joint representation $ts$, formed by combining two individual representations $t$ and $s$. In this section, we aim to study how the length of the perfect description of a joint representation relates to the lengths of the perfect descriptions of the individual representations.

The length of the perfect description of a joint representation is greater than or equal to the length of the perfect description of either individual representation. In other words, the more information a representation contains, the longer it takes to describe.

\begin{proposition}
\label{prop:joint_length}
Let $t,s \in \mathcal{R}_\mathcal{E}$ be two representations, and let $m_{t}^{\star}$, $m_{s}^{\star}$, and $m_{ts}^{\star}$ denote the perfect descriptions of the representations $t$, $s$, and the joint representation $ts$, respectively. Then: $l \left( m_{ts}^{\star} \right) \geq l \left( m_{t}^{\star} \right) \quad \text{and} \quad l \left( m_{ts}^{\star} \right) \geq l \left( m_{s}^{\star} \right).$
\end{proposition}
\begin{proof}
The inequality $l \left( m_{ts}^{\star} \right) \geq l \left( m_{t}^{\star} \right)$ is equivalent to $K(ts) \geq K(t)$. The result then follows from Proposition \ref{prop:excess_kolmogorov}.
\end{proof}

Intuitively, adding more information to a representation is beneficial if the additional information is relevant to describing the entity of interest. However, including irrelevant information leads to unnecessarily long models. Recall that joining representations can serve either to concatenate two partial representations of the same entity or to enrich a representation by adding missing symbols.

If the selected representations partially overlap, we can exploit this redundancy to produce a joint description that is shorter than the mere concatenation of the individual descriptions. In the worst-case scenario, the perfect description of a joint representation would be equal in length to the sum of the perfect descriptions of the individual representations.

\begin{proposition}
\label{prop:joint_sum}
Let $t, s \in \mathcal{R}_\mathcal{E}$ be two representations, and let $m_{t}^{\star}$, $m_{s}^{\star}$, and $m_{ts}^{\star}$ denote the perfect descriptions of the representations $t$, $s$, and the joint representation $ts$, respectively. Then: $l \left( m_{ts}^{\star} \right) \leq l \left( m_{t}^{\star} \right) + l \left( m_{s}^{\star} \right).$
\end{proposition}
\begin{proof}
The inequality $l \left( m_{ts}^{\star} \right) \leq l \left( m_{t}^{\star} \right) + l \left( m_{s}^{\star} \right)$ is equivalent to $K(ts) \leq K(t) + K(s)$. The result follows from Proposition \ref{prop:additive_kolmogorov}.
\end{proof}

One interpretation of Proposition \ref{prop:joint_sum} is that including redundant information in the representation of an entity does not hinder our ability to find its shortest possible description. From the perspective of compression, redundancy can be eliminated during the modeling process. Therefore, in practice, we may prefer to work with representations that are longer but make the process of scientific discovery, i.e., finding the best model, easier, even if they contain superfluous information. In contrast, Proposition \ref{prop:joint_length} highlights a different concern: adding irrelevant or non-informative symbols to a representation should be avoided, as they increase the complexity of the description without contributing useful information about the entity.

Finally, the following proposition shows that the order of the representations in the perfect description of a joint representation does not affect its length.

\begin{proposition}
\label{prop:joint_order}
Let $t, s \in \mathcal{R}_\mathcal{E}$ be two representations, and let $m_{ts}^{\star}$ and $m_{st}^{\star}$ be the perfect descriptions of the joint representations $ts$ and $st$, respectively. Then: $l \left( m_{ts}^{\star} \right) = l \left( m_{st}^{\star} \right).$
\end{proposition}
\begin{proof}
The equality $l \left( m_{ts}^{\star} \right) = l \left( m_{st}^{\star} \right)$ is equivalent to $K(ts) = K(st)$. The result follows from Proposition \ref{prop:kolmogorov_order}.
\end{proof}

It is important to note, however, that joining representations is not a commutative operation, there is no guarantee that the strings $ts$ and $st$ encode the same entity. Moreover, given only the concatenated string $ts$, it is generally not possible to recover the original representations $t$ and $s$, since they are not self-delimiting.

Propositions \ref{prop:joint_length}, \ref{prop:joint_sum} and \ref{prop:joint_order} can be generalized to any arbitrary, but finite, collection of representations $t_1, t_2, \ldots, t_n$.

\begin{proposition}
\label{prop:joint_multiple_topics}
Let $t_1, t_2, \ldots, t_n \in \mathcal{R}_\mathcal{E}$ be a finite collection of representations. Then, we have that:

\renewcommand{\theenumi}{\roman{enumi}}
\begin{enumerate}
\item $l(m_{t_1 t_2 \ldots t_n}^\star) \geq l(m_ {t_i}^\star) \; \forall \, 1 \leq i \leq n$,
\item $l(m_{t_1 t_2 \ldots t_n}^\star) \leq l(m_ {t_1}^\star) + l(m_ {t_2}^\star) + \ldots + l(m_ {t_n}^\star)$,
\item $l(m_{t_1 \ldots t_i \ldots t_j \ldots t_n}^\star) = l(m_{t_1 \ldots t_j \ldots t_i \ldots t_n}^\star) + c \; \forall \, 1 \leq i \leq j \leq n$,
\item $l(m_{t_1 \ldots t_{n-1}}^\star) \leq l(m_{t_1 \ldots t_{n-1} t_n}^\star)$.
\end{enumerate}
\end{proposition}
\begin{proof}
Apply Propositions \ref{prop:joint_length}, \ref{prop:joint_sum} and \ref{prop:joint_order} to individual pairs of representations $i$ and $j$.
\end{proof}

%
% Section: Conditional Descriptions
%

\section{Conditional Descriptions}

It is often cumbersome to include all the information required to reconstruct an entity within a single description, as this would typically result in very long strings for most entities. A more practical approach is to assume the existence of some background knowledge and to quantify our lack of knowledge about an entity relative to that background. In this section, we study the concept of \emph{conditional descriptions}, that is, constructing a description given some prior description. Conditional descriptions also play a crucial role in the discovery of new knowledge: if conditioning a description on some prior knowledge significantly reduces the inaccuracy of a model, it indicates that this prior knowledge is relevant to understanding the entity.

\begin{definition}
\label{def:conditional_description}
Let $r, d, s \in \mathcal{B}^\ast$ be strings. We say that the string $\langle d, s \rangle$ is a \emph{valid conditional description}\index{Conditional model} of the representation $r$ given the string $s$, denoted by $d_{r \mid s}$, if $d = \langle TM, a \rangle$ is a description, and $TM \left(\langle a, s \rangle \right) = r$.
\end{definition}

The conditional description $d_{r \mid s}$ relies on two distinct strings: $a$ and $s$, each fulfilling a different role. The string $a$ is provided as input to the Turing machine $TM$ and is intended to contain the portion of the representation $r$ that cannot be derived from prior knowledge, that is, the incompressible or novel part. In contrast, the string $s$ represents background knowledge: it is a description or representation of another entity that is assumed to be already known and that can facilitate the reconstruction or understanding of $r$. For example, as we will explain in Chapter \ref{chap:Redundancy}, when evaluating the redundancy of a conditional description, the contribution of the string $s$ is disregarded—only the length and content of $a$ are taken into account.

Note that the conditional description $d_{r \mid s}$ does not belong to the set of valid descriptions $\mathcal{D}$ for the representation $r$, since computing $r$ requires the additional string $s$, which is not part of the description itself. Therefore, a new definition is needed to formally capture this concept.

\begin{definition}
Let $r \in \mathcal{B}^\ast$ be a representation and $s \in \mathcal{B}^\ast$ an arbitrary string. We define the \emph{set of conditional descriptions}\index{Set of conditional descriptions} of $r$ given $s$, denoted by $\mathcal{D}_{r \mid s}$, as:
\[
\mathcal{D}_{r \mid s} = \{ d \in \mathcal{B}^\ast, d = \langle TM, a \rangle : TM \left(\langle a, s \rangle \right) = r \}.
\]
\end{definition}

For each representation $r \in \mathcal{B}^\ast$, there always exists a conditional description $d_{r \mid s}$ that describes $r$, as the following proposition shows.

\begin{proposition}
\label{prop:description_implies_conditional}
Let $r \in \mathcal{B}^\ast$ be a representation and $s \in \mathcal{B}^\ast$ an arbitrary string. If $d \in \mathcal{D}_{r}$ then $d \in \mathcal{D}_{r \mid s}$.
\end{proposition}
\begin{proof}
We can construct a conditional description $\langle \langle TM, a \rangle, s \rangle$ based on a Turing machine $TM$ such that, when given the input $\langle a, s \rangle$, the machine safely ignores the string $s$.
\end{proof}

The converse of Proposition \ref{prop:description_implies_conditional} is not true. The fact that $d$ is a conditional description (i.e., $d \in \mathcal{D}_{r \mid s}$) does not implies that $d$ is also a valid description (i.e., $d \in \mathcal{D}_{r})$. Indeed, while we require that $TM \left(\langle a, s \rangle \right) = r$, we do not require that $TM(a) = r$, and in general, this may not hold.

We are interested in the concept of a perfect conditional description. The perfect conditional description of a representation, given some prior knowledge, is the shortest possible string that allows us to fully reconstruct the representation, assuming that the prior knowledge is already known.

\begin{definition}
Let $r \in \mathcal{B}^\ast$ be a representation, and let $d^\star_{r \mid s}$ be the shortest possible description of $r$ given the string $s$. We call $d^\star_{r \mid s}$ the \emph{perfect conditional description}\index{Perfect conditional description} of the representation $r$ given the string $s$, or simply perfect conditional description of $r$ given $s$ for short.
\end{definition}

Note that $d^\star_{r \mid s}$ is a perfect description of the representation $r$ *conditional on* the string $s$. This does not imply that $s$ is a perfect description itself; it may be an incomplete or partially irrelevant representation. In such a case, we would have achieved perfect knowledge with respect to the $d$ component, but not with respect to the $s$ component of the combined string $\langle d, s \rangle$.

The length of a perfect conditional description is always less than or equal to that of its unconditional counterpart. In other words, assuming the existence of some background knowledge can reduce the effort required to describe a representation.

\begin{proposition}
\label{prop:description_conditional_inequality}
Let $r \in \mathcal{B}^\ast$ be a representation and $s \in \mathcal{B}^\ast$ an arbitrary string. Then $l \left( d^\star_{r \mid s} \right) \leq l \left( d^\star_r \right)$.
\end{proposition}
\begin{proof}
The inequality $l \left( d^\star_{r \mid s} \right) \leq l \left( d^\star_r \right)$ is equivalent to the well-known result $K(r \mid s) \leq K(r)$. The proposition follows directly by applying Proposition \ref{prop:kolmogorov_conditional}.
\end{proof}

The notions of unconditional, conditional, and joint descriptions are closely related. In particular, the availability of prior knowledge (as captured by a conditional description) can reduce the length of a description, while describing multiple entities jointly (via a joint description) typically requires more information than describing a single entity. The following proposition formalizes these relationships by comparing the lengths of the perfect conditional description, the perfect (unconditional) description, and the perfect joint description.

\begin{proposition}
\label{prop:description_conditional_joint}
Let $r, s \in \mathcal{B}^\ast$ two different representations. Then:
\[
l \left( d^\star_{r \mid s} \right) \leq l \left( d^\star_r \right) \leq l \left( d^\star_{rs} \right)
\]
\end{proposition}
\begin{proof}
The inequality $l \left( d^\star_{r \mid s} \right) \leq l \left( d^\star_r \right) \leq l \left( d^\star_{rs} \right)$ is equivalent to the Kolmogorov complexity relations $K(r \mid s) \leq K(r)$ and $K(r) \leq K(rs)$. The result follows directly by applying Proposition \ref{prop:kolmogorov_relations}.
\end{proof}

As was the case with joint descriptions, the concept of conditional description can be naturally extended to finite collections of representations.

\begin{definition}
Let $r, d, s_1, s_2, \ldots, s_n \in \mathcal{B}^\ast$ be strings. We say that the string $\langle d, s_1, s_2, \ldots, s_n \rangle$ is a \emph{valid conditional description}\index{Conditional model} of the representation $r$ given the strings $s_1, s_2, \ldots, s_n$, denoted by $d_{r \mid s_1, s_2, \ldots, s_n}$, if $d = \langle TM, a \rangle$ is a description, and $TM \left(\langle a, s_1, s_2, \ldots, s_n \rangle \right) = r$.
\end{definition}

The following definition generalizes the notion of a perfect conditional description to the case of multiple conditioning strings.

\begin{definition}
Let $r \in \mathcal{B}^\ast$ be a representation, and let $d^\star_{r \mid s_1, s_2, \ldots, s_n}$ be the shortest possible description of $r$ given the strings $s_1, s_2, \ldots, s_n$. We call $d^\star_{r \mid s_1, s_2, \ldots, s_n}$ the \emph{perfect conditional description}\index{Perfect conditional description} of the representation $r$ given the string $s_1, s_2, \ldots, s_n$, or perfect conditional description of $r$ given $s_1, s_2, \ldots, s_n$ for short.
\end{definition}

The next proposition generalizes Propositions \ref{prop:description_conditional_inequality} and \ref{prop:description_conditional_joint} to any arbitrary (but finite) collection of strings $s_1, s_2, \ldots, s_n$. In particular, it shows that the more background knowledge we assume for a given representation, the shorter its perfect description becomes.

\begin{proposition}
Let $r, s_1, s_2, \ldots, s_n \in \mathcal{B}^\ast$ be a finite collection of strings. Then:
\[
l \left( d^\star_{r \mid s_1, s_2, \ldots, s_n} \right) \leq l \left( d^\star_r \right) \leq l \left( d^\star_{r,s_1, s_2, \ldots, s_n} \right)
\]
\end{proposition}
\begin{proof}
This follows from the Kolmogorov complexity inequalities $K(r \mid s_1, s_2, \ldots, s_n) \leq K(r) \leq K(r, s_1, s_2, \ldots, s_n)$, which generalize the results stated in Propositions \ref{prop:description_conditional_inequality} and \ref{prop:description_conditional_joint}.
\end{proof}

The following proposition further generalizes the idea that assuming additional background knowledge cannot increase the length of a perfect conditional description.

\begin{proposition}
Let $r, s_1, s_2, \ldots, s_n, s_{n+1} \in \mathcal{B}^\ast$ be a finite collection of strings. Then:
\[
l \left( d^\star_{r \mid s_1, s_2, \ldots, s_n, s_{n+1}} \right) \leq l \left( d^\star_{r \mid s_1, s_2, \ldots, s_n} \right)
\]
\end{proposition}
\begin{proof}
This follows directly from the monotonicity property of conditional Kolmogorov complexity: adding more conditioning information cannot increase the complexity. Formally, $K(r \mid s_1, s_2, \ldots, s_n, s_{n+1}) \leq K(r \mid s_1, s_2, \ldots, s_n)$.
\end{proof}

%
% Section: Areas
%

\section{Research Areas}
\label{sec:areas}

Entities can be grouped into research areas. The concept of an area is useful insofar as all the entities included in the area are related to a common subdomain of knowledge or share a common property. The specific criteria used for grouping depend on the practical application of the theory of nescience.

\begin{definition}
Given a set of entities $\mathcal{E}$, we define a \emph{research area}\index{Research area} $\mathcal{A}$ as a subset of entities, $\mathcal{A} \subset \mathcal{E}$.
\end{definition}

If we want to quantify how much we do not know about a research area, we must first provide a representation for that area. In general, research areas are infinite, but the number of known representations is finite. Therefore, we can only describe an area with respect to our current state of knowledge.

\begin{definition}
Let $\mathcal{A} \subset \mathcal{E}$ be a research area. We define the \emph{known subset of the area}\index{Known subset of an area} $\mathcal{A}$, denoted by $\hat{\mathcal{A}}$, as the set of entities $e_1, e_2, \ldots, e_n \in A$ for which at least one non-pleonastic description is known.
\end{definition}

We must distinguish between the knowable subset of $\mathcal{A}$, composed of those entities for which a representation exists, and the known subset of $\mathcal{A}$, composed of those entities for which at least one non-pleonastic description is known, that is, entities about which some research has already been conducted. Clearly, the set of known entities is a subset of the set of knowable entities.

As our understanding of a research area evolves, the number of entities included in its known subset also changes. Throughout this book, the properties of research areas will always be considered relative to our current state of knowledge.

\begin{definition}
Let $\mathcal{A} \subset \mathcal{E}$ be a research area with known subset $\hat{\mathcal{A}} = \{ e_1, e_2, \ldots, e_n \}$, and let $R_{\hat{\mathcal{A}}} = \{ r_1, r_2, \ldots, r_n \}$ be a set of representations such that $r_i \in \mathcal{R}_{e_i}$. We call $R_{\hat{\mathcal{A}}}$ a \emph{representation of the area $\mathcal{A}$}\index{Representation of an area} given the known subset $\hat{\mathcal{A}}$, abbreviated as \emph{representation of $\mathcal{A}$}.
\end{definition}

In a similar manner to how we describe individual entities, we can also introduce the concept of a description for an entire research area. Since a research area is represented by a collection of known representations corresponding to its known subset, a description of the area must account for the generation of this entire set. Thus, we define a description of a research area as a program that, when executed, produces the sequence of representations associated with the known entities in that area.

\begin{definition}
Let $R_{\hat{\mathcal{A}}} = \{ r_1, r_2, \ldots, r_n \}$ be the representation of an area $\mathcal{A}$. We call a \emph{description of the area $\mathcal{A}$}\index{} given the known subset $\hat{\mathcal{A}}$, abbreviated as \emph{description of $\mathcal{A}$}, and denoted by $d_{\hat{\mathcal{A}}}$, to any string in the form $\langle TM, a\rangle$ such that the Turing machine $TM$, when given input $a$, outputs the sequence $\langle r_1, r_2, \ldots, r_n \rangle$.
\end{definition}

We can also consider all possible descriptions that generate the full set of known representations for a given research area. These descriptions differ in structure, length, or computational efficiency, but they all produce the same output: the sequence of representations associated with the known entities in the area.

\begin{definition}
Let $R_{\hat{\mathcal{A}}} = \{ r_1, r_2, \ldots, r_n \}$ be the representation of an area $\mathcal{A}$. We define the set of \emph{descriptions for $R_{\hat{\mathcal{A}}}$}, denoted by $\mathcal{D}_{R_{\hat{\mathcal{A}}}}$, as:
\[
\mathcal{D}_{R_{\hat{\mathcal{A}}}} = \{ d \in \mathcal{D} : TM(a) = \langle r_1, r_2, \ldots, r_n\rangle \}.
\]
\end{definition}

Finally, we are interested in identifying the perfect model for a research area, that is, the shortest possible string that fully describes its known subset. According to Definition \ref{def:trivial_model}, if we are aware of the existence of an entity $e \in A$, then $e$ should be included in the known subset $\hat{A}$, even if no research has yet been conducted on that specific topic.

\begin{definition}
Let $A \subset \mathcal{E}$ be an area with known subset $\hat{A}$, and let $d_{\hat{A}}^{\star} \in 
\mathcal{D}_{R_{\hat{\mathcal{A}}}}$ be the shortest possible description of $A$. We call  $d_{\hat{A}}^{\star}$ the \emph{perfect description of the area $A$}\index{Perfect description of an area} given the known subset $\hat{A}$\index{Perfect description of an area}, abbreviated as \emph{perfect description of $A$}.
\end{definition}

The following proposition shows the relationship between the description of a research area and the descriptions of the individual entities that compose its known subset. In general, the models for an area are not simply the collection of the models for each individual topic; instead, a joint model may offer a more concise description.

\begin{proposition}
Let $A \subset \mathcal{E}$ be an area with known subset $\hat{A} = \{e_1, e_2, \ldots, e_n\}$, then we have that $l \left( d_{\hat{A}}^{\star} \right) \leq l(d_ {e_1}^\star) + l(d_ {e_2}^\star) + \ldots + l(d_ {e_n}^\star)$.
\end{proposition}
\begin{proof}
Apply Proposition \ref{prop:joint_multiple_topics}-ii. 
\end{proof}

Moreover, as shown in Proposition \ref{prop:joint_multiple_topics}, the order in which the representations are listed in the description of an area does not affect the length of its perfect model.

Research areas can overlap; that is, given two areas $A$ and $B$, it may be the case that $A \cap B \neq \varnothing$. Furthermore, one area can be a subset of another, forming a hierarchy of areas. In this context, we are particularly interested in how the length of perfect models for some areas compares to the length of perfect models for related areas.

\begin{proposition}
Let $A, B \subset \mathcal{E}$ be two areas such that $A \subset B$, and let $\hat{A}$ and $\hat{B}$ be their know subsets respectively, then we have that $l \left( d_{\hat{A}}^{\star} \right) \leq l \left( d_{\hat{B}}^{\star} \right)$.
\end{proposition}
\begin{proof}
Since $A \subset B$, it follows that $\hat{A} \subset \hat{B}$.Let
$$
R_{\hat{A}} = \{r_1, r_2, \ldots, r_m\} \quad \text{and} \quad R_{\hat{B}} = \{r_1, r_2, \ldots, r_m, r_{m+1}, \ldots, r_n\}
$$
be the sets of representations corresponding to $\hat{A}$ and $\hat{B}$ respectively, and let $d_{\hat{B}}^{\star}$ be the perfect description of $\hat{B}$. Then, we can construct a description $d'$ of $\hat{A}$ by modifying $d_{\hat{B}}^{\star}$ to output only the subset $R_{\hat{A}}$. This can be achieved by appending a simple postprocessing step that discards the extra representations. The additional cost of this truncation is at most a constant number of bits, independent of the specific contents of $R_{\hat{B}}$.

Formally, we have:
$$
l(d_{\hat{A}}^{\star}) \leq l(d') \leq l(d_{\hat{B}}^{\star}) + c
$$
for some constant $c$. But since $d_{\hat{A}}^{\star}$ is the shortest possible description of $R_{\hat{A}}$, we conclude:
$$
l(d_{\hat{A}}^{\star}) \leq l(d_{\hat{B}}^{\star}).
$$
\end{proof}

The following proposition shows how the length of the shortest possible description of two areas relates to the length of the description of their union and intersection.

\begin{proposition}
\label{prop:areas_union}
Let $A, B \subset \mathcal{E}$ be two areas with know subsets $\hat{A}$ and $\hat{B}$ respectively, then we have that $l \left( d_{\hat{A} \cup \hat{B}}^{\star} \right) = l \left( d_{\hat{A}}^{\star} \right) + l \left( d_{\hat{B}}^{\star} \right) - l \left( d_{\hat{A} \cap \hat{B}}^{\star} \right)$.
\end{proposition}
\begin{proof}
Let $R_{\hat{A}}$, $R_{\hat{B}}$, and $R_{\hat{A} \cap \hat{B}}$ be the sets of representations corresponding to the known subsets $\hat{A}$, $\hat{B}$, and $\hat{A} \cap \hat{B}$, respectively.

From the theory of Kolmogorov complexity, the minimal description length of the union of two finite sets of strings satisfies the following identity:
$$
K(R_{\hat{A} \cup \hat{B}}) = K(R_{\hat{A}}) + K(R_{\hat{B}}) - K(R_{\hat{A} \cap \hat{B}}),
$$
Since, by definition, the perfect description $d_{\hat{A}}^\star$ satisfies:
$$
l(d_{\hat{A}}^\star) = K(R_{\hat{A}}), \quad l(d_{\hat{B}}^\star) = K(R_{\hat{B}}), \quad l(d_{\hat{A} \cap \hat{B}}^\star) = K(R_{\hat{A} \cap \hat{B}}),
$$
it follows that:
$$
l(d_{\hat{A} \cup \hat{B}}^\star) = K(R_{\hat{A} \cup \hat{B}}) = l(d_{\hat{A}}^\star) + l(d_{\hat{B}}^\star) - l(d_{\hat{A} \cap \hat{B}}^\star).
$$
\end{proof}

A consequence of Proposition \ref{prop:areas_union} is that $l \left( d_{\hat{A} \cup \hat{B}}^{\star} \right) \leq l \left( d_{\hat{A}}^{\star} \right) + l \left( d_{\hat{B}}^{\star} \right)$, that is, when we combines two different research areas, how much we do not know about these areas decreases.

Just as we introduced a chain rule for entropy in Proposition \ref{prop:chain_rule_entropy}, we can also establish a chain rule for the shortest length of a description of a research area.

\begin{proposition}
Let $A, B \subset \mathcal{E}$ be two areas with know subsets $\hat{A}$ and $\hat{B}$, then we have that $l \left( d_{\hat{A} \cup \hat{B}}^{\star} \right) = l \left( d_{\hat{A}}^{\star} \right) + l \left( d_{\hat{B} \backslash \hat{A}}^{\star} \right)$.
\end{proposition}
\begin{proof}
Let $R_{\hat{A}}$ be the set of representations associated with $\hat{A}$, and let $R_{\hat{B} \backslash \hat{A}}$ be the set of representations corresponding to entities in $\hat{B}$ that are not in $\hat{A}$.

By definition, the known subset of the union $\hat{A} \cup \hat{B}$ corresponds to the set of representations:
$$
R_{\hat{A} \cup \hat{B}} = R_{\hat{A}} \cup R_{\hat{B} \backslash \hat{A}}.
$$
Let $d_{\hat{A}}^{\star}$ be the shortest (perfect) description that generates $R_{\hat{A}}$, and let $d_{\hat{B} \backslash \hat{A}}^{\star}$ be the shortest description that generates $R_{\hat{B} \backslash \hat{A}}$. Because the two subsets are disjoint, we can concatenate these two descriptions to produce a description of $R_{\hat{A} \cup \hat{B}}$.

Hence, the length of the shortest description of the union satisfies:
$$
l \left( d_{\hat{A} \cup \hat{B}}^{\star} \right) \leq l \left( d_{\hat{A}}^{\star} \right) + l \left( d_{\hat{B} \backslash \hat{A}}^{\star} \right).
$$
To prove equality, assume there exists a shorter description $d'$ for $\hat{A} \cup \hat{B}$ such that:
$$
l(d') < l \left( d_{\hat{A}}^{\star} \right) + l \left( d_{\hat{B} \backslash \hat{A}}^{\star} \right).
$$
Then, one could extract from $d'$ both $R_{\hat{A}}$ and $R_{\hat{B} \backslash \hat{A}}$, which would imply that at least one of $d_{\hat{A}}^{\star}$ or $d_{\hat{B} \backslash \hat{A}}^{\star}$ is not minimal—contradicting the assumption that they are perfect descriptions.

Therefore:
$$
l \left( d_{\hat{A} \cup \hat{B}}^{\star} \right) = l \left( d_{\hat{A}}^{\star} \right) + l \left( d_{\hat{B} \backslash \hat{A}}^{\star} \right).
$$
\end{proof}


%
% Section: References
%

\section{References}

For more information about Russell's paradox, Cantor's theorem, and universal sets, refer, for example, to \cite{jech2013set}. This book also covers the Zermelo–Fraenkel system of axioms, including the Axiom of Choice. The idea of using a function to assign to each symbol and well-formed formula of a formal language a unique natural number (called a Gödel number) was introduced by Kurt Gödel in his proof of the incompleteness theorems \cite{godel1931formal}. A detailed discussion of the Berry paradox from the perspective of computability can be found in \cite{chaitin1995berry}. For a review of the problem of representation in Kolmogorov complexity, as well as a detailed account of the implications of Kolmogorov complexity being defined only up to an additive constant, see \cite{li2013introduction}. That oracle machines are not mechanical was originally stated by Turing when he introduced the concept of the oracle machine in \cite{turing1939systems}. For a comprehensive review of oracle Turing machines, refer to \cite{robivc2015foundations}.
