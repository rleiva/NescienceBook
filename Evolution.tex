%
% CHAPTER 13.- Evolution
%

\chapterimage{Philosophers.pdf}

\chapter{Evolution}
\label{chap:evolution}

In this chapter we are going to study if the theory of nescience can be applied to biological organisms. In particular, we will try to answer the question if \emph{mother nature} is also trying to minimize nescience, what that means, and which are the implications.

A direct application of the theory is to measure the nesciencie that different species have achieved by natural selection. By means of computing the redundancy of the corresponding DNAs we will see that the more developed species present a smaller nescience. Of course, that does not mean that their DNA are shorter, but that they contain a higher amount of information in less space.

A more interesting application of the theory of nescience to biology is achieved when we consider that a specie must be effectively computable, in nature terms. With this respect we have to take into account that it is not only required to study the complexity or redundancy of the DNA, but also that of the machine that given the DNA can produce a new organism. New results are achived when we analyze both elements together.

But the most interesting result is achieved when we consider that the evolution mechanism used by nature to select the best individuals, basically an optimization algorithm, has also to be taken into consideration in order to compute the nescience. In this way, evolution is not the algorithm used by nature to optimize species, but evolution itself is the result of a meta-optimization algorithm to reduce nescience, a universal machine that produces life.
