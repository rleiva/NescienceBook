%
% CHAPTER: Mismodel
%

\chapterimage{Alexander_cuts_the_Gordian_Knot.pdf} % Chapter heading image

\chapter{Mismodel}
\label{chap:Mismodel}

\begin{quote}
\begin{flushright}
\emph{Everything should be made as simple as possible,\\
but not simpler.}\\
Albert Einstein 
\end{flushright}
\end{quote}
\bigskip

{\color{red} TODO: Nice introduction}

Mismodel, being a non-computable quantity, has to be approximated in practice. We propose to approximate the concept of mismodel by splitting it into two complementary subconcepts: inaccuracy and surfeit. Inaccuracy compares the output of our description with the representation selected to encode the entity.

%
% Section: Mismodel
%

\section{Mismodel}

In Section \ref{sec:descriptions_models} we defined the concept of description, or model, of an entity as a computer program that, when executed, recreates one of the representations that encode that entity. More specifically, a description $d$ of a representation $r$ encoding an entity $e$ is a Turing machine that, when interpreted by a universal Turing machine $\delta$, prints out the string $r$. We also saw that the objective of science is to find from the shortest desription $d^\star_r$ from the collection of perfect descriptions $\mathcal{D}^\star_r$ that model the representation $r$. This shortest desription $d^\star_r$ might not be unique.

Since our knowledge about the entity $e$ under study is incomplete, and in particular our knowledge about the representation $r$ of $e$ is incomplete, we will be usually working with a description $d$ that is, hopefully, colose to one of the shortest descriptions $d^\star_r$, but not necessarily equal. We are interested in to quantify the error due to the use of a description $d$ that is not perfect.

We define the mismodel of a description for a representation as the normalized information distance between our description and all possible minimal descriptions of $r$.

\begin{definition} [Mismodel]
\label{def:mismodel}
Let $r \in \mathcal{B}^\ast$ be a representation, and $d \in \mathcal{D}$ be a description. We define the \emph{mismodel} of the description $d$ for the representation $r$, denoted by $\sigma(d, r)$, as:
\[
\sigma (d, r) = { \underset{ d^\star \in d^\star_r } \min} \frac{ \max\{ K \left( d \mid d^\star \right), K \left( d^\star \mid d \right) \} } { \max\{ K \left( d \right), K \left( d^\star \right) \} }
\]
\end{definition}

The description $d$ we are using not only it might not be one of the shortest possible models that print out $r$, but in fact, it might happen that what $d$ prints is not $r$.

We cannot use the more ...

The mismodel of a description is, conveniently, a number between $0$ and $1$, as next proposition shows.

\begin{proposition}
We have that $0 \leq \sigma(d, r) \leq 1$ for all representations $r \in \mathcal{B}^\ast$ and all the descriptions $d \in \mathcal{D}$.
\end{proposition}
\begin{proof}
Given that $0 \leq \frac{ \max\{ K(x \mid y), K(y \mid x) \} } { \max\{ K(x), K(y) \} } \leq 1$ for all $x, y \in \mathcal{B}^\ast$ according to Proposition \ref{prop:ncd_between_zero_and_one}.
\end{proof}

{\color{red} The above proposition holds for all possible descriptions $d$ and all possible representations $r$, even in the case that a description $d$ is not intended as a model for the representation $r$, in which case the inaccuracy would be close to one.}

Mismodel is equal to zero if, and only if, the description $d$ is one of the possible valid descriptions of the representation $r$.

\begin{proposition}
Let $d \in \mathcal{D}$ be a description for a representation $r \in \mathcal{B}^\ast$, we have that $\iota(d, r) = 0$ if, and only if, $d \in \mathcal{D}_r$.
\end{proposition}
\begin{proof}
If $d \in \mathcal{D}_r$ we have that $K \left( r \mid \delta(d) \right) = K \left( \delta(d) \mid r \right) = 0$ and that $\iota(d, r) = 0$. If $\iota(d, r) = 0$ we have that $\max\{ K \left( r \mid \delta(d) \right) = K \left( \delta(d) \mid r \right) \} = 0$, which implies that $K \left( r \mid \delta(d) \right) = K \left( \delta(d) \mid r \right) = 0$ and that $d \in \mathcal{D}_r$.
\end{proof}

%
% Section: Mismodel of Joint Representations
%

\subsection{Mismodel of Joint Representations}

{\color{red}

Given two representations $r$ and $s$, we want to know the inaccuracy of the model $d$ when describing the joint representation $rs$. Since we require that $rs$ must be a valid representation, the formalization of the concept of inaccuracy applied to joint representation is straightforward, and it does not require a new definition:
\[
\iota(d, rs) = \frac{ \max\{ K \left(rs \mid \delta(d) \right), K \left( \delta(d) \mid rs \right) \} } { \max\{ K(rs), K \left(\Gamma(d) \right) \} }
\]
As a direct consequence of Proposition \ref{prop:range_miscoding}, if $r, s \in \mathcal{B}^\ast$ are two arbitrary representations and $d \in \mathcal{D}$ is a description, we have that $0 \leq \iota(d, rs) \leq 1$.

{\color{red} TODO: Recall the properties of NID, and particularize for the case of inaccuracy of joint representations.}

}

%
% Section: Mismodel of Conditional Descriptions
%

\subsection{Conditional Mismodel}

{\color{red}

In this section we are going to extend the concept of inaccuracy from descriptions to conditional descriptions, that is, the inaccuracy of a description assuming the existence of some background knowledge, what we call conditional inaccuracy.

We have to start by defining what we mean when we say that a conditional description is inaccurate.

\begin{definition}
Let $r \in \mathcal{B}^\ast$ be a representation, and $d_{r \mid s} = \langle d, s \rangle$ a conditional description of $r$ given the string $s \in \mathcal{B}^\ast$, with $ d = \langle TM, a \rangle$. If $TM \left(\langle a, s \rangle \right) = r'$, such that $r \neq r'$, we say that $d_{r \mid s}$ is an \emph{inaccurate conditional description} for $r$.
\end{definition}

In the same way we defined the concept of inaccuracy of a description in Definition \ref{def:inaccuracy:inaccuracy:inaccuracy}, we can define the concept of conditional inaccuracy to characterize the error made when using an inaccurate conditional description.

\begin{definition}
Let $r \in \mathcal{B}^\ast$ be a representation, $s \in \mathcal{B}^\ast$ a string, and $d_{r \mid s} = \langle d, s \rangle$ a inaccurate condtional description. We define the \emph{conditional inaccuracy} of the description $d_{r \mid s}$ for the representation $r$ given the string $s$, denoted by $\iota(d_{r \mid s})$, as:
\[
\iota(d_{r \mid s}) = \frac{ \max\{ K \left(r \mid \delta(\langle d, s \rangle) \right), K \left( \delta(\langle d, s \rangle) \mid r \right) \} } { \max\{ K(r), K \left(\delta(\langle d, s \rangle) \right) \} }
\]
\end{definition}

The conditional inaccuracy of a description is a number between $0$ and $1$.

\begin{proposition}
\label{prop:mismodel:range_conditional_inaccuracy}
Let $r \in \mathcal{B}^\ast$ be a representation, $s \in \mathcal{B}^\ast$ a string, and $d_{r \mid s} = \langle d, s \rangle$ a inaccurate condtional description. We have that $0 \leq \iota(d_{r \mid s}) \leq 1$.
\end{proposition}
\begin{proof}
Given that $0 \leq \frac{ \max\{ K(x \mid y), K(y \mid x) \} } { \max\{ K(x), K(y) \} } \leq 1$ for all $x, y \in \mathcal{B}^\ast$ according to Proposition \ref{prop:ncd_between_zero_and_one}.
\end{proof}

Conditional inaccuracy is equal to zero if, and only if, the description $d$ is one of the possible valid descriptions of the representation $r$.

\begin{proposition}\label{prop:perfect_description}
Let $r \in \mathcal{B}^\ast$ be a representation, $s \in \mathcal{B}^\ast$ a string, and $d_{r \mid s} = \langle d, s \rangle$ a condtional description, with $ d = \langle TM, a \rangle$. We have that $\iota(d_{r \mid s}) = 0$ if, and only if, $TM \left(\langle a, s \rangle \right) = r$.
\end{proposition}
\begin{proof}
If $TM \left(\langle a, s \rangle \right) = r$ we have that $K \left( r \mid \delta(d_{r \mid s}) \right) = K \left( \delta(d_{r \mid s}) \mid r \right) = 0$ and that $\iota(d_{r \mid s}) = 0$. If $\iota(d_{r \mid s}) = 0$ we have that $\max\{ K \left( r \mid \delta(d_{r \mid s}) \right) = K \left( \delta(d_{r \mid s}) \mid r \right) \} = 0$, which implies that $K \left( r \mid \delta(d_{r \mid s}) \right) = K \left( \delta(d_{r \mid s}) \mid r \right) = 0$ and that $TM \left(\langle a, s \rangle \right) = r$.
\end{proof}

Given two representations $r$ and $s$, we want to know the inaccuracy of a conditional model $d$ geniven $t$ when describing the joint representation $rs$. Since we require that $rs$ must be a valid representation, the formalization of the concept of contional inaccuracy applied to joint representation is straightforward, and it does not require a new definition:
\[
\iota(d_{rs \mid t}) = \frac{ \max\{ K \left(r \mid \delta(\langle d, t \rangle) \right), K \left( \delta(\langle d, t \rangle) \mid r \right) \} } { \max\{ K(rs), K \left(\delta(\langle d, t \rangle) \right) \} }
\]
As a direct consequence of Proposition \ref{prop:range_conditional_inaccuracy}, if $r, s \in \mathcal{B}^\ast$ are two arbitrary representations, $d_{r \mid s} = \langle d, s \rangle$ is a condtional description and $t \in \mathcal{B}^\ast$ an arbitrary string, we have that $0 \leq \iota(d_{rs \mid t}) \leq 1$.

}

%
% Section: Reducing Mismodel
%

\section{Reducing Mismodel}

{\color{red}

A valid representation of an entity is a string that contains all the necessary information required by the oracle to reconstruct that entity and only that information. If the representation is non-valid, meaning its miscoding exceeds zero, it may be because some crucial information is missing, some symbols are incorrect, or it contains irrelevant symbols. To reduce the miscoding of a representation, we can add the missing information, or remove the incorrect or irrelevant symbols. We cannot know in advance if any information is missing or some symbols need to be removed. Fortunately, as the next theorem illustrates, it must be one of these two cases.

\begin{theorem}
\label{th:reduce_mismodel}
Let $r \in \mathcal{B}^\ast$ be a representation such that $\mu(r) >0$, then at least one of the following cases is true:
\begin{enumerate}[label=(\roman*)]
\item there exist a $s \in \mathcal{B}^\ast$ such that $\mu(rs) < \mu(r)$ or $\mu(sr) < \mu(r)$,
\item there exists a $s \in \mathcal{B}^\ast$ in the form $r = \alpha s \beta$ with $\alpha, \beta \in \mathcal{B}^\ast$ such that $\mu(r) < \mu(s)$.
\end{enumerate}
\end{theorem}
\begin{proof}
{\color{red} Finish}
Assume that $\mu(r) >0$. We have that
\[
\overset{o}{ \underset{ r^\star_e \in \mathcal{R}^\star_\mathcal{E} } \min} \frac{ \max\{ K \left( r \mid r^\star_e \right), K \left( r^\star_e \mid r \right) \} } { \max\{ K \left( r \right), K \left( r^\star_e \right) \} } > 0
\]
Let $r^\star_e = arg\,min \left( \mu(r) \right)$. We have that $\max\{ K \left( r \mid r^\star_e \right), K \left( r^\star_e \mid r \right) \} > 0$. If $K \left( r \mid r^\star_e \right) > 0$ we have that $r$ contains non-relevant symbols. If $K \left( r^\star_e \mid r \right) > 0$ we have that $r$ is missing some relevant symbols.
\end{proof}

\begin{example}
{\color{red} TODO: Provide a practical example.}
\end{example}

}

%
% Section: Mismodel of Areas
%
\section{Mismodel of Areas}

{\color{red}

The concept of miscoding can be extended to research areas in order to quantitative measure the amount of effort required to fix an inaccurate representation of the area.

\begin{definition}
Let $A \subset \mathcal{E}$ be an area with known subset $\hat{A} = \{e_1, e_2, \ldots, e_n\}$. We define the \emph{miscoding of the area} given the known subset $\hat{A}$ as:
\[
\mu(\hat{A}) = \min_{(r^\star_{e_1}, r^\star_{e_2}, \ldots, r^\star_{e_n}) \in \mathcal{R}^\star_\mathcal{E}}  \frac{K \left( \langle t_{e_1}, t_{e_2}, \ldots, t_{e_n} \rangle \mid \langle t_1, t_2, \ldots, t_n \rangle \right) }{K \left( \langle t_{e_1}, t_{e_2}, \ldots, t_{e_n} \rangle \right)}
\]
\end{definition}

}

%
% Section: References
%

\section*{References}

A good introduction to the study of uncertaintines (error analysis in models) in science, and in particular in physics, chemistry, and engineering, is the best-selling text \cite{taylor2022introduction}, which also features the same image of a crashed train than in the introduction to this chapter.

The concept of redundancy has been also investigated in the context of information theory, since we are interested on using codes with low redundancy (see for example \cite{abramson1963information}).

{\color{red} TODO: Add more references.}
