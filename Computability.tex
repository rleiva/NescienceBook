%
% CHAPTER.- Computability
%

\chapterimage{TuringMachine.pdf}

\chapter{Computability}
\label{chap:Computability}

\begin{quote}
\begin{flushright}
\emph{Wanderer, there is no road,\\
the road is made by walking.}\\
Antonio Machado
\end{flushright}
\end{quote}
\bigskip

We continue our review of the background required to understand the theory of nescience by providing a mathematical formalization of the concept of a \emph{computable procedure}\index{Computable procedure}. Intuitively, a computable procedure is a method consisting of a finite sequence of instructions that, when applied to a problem, produces the correct answer after a finite number of steps. This informal notion rests on the requirement that the instructions be clear and precise enough for any human to follow without additional guidance. We may strengthen this requirement further by demanding that the instructions be so unambiguous that a machine could execute them.

In 1936, the British mathematician Alan Turing introduced a formal model of a family of hypothetical machines and argued that, for every intuitively computable procedure, there exists a \emph{Turing machine}\index{Turing machine} capable of carrying it out. The model is not only simple enough to allow precise mathematical analysis but also sufficiently general to capture the intuitive concept of effective computation.

Over the years, many alternative approaches have been proposed to formalize the notion of computability. Some of these are technically intricate, yet all have been shown to be equivalent in expressive power to Turing machines; in other words, they characterize the same class of effectively computable functions. Two notable examples are Alonzo Church's \emph{lambda calculus}\index{Lambda calculus} and the \emph{theory of recursive functions}\index{Recursive function} developed by Kurt Gödel and Stephen Kleene. The \emph{Church-Turing thesis}\index{Church-Turing thesis} asserts that any reasonable formalization of a computable procedure (subject to minimal requirements, such as performing only a finite amount of work in a single step) coincides with the Turing machine model. While not a theorem, this thesis has become a widely accepted working principle, providing a stable and shared notion of computability that is independent of any specific formalism.

The concept of the Turing machine, originally conceived as a model of a mechanical device designed to solve a particular problem, was later extended and generalized. A \emph{universal Turing machine}\index{Universal Turing machine} can simulate the behavior of any other Turing machine and thereby compute any function that is computable in the intuitive sense. This is analogous to modern digital computers, which can execute algorithms expressed in diverse programming languages. This universality raises a natural question: Are there problems that no Turing machine can solve? The answer is yes. Certain well-defined problems, such as the Halting problem\index{Halting problem}, lie beyond the reach of computation, and such problems are more common than one might initially expect. The existence of uncomputable functions will play a central role in our theory of nescience.

Given the abstract character of many entities studied in science, we employ the concept of the \emph{oracle Turing machine}\index{Oracle Turing machine} to formalize our framework. An oracle Turing machine resembles an ordinary Turing machine but is augmented with the ability to query an external oracle. The oracle, whose internal mechanism is unspecified, provides answers to questions that may be uncomputable for standard machines. Different oracles yield different computational powers, giving rise to a hierarchy of relative computability. For our purposes, the oracle can be viewed as a theoretical construct that models access to sources of information beyond the limits of mechanical computation. It should be emphasized that the oracle is not a physical device but an abstract tool for reasoning about the boundaries of computation.

Turing machines reveal the intrinsic limitations of computation. Exploring these limitations is not merely a philosophical exercise; it has profound implications for the field of \emph{computational complexity}\index{Computational complexity}. Situated at the intersection of computer science and mathematics, computational complexity studies the resources (most notably time and space) required to solve problems. Problems are classified into \emph{complexity classes}\index{Complexity classes} according to the asymptotic resources needed by the best algorithms known for their solution. One of the most famous open problems in this field is the $P\overset{?}{=}NP$\index{$P\overset{?}{=}NP$} question, which asks whether the class $P$ of problems solvable in polynomial time coincides with the class $NP$ of problems for which proposed solutions can be verified in polynomial time. In this book, our interest extends beyond the epistemological issue of determining which problems can be solved in principle, given unlimited resources, to the more practical question of which problems can be solved efficiently with respect to time.

%
% Section: Turing Machines
%

\section{Turing Machines}
\label{sec:Turing-Machines}

A Turing machine is an extremely simplified model of a general-purpose computer, yet it is capable of solving any problem that real computers can address. Intuitively, one can envision the machine as consisting of a head that operates on a two-way infinite tape striped with symbols. At each time step, the machine reads the symbol under the head and decides to either write a new symbol on the tape, move the head one square to the left or right, or execute both actions. Algorithms are implemented using an internal table of rules housed within the control head, and the actual input to the algorithm is encoded on the tape. Once the machine reaches its final state, the algorithm's output can be read from the tape. Figure \ref{fig:Turing-Machine} depicts an example of a machine in its initial state, with the head located at the beginning of the input string.

\begin{figure}[t]
\centering
\begin{tikzpicture}[>=stealth', shorten >=1pt, auto, node distance=2cm]
    % Draw tape
    % \draw[thick] (-3,0) -- (3,0);
    \foreach \x in {-2.5,-1.5,...,2.5} {
        \draw (\x,-0.5) rectangle ++(1,1);
    }
    
    % Draw tape head
    \node[draw, rectangle, minimum height=1cm, minimum width=1cm, thick] (head) at (0.5,1.5) {Head};
    
    % Draw symbols on the tape
    \node at (-2,0) {1};
    \node at (-1,0) {0};
    \node at (0,0) {1};
    \node at (1,0) {0};
    \node at (2,0) {1};
    
    % Draw arrow from head to tape
    \draw[->] (head) -- (1,0.5);
\end{tikzpicture}
\caption{\label{fig:Turing-Machine}Turing Machine}
\end{figure}

The following definition formally introduces the concept of a Turing machine.

\begin{definition}[Turing Machine]
\label{def:Turing-Machine}\index{Turing machine}
A \emph{Turing machine} is a 7-tuple $\left(Q,\Gamma,\sqcup,\Sigma,q_{i},q_{f},\tau\right)$ where:
\begin{align*}
 & Q \quad \text{is a finite, non-empty, set of \emph{states},} \index{Machine State} \\
 & \Gamma \quad \text{is a finite, non-empty, set of \emph{tape symbols},} \index{Tape Symbol} \\
 & \sqcup\in\Gamma \quad \text{is the \emph{blank symbol},} \index{Blank Symbol} \\
 & \Sigma\subseteq\Gamma\setminus\sqcup \quad \text{is the set of \emph{input symbols},}  \index{Input Symbol} \\
 & q_{o}\in Q \quad \text{is the \emph{initial state},} \index{Initial State} \\
 & q_{f}\in Q \setminus \{q_o\} \quad \text{is the \emph{final state},} \index{Final State} \\ 
 & \tau:\left(Q\setminus \{q_{f}\}\right)\times\Gamma \rightarrow  Q\times\Gamma\times\left\{ L,R,S\right\} \quad\text{is a partial \emph{transition function}. \index{Transition Function} }
\end{align*}
\end{definition}

The algorithm executed by the machine is defined by the transition function $\tau$. This function dictates the machine's actions based on its current state and the tape symbol currently under the head. According to $\tau$, the machine transitions to a new state, writes a new symbol on the tape (or retains the existing one), and moves the head left, right, or keeps it stationary ($L$, $R$, or $S$ respectively). The machine follows a finite, uniquely determined sequence of steps until it reaches the final state $q_f$ and \emph{halts}, making no subsequent moves. The algorithm's output is the string of symbols $s \in \Sigma^\ast$ remaining on the tape after halting. Some machines, however, may enter an infinite loop, never reaching a halting state. If a machine encounters an undefined transition, it will enter in an infinite loop, never halting.

The machine's input consists of a string of symbols, with the assumption that the machine's head is initially positioned at the first symbol of the input string. To address problems involving an object $O$ that isn’t a string, we must first develop a method to encode that object as a string, denoted as $\left\langle O \right\rangle$.

\begin{example}
\label{ex:Turing-Machine}
The following Turing machine is designed to solve the problem of adding two natural numbers. It consists of the set of states $Q = \left\{q_{o}, q_{1}, q_{f}\right\}$, the set of tape symbols $\Gamma = \left\{0, 1, \sqcup \right\}$, and the set of input symbols $\mathcal{B} = \left\{0, 1 \right\}$. The transition function is defined in the table below, where rows are indexed by machine states, and columns by tape symbols:

\begin{table}[h]
\centering
\begin{tabular}{l l l l}
\toprule
 & 0 & 1 & $\sqcup$ \\
\midrule
$q_{o}$ & $\left(q_{f}, \sqcup, S\right)$ & $\left(q_{1}, \sqcup, R\right)$ & $\uparrow$ \\
$q_{1}$ & $\left(q_{f}, 1     , S\right)$ & $\left(q_{f}, 1     , R\right)$ & $\uparrow$ \\
\bottomrule
\end{tabular}
\caption{Transition Rules}
\end{table}

For natural numbers $n$ and $m$, the input string is composed of $n$ occurrences of the symbol ‘1’, followed by a ‘0’, and then followed by $m$ occurrences of ‘1’. The machine's output will be a string of $n+m$ consecutive ‘1’s. For instance, to add the numbers 2 and 3, the input string should be $\sqcup 1 1 0 1 1 1 \sqcup$, resulting in the output string $\sqcup 1 1 1 1 1 \sqcup$.
\end{example}

A Turing machine can also be represented by a \emph{state diagram}\index{State diagram}. A state diagram is similar to a labeled directed graph\footnote{In this particular case, we allow loops and multiple edges originating from vertices.} where the vertices represent the states of the machine. The edges signify transitions from one state to another, and the edge labels indicate the symbol under the head that leads to the new state, the symbol that gets written on the tape, and the direction in which the head moves. Following these conventions, the state diagram for the Turing machine in Example \ref{ex:Turing-Machine} is depicted in Figure \ref{fig:Example-Turing-Machine}.

\begin{figure}[t]
\centering
\begin{tikzpicture}[shorten >=1pt,node distance=3cm,on grid,auto] 
   \node[state,initial] (q_0)   {$q_0$}; 
   \node[state] (q_1) [right=of q_0] {$q_1$}; 
   \node[state] (q_f) [right=of q_1] {$q_f$}; 
   
   \path[->] 
    (q_0) edge  node {1$\to$$\sqcup$,R} (q_1)
    (q_0) edge[bend right] node[below] {0$\to$$\sqcup$,R} (q_f)
    (q_1) edge  node {0$\to$1,S} (q_f)
    (q_1) edge [loop above] node {1$\to$1,R} ();
\end{tikzpicture}
\caption{\label{fig:Example-Turing-Machine}Example of Turing Machine}
\end{figure}

It is a remarkable fact that minor alterations to the definition of a Turing machine do not change its computational power. In other words, the definition is highly robust. In Example \ref{ex:multitape_turing_machine}, it's demonstrated that adding more tapes to the machine doesn't expand the range of problems it can solve. Similar arguments can be made when adding finite storage to the control tape, allowing for parallel processing with multiple control heads, and so on.

\begin{example}
\label{ex:multitape_turing_machine}
A \emph{multitape Turing machine}\index{Multitape Turing machine} is a Turing machine equipped with multiple heads and their respective tapes. In the initial configuration, the input string resides in tape 1, while the other tapes are blank. The transition function for a multitape Turing machine is:
\[
\tau:\left(Q \setminus q_{f} \right) \times \Gamma^k \rightarrow  Q \times \Gamma^k \times \left\{L,R,S\right\}^k,
\]
where $k$ denotes the number of tapes. Multitape Turing machines are equivalent in power to standard Turing machines. We can validate this claim by devising a method for a standard Turing machine to mimic a multitape machine's behavior. This requires encoding the content of multiple tapes onto a single tape, introducing a new symbol as a tape separator, and encoding the positions of the heads across the tapes with a distinct head location symbol. If we designate the tape separation symbol as $|$ and the head location symbol as $h$, a simulation tape for a machine with 3 tapes might appear as $\sqcup01h00|000h1|h0101\sqcup$. The standard machine's operation would involve scanning the subtapes one by one, pinpointing the head's location, and executing the necessary transition. If the computation on one subtape necessitates writing a new symbol beyond its boundary, we'd need to shift subsequent symbols to accommodate the new one. While the simulation might operate at a slower pace than the original multitape machine, both machine types can solve an identical set of problems.
\end{example}

For the remainder of this book, without any loss of generality, we'll assume that the set of input symbols is $\Sigma = \mathcal{B}$ and the set of tape symbols is $\Gamma = \left\{0, 1, \sqcup \right\}$.

In addition to providing a formal definition of a Turing machine, it's essential to formally outline its computational process. This entails detailing how the machine reads the input string, produces the output string, and transitions between states during computation. We will start by defining the concept of the machine's internal configuration. This configuration captures the machine's current state and position, as well as the present state of the tape.

\begin{definition}\index{Configuration}
A \emph{configuration}\index{Configuration} of a Turing machine $T$ is the 3-tuple $\left(q,s,i\right)$, where $q\in Q$ represents a state of the machine, $s\in\Gamma^+$ denotes a string containing the tape's content (excluding the blank symbols), and $1 \le i \le n$ is the index of the symbol $s_i$ beneath the head. Here, $s_1$ is the first non-blank symbol on the tape, and $n=l(s)$. 
\end{definition}

Configurations enable us to describe the current state of a Turing machine without any loss of information. At any stage of computation, one could halt the machine, record its configuration, and later resume the computation from the exact point of interruption using this configuration.

The following definition explains how we transition from one configuration to the next during computation.

\begin{definition}\index{Configuration yields configuration}
A configuration $C=\left(q,s,i\right)$ \emph{yields}\index{Yield} another configuration $C'=\left(r,s',j\right)$ if there exists a transition $\tau:\left(q, s_{i}\right) = \left(r, s'_{i}, a\right)$, where $s=s_{1} \dots s_{i-1}s_{i}s_{i+1} \dots s_{n}$, $s'=s_{1} \dots s_{i-1}s'_{i}s_{i+1} \dots s_{n}$, and
\begin{equation}
  j = \begin{cases}
        i+1 & \text{if $a=R$} \\
        i-1 & \text{if $a=L$} \\
        i   & \text{if $a=S$}
  \end{cases}
\end{equation}
\end{definition}

Building on the concepts of configuration and one configuration yielding another, we can now formally articulate the notion of computation.

\begin{definition}[Computation]\index{Computation}
Let $T$ be a Turing machine, $C_{0}$ its initial configuration, and $C_n$ a configuration encompassing the final state $q_f$. A \emph{computation}\index{Computation} under machine $T$ refers to a finite sequence of $n+1$ configurations $\left(C_{0},C_{1},\ldots,C_n\right)$ wherein each configuration $C_{k}$ yields the subsequent configuration $C_{k+1}$, for all $0\leq k < n$.
\end{definition}

Computations are deterministic; meaning, for a given Turing machine $T$ and an input string $s$, the configuration sequence is preordained. If machine $T$ neither halts nor progresses with input $s$, we deduce the absence of computation.

\begin{example}
The computation of the Turing machine described in Example \ref{ex:Turing-Machine} using the input string $110111$ results in the following sequence of configurations:

\begin{enumerate}
\item $(q_0, 110111, 1)$
\item $(q_1, 10111,  1)$
\item $(q_1, 10111,  2)$  
\item $(q_f, 11111,  2)$  
\end{enumerate}

\end{example}

Intuitively, a procedure is deemed computable by a human if it can be delineated through specific steps, executed systematically, without relying on intuition or ingenuity. This intuitive grasp aligns with the formalized concept of a Turing machine, bridging informal comprehension and the machine's rigorous definition—a cornerstone in the theory of computation. However, this alignment presents an intriguing challenge. Affirming that our grasp of computability mirrors a Turing machine's capabilities cannot be proven traditionally, as 'computability' lacks a well-defined interpretation. Consequently, some researchers categorize this as a \emph{thesis}, avoiding the formal 'theorem' label. Turing himself opted to term it a \emph{definition}, steering clear of denoting it as a theorem.

\begin{theorem}[Turing's Thesis]
\label{th:turing_thesis}\index{Turing's thesis}
A procedure is computable if, and only if, it can be executed by a Turing machine.
\end{theorem}

To further underscore the significance and robustness of the Turing machine as a model of computation, it's worth noting, as mentioned earlier in this chapter, that all alternative formalizations of computability proposed to date align in terms of their computational capabilities with that of the Turing machine. This universality underscores the Turing machine's central position in the realm of theoretical computer science.

%
% Section: Universal Turing Machine
%

\section{Universal Turing Machines}
\label{sec:Universal-Turing-Machines}

In Section \ref{sec:Turing-Machines}, we explored storing the current state of a Turing machine, its configuration, to pause and later resume computation. In Example \ref{ex:Encoding_TM}, we will delve into a similar procedure, not for storing the machine's current state, but for saving a comprehensive description of the machine itself. This methodology facilitates the enumeration, or listing, of all possible Turing machines. Such enumeration is instrumental in demonstrating the existence of problems that cannot be solved by any Turing machine (refer to Section \ref{sec:non_computable_problems}) and unveiling the pivotal concept of the \emph{Universal Turing Machine}.

\begin{example}
\label{ex:Encoding_TM}
To describe a Turing machine concisely, we need to encode the transition function $\tau:\left(Q\setminus \{q_{f}\}\right)\times\Gamma\rightarrow Q\times\Gamma\times\left\{ L,R,S\right\}$. This function can be represented as a collection of quintuples $\left(q,s,r,t,a\right)$, where $q \in \left(Q\setminus \{q_{f}\}\right)$, $r \in Q$, $s, t\in\Gamma$, and $a\in\left\{ L,R,S\right\}$. In this manner, any Turing machine $T$ is fully described by a collection of quintuples:
\[
\left(q_{1},s_{1},r_{1},t_{1},a_{1}\right),\left(q_{2},s_{2},r_{2},t_{2},a_{2}\right),\ldots,\left(q_{m},s_
{m},r_{m},t_{m},a_{m}\right)
\]
where $m \leq d\left(Q\setminus \{q_{f}\}\right) \times d(\Gamma)$, with the stipulation that the first quintuple refers to the initial state and the second one to the final state; i.e., $q_{1} = q_{o}$ and $r_{2} = q_{f}$. A possible approach to describe these quintuples is to encode the elements of the set $Q\cup\Gamma\cup\left\{ L, R, S \right\}$ using a fixed-length binary code (refer to Definition \ref{def:Fixed-Length-Codes} for more details), encoding the quintuple $\left(q,s,r,t,a\right)$ as $\left\langle q, s, r, t, a \right\rangle$. The length of an encoded quintuple is $5l$, where $l=\left\lceil \log\left(d\left(Q\cup\Gamma\cup\left\{ L,R,S\right\}\right)\right)\right\rceil$. Following this convention, machine $T$ is encoded as the binary string:
\[
\left\langle T \right\rangle = \left\langle \bar{l}, \left\langle q_{1}, r_{1}, s_{1}, t_{1}, a_{1} \right\rangle, \ldots, \left\langle q_{r}, r_{r}, s_{r}, t_{r}, a_{r} \right\rangle \right\rangle 
\]
The length of the encoded machine, following this schema, would be $l(\left\langle T \right\rangle) \leq 5lm + \log l + 1$.
\end{example}

Since each Turing machine is composed by a finite set of quintuples, we can encode and list all the machines using a shortlex ordering. We associate each machine $T$ with the index $i$ corresponding to its position in this list, and we denote by $T_i$ the i-th Turing machine. Each positive integer $i$ encodes one, and only one, Turing machine. However, as Proposition \ref{prop:padding_lemman} shows, all Turing machines have an infinite number of indexes. We associate each Turing machine with its smallest index.

\begin{proposition}[Padding Lemma]
\label{prop:padding_lemman}\index{Padding lemma}
Each Turing machine has infinitely many indexes.
\end{proposition}
\begin{proof}
Consider a Turing machine $T_i$ encoded by the string $\langle T_i \rangle$. We can create a new encoding $\langle T_j \rangle$  by appending a finite number of 0's to $\langle T_i \rangle$,  such that $\langle T_j \rangle = \langle T_i \rangle 0^n$  for some positive integer $n$. Since $n$ can take on any positive integer value, there are infinitely many possible encodings $\langle T_j \rangle$ for the same Turing machine $T_i$.
\end{proof}

A universal Turing machine is a machine that can simulate the behavior of any other Turing machine on arbitrary input. The universal machine achieves this by reading both the description of the machine to be simulated (for instance, using the coding schema described in Example \ref{ex:Encoding_TM}) and the input string for the computation from its own tape.

\begin{definition}[Universal Turing Machine]
\label{def:Universal-Turing-Machine}
\index{Universal Turing machine}
A \emph{Universal Turing Machine} is a Turing machine $U$ such that $U(\langle \langle T_i\rangle, s \rangle) = T_i(s)$ for all Turing machines $T_i$ and all input strings $s \in \mathcal{B}$.
\end{definition}

Naturally, we must prove that such a machine exists before we can utilize it. One could argue that a human being could decode the machine $T_i$ and simulate its behavior with the input string $s$, and then refer to Theorem \ref{th:turing_thesis}. A more rigorous approach would be to explicitly construct a universal Turing machine. However, providing a detailed description of one of these machines is beyond the scope of this book. Instead, we direct the reader to the references included at the end of the chapter for further exploration.

%
% Section: Non computable problems
%

\section{Non-Computable Problems}
\label{sec:non_computable_problems}

Turing machines enable us to delineate the set of problems that can be resolved through effective procedures or, in other words, by computers. It may be surprising to learn that numerous problems cannot be addressed using algorithms; such challenges lie beyond the computational capabilities of machines. We are not alluding to speculative queries like whether a computer can be intelligent or self-aware but to concrete, well-defined mathematical problems. We are also not referring to complex problems that demand a substantial amount of time to solve, as those, irrespective of their time consumption, remain computable.

One classic exemplar of non-computability is the \emph{halting problem}\index{Halting problem}. As illustrated in Algorithm \ref{alg:halt}, it involves a program or algorithm tasked with determining whether any given program (including itself) and input will eventually halt or continue to run indefinitely. Alan Turing proved that no algorithm can exist to solve this problem for all possible program-input pairs. This revelation wasn't a reflection on the limitations of technology or processing power but highlighted a profound theoretical limit intrinsic to computation.

\begin{algorithm}
\caption{HALT function}
\label{alg:halt}
\begin{algorithmic}
\Procedure{HALT}{$A, I$}
    \If{$A(I)$ halts} 
        \State \textbf{return} $1$
    \Else
        \State \textbf{return} $0$
    \EndIf
\EndProcedure
\end{algorithmic}
\end{algorithm}

The proposition below proves that the halting problem is non-computable.

\begin{proposition}[Halting Problem]
\index{Halting problem}
\label{th:halting-problem}
Define HALT as in Algorithm \ref{alg:halt}. There does not exist a Turing machine that computes the HALT function for all possible pairs $(A, I)$, where $A$ is a Turing machine and $I$ is the input string to that machine.
\end{proposition}
\begin{proof}
The proof is by contradiction. Assume that the machine $HALT$  exists, and define a new Turing machine $TC$ such that $TC(A) = 1$ if $HALT(A,A) = 0$, and $TC(A)$ will never stop if $HALT(A,A) = 1$. Then the contradiction arises when we ask about the result of $TC(TC)$: if $TC(TC)$ stops we have that $HALT(TC,TC) = 0$ and that $TC(TC)$ should not stop, and if $TC(TC)$ does not stop then we have that $H(TC,TC) = 1$ and thus $TC(TC)$ should stop.
\end{proof}

The existence of such non-computable problems underscores the boundaries of mechanical computation. It illustrates that while Turing machines, and by extension, computers are profoundly powerful tools capable of solving an extensive array of problems, they are not omnipotent. A frontier of unsolvable problems exists, necessitating deeper exploration into the realms of mathematics, logic, and perhaps even philosophy to understand the inherent limits of computation.

The Halting Problem also has significant practical consequences in computer programming. For instance, it is impossible to write a program that can guarantee any other arbitrary program is bug-free or that all infinite loops with conditional exits will eventually halt for all possible inputs.

The next example introduces a well-defined, practical problem involving simple string manipulation that cannot be solved using computers.

\begin{example}
\label{ex:PCP}
Given two finite lists $\left( \alpha_1, \ldots, \alpha_n \right)$ and $\left( \beta_1, \ldots, \beta_n \right)$ of strings over some alphabet $\Sigma$, where $d(\Sigma) \ge 2$, the \emph{Post Correspondence Problem}\index{Post Correspondence Problem} (PCP) asks to determine if there exists a sequence of $K \geq 1$ indices $(i_k)$, with $1 \le i_k \le n$ for all $1 \le k \le K$, such that $\alpha_{i_1} \ldots \alpha_{i_K} = \beta_{i_1} \ldots \beta_{i_K}$. For instance, given the sequences $(a, ab, bba)$ and $(baa, aa, bb)$, a solution would be $\alpha_3 \alpha_2 \alpha_3 \alpha_1 = \beta_{3} \beta_{2} \beta_{3} \beta_{1}$. No algorithm exists to solve PCP. Like many proofs of incomputability, the proof proceeds by showing that HALT can be reduced to PCP, meaning if PCP is decidable, then the Halting Problem should be decidable as well. We will not detail the proof in this section; for interested readers, we refer to the references at the end of this chapter.
\end{example}

Non-computable problems are generally not derived directly from natural phenomena but from logical and mathematical constructs. To date, there are no known examples of non-computable problems manifesting plainly in natural phenomena. It's essential to distinguish between non-computability and unpredictability. Non-computable problems are those for which no algorithm can ever be created to solve them. In contrast, unpredictable systems (such as chaotic or complex systems) are theoretically computable but are unpredictable in practice due to factors like sensitivity to initial conditions or measurement precision.

%
% Section: Computable Functions and Sets
%

\section{Computable Functions and Sets}
\label{sec:computable_functions}

Each Turing machine \(T\) defines a function \(f_T:\mathcal{B}^{\ast} \rightarrow \mathcal{B}^{\ast}\) that assigns to each input string \(s \in \mathcal{B}^{\ast}\) an output string \(T(s) \in \mathcal{B}^{\ast}\). The relationship between Turing machines and functions forms the basis for introducing the concept of a \emph{computable function}.

\begin{definition}
\label{def:computable-function}
\index{Computable function}
A function \(f: \mathcal{B}^{\ast} \rightarrow \mathcal{B}^{\ast}\) is \emph{computable}\index{Computable function} if there exists a Turing machine \(T\) that defines the function \(f\) and halts for all the values of \( \mathcal{B}^{\ast} \).
\end{definition}

The terminology in computational theory can vary. While computable functions are occasionally referred to as \emph{recursive functions}\index{Recursive function}, this book opts for the term computable functions for consistency.

\begin{example}
The function that assigns to each pair of natural numbers \(x\) and \(y\) their sum \(x + y\) is computable, as demonstrated in Example \ref{ex:Turing-Machine}. 
\end{example}

In real-world scenarios, certain functions don’t provide a defined output for all possible inputs. Partial computable functions, characterized by Turing machines that don’t halt for specific inputs, model these cases.

\begin{definition}
A partial function \(f:\mathcal{B}^{\ast} \rightarrow \mathcal{B}^{\ast}\) is \emph{partial computable}\index{Partial computable function} if there exists a Turing machine \(T\) that defines \(f\) for defined values and does not halt for undefined values.
\end{definition}

The distinction between total computable functions and partial computable functions is significant in computability theory because it reflects the difference between problems that are always solvable by an algorithm (total) and those that are only solvable in some cases (partial).

\begin{example}
The function $f: \mathbb{N} \times \mathbb{N} \rightarrow \mathbb{N}$ that assigns to each pair of natural numbers $x$ and $y$ the number $x - y$ is a partial computable function, since it is not defined in the case that $x < y$. Recall that according to our definition of Turing machine (see Definition \ref{def:Turing-Machine}), when the machine reaches an undefined configuration enters an infinite loop without ever halting.
\end{example}

We can expand the application of the principles of computability and partial computability to the domain of sets. We characterize sets through the lens of theirs characteristic functions that discern the membership of elements within the sets.

\begin{definition}
A set $A \in \mathcal{B}^\ast$ is \emph{computable}\index{Computable set} if its characteristic function $\mathcal{X}_A$ is a total computable function. A set $A \in \mathcal{B}^\ast$ is \emph{computably enumerable}\index{Computably enumerable set} if its characteristic function $\mathcal{X}_A$ is a partial computable function, that is, $\mathcal{X}_A(a) = 1$ if $a \in A$, but $\mathcal{X}_A(a)$ is undefined if $a \not\in A$.
\end{definition}

The application of these concepts is illustrated through the example of the set of all Turing machines that halt for all inputs.

\begin{example}
The set of all Turing machines that halt on all inputs, as demonstrated in \ref{th:halting-problem}, is not computable but is computably enumerable.
\end{example}

Next proposition provides an alternative characterization of computable sets.

\begin{proposition}
A set $A \in \mathcal{B}^\ast$ is computable if and only if $A$ and its complement $A^c$ are computably enumerable.
\end{proposition}
\begin{proof}
If $A$ is computable, by definition, there exists a Turing machine that decides for any input $x \in \mathcal{B}^\ast$ whether $x \in A$ or $x \notin A$, halting in both cases. This implies that both $A$ and its complement $A^c$ can be enumerated by Turing machines. Thus, both $A$ and $A^c$ are computably enumerable.

Conversely, suppose $A$ and $A^c$ are computably enumerable. This means there exist two Turing machines, $T_A$ and $T_{A^c}$, that enumerate the elements of $A$ and $A^c$, respectively. To show that $A$ is computable, construct a Turing machine $T$ that, given an input $x \in \mathcal{B}^\ast$, simulates $T_A$ and $T_{A^c}$ in parallel to search for $x$. If $x$ appears in the enumeration produced by $T_A$, $T$ halts and accepts $x$ as an element of $A$. If $x$ appears in the enumeration produced by $T_{A^c}$, $T$ halts and accpts $x$, indicating $x \notin A$. Since every element of $\mathcal{B}^\ast$ must be in either $A$ or $A^c$ and both sets are computably enumerable, $T$ will eventually halt for every input $x$, proving that $A$ is computable.
\end{proof}


%
% Section: Oracle Turing Machine
%

\section{Oracle Turing Machine}
\label{sec:oracle_turing_machine}

An oracle Turing machine\index{Oracle Turing machine} (see Figure \ref{fig:Oracle-Turing-Machine}) is a theoretical model of computation that extends the capabilities of a standard Turing machine by providing it with an oracle. The oracle is a black box that can instantly compute certain answers, even for problems that are unsolvable or would take an impractical amount of time for a standard Turing machine to process. This model helps computer scientists and mathematicians explore the implications and boundaries of computational theory, including questions about complexity classes and the limits of what is computationally possible. The oracle Turing machine isn't a physical or implementable machine but rather a conceptual tool used in theoretical studies.

\begin{figure}[t]
\centering
\begin{tikzpicture}[>=stealth', shorten >=1pt, auto, node distance=2cm]
    % Draw tape
    \foreach \x in {-2.5,-1.5,...,2.5} {
        \draw (\x,-0.5) rectangle ++(1,1);
    }
    
    % Draw tape head
    \node[draw, rectangle, minimum height=1cm, minimum width=1cm, thick] (head) at (0.5,1.5) {Head};
    
    % Draw symbols on the tape
    \node at (-2,0) {1};
    \node at (-1,0) {0};
    \node at (0,0) {1};
    \node at (1,0) {0};
    \node at (2,0) {1};
    
    % Draw oracle
    \node[draw, rectangle, minimum height=1.5cm, minimum width=2cm, right of=head, node distance=3cm] (oracle) {Oracle};
    
    % Draw arrow from head to tape
    \draw[->] (head) -- (1,0.5);
    
    % Draw arrows to and from the oracle
    \draw[->] (head.east) to[out=0,in=180] node[midway, above] {Query} (oracle.west);
    \draw[->] (oracle.west) to[out=180,in=0] node[midway, below] {Answer} (head.east);
\end{tikzpicture}
\caption{\label{fig:Oracle-Turing-Machine}Oracle Turing Machine}
\end{figure}

\begin{definition}[Oracle Turing Machine]
\label{def:Oracle-Turing-Machine}
An \emph{oracle Turing machine} with oracle set $\mathcal{O}$ is a 8-tuple $\left(Q, \Gamma, \sqcup, \Sigma, q_i, q_f, \tau, \mathcal{O} \right)$ where:
\begin{align*}
 & Q \quad \text{is a finite, non-empty, set of \emph{states},} \index{Machine State} \\
 & \Gamma \quad \text{is a finite, non-empty, set of \emph{tape symbols},} \index{Tape Symbol} \\
 & \sqcup\in\Gamma \quad \text{is the \emph{blank symbol},} \index{Blank Symbol} \\
 & \Sigma\subseteq\Gamma\setminus\sqcup \quad \text{is the set of \emph{input symbols},}  \index{Input Symbol} \\
 & q_{o}\in Q \quad \text{is the \emph{initial state},} \index{Initial State} \\
 & q_{f}\in Q \setminus \{q_o\} \quad \text{is the \emph{final state},} \index{Final State} \\ 
 & \tau: \left(Q \setminus \{q_{f}\}\right) \times \Gamma \times \{0, 1\} \rightarrow  Q\times\Gamma\times\left\{ L,R,S\right\} \quad\text{is the \emph{transition function}, \index{Transition Function} } \\
 & \mathcal{O} \subseteq \Sigma^\ast \quad \text{is the \emph{oracle set}.}
\end{align*}
\end{definition}

Building on the concept of a regular Turing machine (refer to Definition \ref{def:Turing-Machine}), an oracle Turing machine introduces the unique feature of an oracle set $\mathcal{O}$. This set comprises a subset of strings for which the oracle can instantly provide answers. The true strength of an oracle machine emerges when the set $\mathcal{O}$ is non-computable; in cases where $\mathcal{O}$ is computable, a regular Turing machine would be sufficient.

The transition function $\tau$ for the oracle Turing machine is nuanced. At each step the machine will query the oracle. Specifically, it sends the current string $w$ —starting under the head and extending to the rightmost non-empty cell of the input tape—to the oracle. The oracle then responds with a binary answer: '1' if $w$ is in $\mathcal{O}$ or '0' if it isn't. The machine doesn't always utilize this response. There are steps where, despite receiving an answer, the computation proceeds unaffected by the oracle's response—essentially ignoring it. However, when the oracle's answer is pivotal, the machine makes a decision based on it. This decision could affect the next state, the next symbol to be written, or the next move (left, right, or stay). In essence, while the machine has the capability to continuously consult the oracle, it strategically chooses when to act on the information received. We could have modified the $\tau$ funcion in such a way that the oracle is queried only when the answer is relevant for the computation, but that would require making important changes to the behaviour of the function, such as the introduction of new control states.

\begin{example}
In the realm of theoretical computer science, an oracle can be invoked to "solve" the halting problem (refer to Theorem \ref{th:halting-problem}). The oracle is a hypothetical device or black box that, as if by magic, provides an instantaneous answer to a specific problem instance. The oracle set $\mathcal{O}$ would consist of a collection of strings in the form $\langle P, I \rangle$, where $P$ represents a program encoded as a string and $I$ denotes its input. Given a program and its input, this oracle would instantly inform us whether the program halts on that input. Naturally, this concept is purely theoretical. No such oracle exists in reality, and the halting problem remains unsolvable in practical terms.
\end{example}

{\color{red} Turing machines are a subset of oracle Turing machines}

\begin{proposition}
\end{proposition}

{\color{red} The machine is independent of the oracle set}

\begin{proposition}
\end{proposition}

{\color{red} An oracle machie is characterized by two independent components: a particular Turing machine and a particular oracle set.}

{\color{red} How to encode oracle Turing machines}

\begin{example}
\end{example}

{\color{red} What means a function or set to be oracle computable}

\begin{definition}
\end{definition}

%
% Section: Computational Complexity
%

\section{Computational Complexity}
\label{sec:computational_complexity}

\emph{Computational complexity}\index{Computational complexity} refers to the study of the efficiency of algorithms in terms of the resources they consume, such as time and space, to solve a given problem. By categorizing problems based on their inherent difficulty, computational complexity provides insights into the feasibility of solving problems within practical limits. It involves classifying problems into \emph{complexity classes}\index{Complexity class} offering a framework to analyze and compare the performance of algorithms. In this book we are insterested mostly in the time required to solve a problem. We compute the running time of an algorithm as a function of the length of the string representing the input.

\begin{definition}
Let $M$ a Turing machine. The running time or \emph{time complexity}\index{Time complexity} of $M$ is the function $T_M:\mathbb{N} \to \mathbb{N}$, where $T_M(n)$ is the maximum number of steps that machine $M$ takes for any input of length $n$.
\end{definition}

\emph{Big-O notation}\index{Big-O notation} is a mathematical notation that describes the upper bound of an algorithm's running time in the worst-case scenario, serving as a measure of its efficiency. Big-O notation provides a high-level analysis of an algorithm's performance, offering insights into how the running time requirements grow as the size of the input increases. It abstracts the details of the machine model and focuses on the most significant factors that contribute to the growth rate of the algorithm's complexity.

\begin{definition}
Let $f:\mathbb{N} \to \mathbb{N}$ and $g:\mathbb{N} \to \mathbb{N}$ be two functions. We say that the function $f$ is or order $g$, denoted \(f(n) = O(g(n))\), if there exist constants \(c > 0\) and \(n_0 \geq 0\) such that for all \(n \geq n_0\), \(f(n) \leq c \cdot g(n)\).
\end{definition}

Big-O notation is instrumental in comparing different algorithms and selecting the most appropriate one for a given problem and dataset size.

\begin{example}
Consider a polynomial function that represents the time complexity of an algorithm:
\[
f(n) = 3n^3 + 2n^2 + 5n + 7
\]
In the context of big-O notation, we are interested in the term with the highest growth rate as \(n\) approaches infinity, because it will eventually dominate the function. In this case, the term \(3n^3\) has the highest growth rate. Thus, we have
\[
f(n) = O(n^3)
\]
\end{example}

A complexity class consists of a collection of problems that are classified together due to their shared level of computational complexity. Each problem within a class shares similar characteristics of time complexity, ensuring a consistent measure of computational effort needed for resolution. These classes offer valuable insights and are integral in assessing and distinguishing the practical solvability and resource requirements of various computational challenges.

\begin{definition}\index{Complexity class}
A complexity class $C$ is a collection of problems for which there exists a time bound function $f(n)$ such that any problem $P$ in $C$ can be solved by a Turing machine $M$ with time compexity $T_M = f(n)$, where $n$ is the size of the input.
\end{definition}

The class $P$ consists of problems that can be solved in polynomial time by a Turing machine. In other words, for every problem in $P$, there exists an algorithm that can determine the solution in a number of steps that is a polynomial function of the size of the input. This class is foundational in computational complexity, serving as a baseline for measuring the efficiency of algorithms. 

\begin{definition}\index{Class P}
The class problems that can be solved in polynomial time, denoted by $P$, is defined as the collection of problems for which there exists a Turing machine $M$ that solve them and a polynomial $p(n)$ such that the time complexity of $M$ is $p(n)$.
\end{definition}

Problems within $P$ are considered tractable, meaning they can be practically solved even for large inputs. The concept of polynomial time solvability is crucial in distinguishing between problems that have efficient solutions in practice and those that do not.

\begin{example}
A classic example of a problem in class $P$ is the \emph{shortest path problem}\index{Shortest-path problem}. In this problem, you're given a weighted graph and two vertices, and the objective is to find the path of minimum total weight between these two vertices. One common algorithm to solve this problem is Dijkstra's algorithm. It works by iteratively selecting the vertex with the smallest known distance from the start vertex, and updating the estimated distances to its neighbors. This algorithm runs in polynomial time, specifically $O(V^2)$ for a graph with $V$ vertices. Since there exists a polynomial-time solution to this problem it is in class $P$.
\end{example}

A \emph{verifier}\index{Verifier} is a Turing machine that, given an input string and an additional string known as a \emph{certificate}\index{Certificate} or witness, determines whether the input satisfies a particular property, resulting in a binary "yes" or "no" answer. The core idea behind a verifier is not necessarily to find a solution to a problem, but rather to efficiently check or "verify" the validity of a proposed solution. 

\begin{definition}
A verifier is a Turing machine $M$ that, given an input string $x$ and an additional certificate string $y$, determines if $x$ belongs to a certain subset $S$ of strings. The machine satisfies two conditions:
\begin{enumerate}
    \item If $x$ belongs to subset $S$, there exists a certificate $y$ such that $V(x, y) = 1$.
    \item If $x$ does not belong to subset $S$, then for every possible certificate $y$, $V(x, y) = 0$.
\end{enumerate}
\end{definition}

Being the verifier a Turing machine, we can measure its time complexity, in terms of the length of the certificate. When a problem has solutions that can be verified in polynomial time using a verifier, it sheds light on the inherent complexity of that problem, situating it within specific computational classes.

\begin{definition}
A problem is in the class $NP$ if there exists a polynomial $p(n)$ and a Turing verifier $V$ such that for every instance $x$:
\begin{enumerate}
    \item If $x$ has a solution in subset $S$, then there exists a certificate $y$ with length at most $p(|x|)$ such that $V$ confirms $x$ as a valid instance of $S$ using $y$ in polynomial time.
    \item If $x$ does not have a solution in subset $S$, then for every possible certificate $y$ with length at most $p(|x|)$, $V$ does not confirm $x$ as a valid instance of $S$.
\end{enumerate}
\end{definition}

\begin{example}
{\color{red} TODO: Rewrite}
A classic example of a problem in class $NP$ is the \emph{Traveling Salesman Problem}\index{Traveling salesman problem} or TSP: given a list of cities and the distances between each pair of cities, the problem is to find the shortest possible route that visits each city exactly once and returns to the original city. To show that TSP is in $NP$, we don't need to demonstrate how to efficiently find the shortest route; we only need to show that, given a proposed route (or tour), we can efficiently verify whether that route satisfies the conditions of the problem and is shorter than or equal to a certain length. The verification process would be:
\begin{itemize}
\item Check if the proposed route visits each city exactly once and returns to the starting city. This can be done by traversing the proposed route and marking visited cities.
\item Compute the total distance of the proposed route by summing up the distances between consecutive cities in the route. 
\item Compare the computed distance to the given threshold or limit.
\end{itemize}
Given a proposed solution (the route) and the distances between cities, a verifier can perform the above steps in polynomial time with respect to the number of cities, thereby verifying the solution's validity. In the context of the $NP$ definition, the "certificate" for TSP would be the proposed route. If TSP has a route shorter than or equal to a specific distance, then there exists a certificate (the route itself) that can be verified in polynomial time. If not, no such certificate would make the verifier confirm the solution.
\end{example}

The $P=NP$ problem is one of the most fundamental questions in computer science. It asks whether every problem for which a potential solution can be verified quickly can also have its solution found quickly. In essence, if $P$ equals $NP$, it would mean that finding solutions is as "easy" as checking them. Despite extensive study, no one has been able to prove that $P=NP$ or $P\neq NP$, making it a longstanding open challenge in theoretical computer science.

\begin{theorem}[P=NP Problem]\index{P=NP Problem}
The class of problems $P$ is equal to the class of problems $NP$.
\end{theorem}

If $P$ were to equal $NP$, it would signify a monumental shift in our understanding of computational tasks. Practically, many problems considered intractable today, especially those in cryptography, would become tractable. Secure communication over the internet relies on certain problems being hard to solve (like factoring large numbers). If $P=NP$, then encryption methods underpinning online security could be broken, potentially leading to a crisis in digital security. Conversely, numerous challenges in fields like optimization, drug discovery, and logistics could see groundbreaking solutions, revolutionizing industries. While it's tempting to view $P=NP$ solely in the context of its challenges, especially to security, it would also herald unprecedented opportunities and advancements in multiple domains.

\emph{Turing reducibility}\index{Turing reducibility} is a fundamental concept in the realm of theoretical computer science and computability theory. It provides a means to compare the "computational difficulty" of problems. Specifically, a problem $A$ is Turing reducible to a problem $B$ if there exists an algorithm for solving $A$ that may make use of a subroutine solving $B$ an arbitrary number of times.

\begin{definition}
Let $A$ and $B$ be two problems. We say that $A$ is \textbf{Turing reducible} to $B$, denoted by $A \leq_T B$, if there exists a Turing machine that solves $A$ by making use of one or more Turing machines that solve $B$.
\end{definition}

If $A$ is Turing reducible to $B$ and vice versa, then the two problems are said to be Turing equivalent, indicating they have the same computational complexity.

\begin{example}
{\color{red} TODO: rewrite.}
Consider the problems of GRAPH ISOMORPHISM and SUBGRAPH ISOMORPHISM.

GRAPH ISOMORPHISM (GI): Given two graphs \(G_1\) and \(G_2\), is there a one-to-one correspondence (isomorphism) between their vertices such that the edges are preserved?

SUBGRAPH ISOMORPHISM (SI): Given two graphs \(G_1\) and \(G_2\), does \(G_1\) contain a subgraph that is isomorphic to \(G_2\)?

The GRAPH ISOMORPHISM problem is Turing reducible to the SUBGRAPH ISOMORPHISM problem. Here's the reasoning:

If we have an efficient solver for SUBGRAPH ISOMORPHISM, then we can use it to solve GRAPH ISOMORPHISM. For two graphs \(G_1\) and \(G_2\) of the same size, we ask the SI solver if \(G_2\) is a subgraph of \(G_1\) and vice versa. If both answers are affirmative, then \(G_1\) and \(G_2\) are isomorphic. Thus, an efficient solution for SI can be used to determine a solution for GI.

This example demonstrates the essence of Turing reducibility: by solving the more general problem (SUBGRAPH ISOMORPHISM), we inherently find a solution for the more specific problem (GRAPH ISOMORPHISM).
\end{example}

{\color{red} NP-hard}

{\color{red} NP-Complete}

%
% Section: References
%

\section*{References}

{\color{red} TODO:  Briefly mention the historical emergence of the concept, including the works of pioneers like Alan Turing, Alonzo Church, and others.}

The original paper form Alan Turing where the concepts of Turing machine, universal Turing machine, and non-computable problems were introduced is \cite{turing1936computable}, however it is a difficult to read paper for the contemporary reader. An easier to read introduction to computability theory, from the point of view of languages, can be found in \cite{sipser2012introduction}, and a more advanced introductions in \cite{cooper2003computability} and \cite{soare2016turing}. In \cite{fernandez2009models} we can find a description of the most important computability models proposed so far. The Post Correspondence Problem was introduced by Emil Post in \cite{post1946variant}; for the details of the proof sketched in Example \ref{ex:PCP} please refer to \cite{sipser2012introduction}.

{\color{red} TODO: Here are a few seminal academic works on non-computability:

Turing, A. M. (1936). On Computable Numbers, with an Application to the Entscheidungsproblem. Proceedings of the London Mathematical Society, s2-42(1), 230-265. Content: This foundational paper by Alan Turing introduces the concept of Turing machines, laying the groundwork for the theory of computation and establishing the halting problem's non-computability.

Church, A. (1936). An Unsolvable Problem of Elementary Number Theory. American Journal of Mathematics, 58(2), 345-363. Content: Alonzo Church presents the $\lambda$-calculus and establishes Church’s Thesis, claiming that his formalism captures the intuitive notion of "computable," and proving the unsolvability of the Entscheidungsproblem.

Gödel, K. (1931). Über formal unentscheidbare Sätze der Principia Mathematica und verwandter Systeme, I. Monatshefte für Mathematik, 38(1), 173-198. Content: Kurt Gödel's landmark paper where he introduces his incompleteness theorems, showing that within any sufficiently powerful mathematical system, there are statements that cannot be proven or disproven.

Post, E. (1946). A Variant of a Recursively Unsolvable Problem. Bulletin of the American Mathematical Society, 52(4), 264-268. Content: Emil Post presents a simpler, more accessible proof of unsolvability (non-computability) related to the halting problem and delves into the Post Correspondence Problem, a classic example of a non-computable problem.

Davis, M. (1958). Computability and Unsolvability. McGraw-Hill. Content: In this book, Martin Davis offers a comprehensive overview of the theory of computability and unsolvability, addressing topics like recursive functions, Turing machines, and Gödel’s incompleteness theorems in an accessible manner.

These references should provide a solid academic foundation for exploring the multifaceted and intriguing world of non-computability. Each offers unique insights and perspectives that collectively illuminate the complexity and depth of this area of study.
}

{\color{red} TODO: Add a reference to how to build a universal Turing machine.}

{\color{red} TODO: Introduce other well-known non-computable problems besides the halting problem, like Turing’s “Entscheidungsproblem” or the Busy Beaver function.}

{\color{red} TODO: Relate non-computability to Gödel's incompleteness theorems to illustrate the inherent limitations in formal mathematical systems.}

{\color{red} TODO: Explore philosophical discussions on the implications of non-computability for artificial intelligence and human cognition.}

