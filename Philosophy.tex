%
% CHAPTER - Philosophy
%

\chapterimage{Philosophers.pdf}

\chapter{Philosophy of Science}
\label{chap:Philosophy}

\begin{quote}
\begin{flushright}
\emph{To go where you don't know, \\
you have to go the way you don't know.} \\
San Juan de la Cruz 
\end{flushright}
\end{quote}
\bigskip

The \emph{philosophy of science}\index{Philosophy of science} is the branch of philosophy that examines the foundations, methods, and implications of scientific inquiry. It explores how scientific knowledge is generated, the validity of scientific theories, the nature of scientific reasoning, and the role of values in science. By addressing fundamental questions about objectivity, reality, and the limits of scientific explanation, the philosophy of science helps us understand how science works and its impact on our understanding of the world.

The philosophy of science provides us a theoretical framework for analyzing the core elements that constitute our theory of nescience. It enables us to critically examine the foundations and assumptions of our theory, guiding us toward the identification of the essential questions that we must address. Moreover, philosophical inquiry compels us to rigorously push the boundaries of our analysis, ensuring that our theory is explored to its ultimate consequences, both logically and conceptually. This approach strengthens the robustness of the theory, even if it these results do not directly translate into practical applications.

In the chapter, we provide a concise overview of key elements from the philosophy of science, as well as relevant concepts from other branches of philosophy, such as metaphysics, epistemology, and ontology, that are important to the theory of nescience. Certain other topics within the philosophy of science, such as the problem of objectivity (examining whether science can be truly objective or is influenced by social and personal values) or the role of values in science, including the relationship between ethical, social, and political values and scientific practices, are not included in this review, as they are not directly relevant to a mathematical theory of nescience.

The chapter begins with a brief introduction to the field and its importance in understanding scientific inquiry. We delve into the problem of which entities can be known, examining the scope of scientific knowledge and the nature of observable and unobservable phenomena. The chapter also addresses the concept of scientific representation, discussing how models, theories, and laws reflect aspects of reality. We examine how science discovers new knowledge, the principles behind the scientific method, and the various ways in which scientists formulate and test hypotheses. Finally, we explore the limits of science, considering the boundaries of what science can explain and where its explanatory power may fall short.

%
% Section: Metaphysics
%

\section{What is Science}

\emph{Science}\index{Science} is a systematic method of investigating the world around us, aimed at generating reliable knowledge through observation and reasoning. Unlike other forms of inquiry, science is rooted in the idea that knowledge must be based on evidence that can be tested and verified. By combining theoretical thinking with empirical data, science offers a powerful way to explain, predict, and understand natural phenomena. The following is a list of the key features that make science distinct and valuable as a way of knowing.

\begin{itemize}

\item \emph{Empiricism}\index{Empiricism}: At its core, science is an empirical endeavor, meaning it relies on observation, experimentation, and measurable evidence to understand the natural world. Scientific knowledge is grounded in data that can be gathered through direct or indirect observation, ensuring that claims can be tested and verified by others.

\item \emph{Testability}\index{Testability}: A key characteristic of science is its focus on developing testable hypotheses, that is, statements or predictions that can be empirically investigated. This means that scientific claims must be falsifiable, open to potential disproof if the evidence does not support them. This distinguishes science from fields that rely on unfalsifiable or speculative claims.

\item \emph{Theoretical Frameworks}\index{Theoretical framework}: Science is not merely about collecting facts; it seeks to develop broader explanations through models, laws, and theories. Models offer simplified representations of complex systems, while theories are well-supported frameworks that explain the underlying mechanisms of phenomena. These frameworks help interpret data and guide further investigation, offering coherence to the body of scientific knowledge.

\item \emph{Self-Correction}\index{Self-correction}: A defining feature of science is its self-correcting nature. Scientific theories are not static; they evolve as new evidence comes to light. When new data contradicts a theory, science adapts, modifies, refines, or sometimes discards the theory in favor of one that better fits the evidence. This continuous process of revision ensures that scientific knowledge becomes more accurate over time.

\item \emph{Generalizability}\index{Generalizability}: Science strives to uncover universal principles that apply across various contexts, not just to isolated cases. While science begins with specific observations, its goal is to identify general laws and patterns that explain a broad range of phenomena. This pursuit of generalizable knowledge allows science to predict future occurrences and provide deeper insights into the workings of the natural world.

\end{itemize}

 \emph{Ontology}\index{Ontology}, the branch of philosophy concerned with the nature of existence, addresses the question of which entities exist in the real world. It examines the types of entities that science can study, whether they are observable, abstract, or indeterminate, and engages with the philosophical debates surrounding their existence. Ontology provides the foundation for exploring the boundaries of scientific knowledge and the limits of inquiry. \emph{Epistemology}\index{Epistemology} is the branch of philosophy concerned with knowledge itself—how we acquire it, what justifies it, and what its limits are. While ontology deals with what exists, epistemology addresses how we come to know what exists. In the context of science, epistemology examines the methods, evidence, and reasoning that underpin scientific investigation, exploring how scientific knowledge is built, how reliable it is, and what counts as a justified belief in scientific practice. Both ontology and epistemology are part of the broader field of \emph{metaphysics}\index{Metaphysics}, which deals with the fundamental nature of reality and existence. Metaphysics explores the most basic concepts and categories of being, such as time, space, causality, and possibility, as well as the relationship between mind and matter. Together, ontology, epistemology, and metaphysics provide a comprehensive philosophical foundation for understanding what science studies, how it builds knowledge, and the fundamental nature of the reality science seeks to explain.


%
% Section: What is an Entity
%

\section{What is an Entity}

In this section, we focus on the fundamental problem of determining which kinds of entities can be known or investigated by science.

\begin{figure}[t]
\centering
\begin{tikzpicture}

  % Define the ovals
  \node[draw, ellipse, minimum width=4cm, minimum height=1.1cm, align=center] (knowable) {Knowable};
  \node[draw, ellipse, minimum width=4cm, minimum height=1.1cm, right=of knowable, xshift=2+1cm] (nonknowable) {Non-Knowable};
  \node[draw, ellipse, minimum width=3.5cm, minimum height=1.1cm, below=of knowable, xshift=-2.5cm] (concrete) {Concrete};
  \node[draw, ellipse, minimum width=3.5cm, minimum height=1.1cm, below=of knowable, xshift=2.5cm] (abstract) {Abstract};
  \node[draw, ellipse, minimum width=3cm, minimum height=1.1cm, below=of concrete, xshift=-2.5cm] (observable) {Observable};
  \node[draw, ellipse, minimum width=3cm, minimum height=1.1cm, below=of concrete, xshift=2.5cm] (nonobservable) {Non-Observable};

  % Arrows indicating subset relationships
  \draw[<-] (concrete) -- (knowable) node[midway, left] {};
  \draw[<-] (abstract) -- (knowable) node[midway, right] {};
  \draw[<-] (observable) -- (concrete) node[midway, left] {};
  \draw[<-] (nonobservable) -- (concrete) node[midway, right] {};

\end{tikzpicture}
\caption{Classification of Research Entities}
\end{figure}

A \emph{knowable entity}\index{Knowable entity} is an object, phenomenon, or concept that can be investigated, understood, or described through scientific or intellectual inquiry. Knowable entities are those that, either directly or indirectly, can be observed, measured, inferred, or modeled, using available tools, methods, or theories. They are within the scope of human knowledge, and their properties can be analyzed or explained. Examples of knowable entities include the stars and plantes, animals, or computer algorithms. A \emph{non-knowable entity}\index{Non-knowable entity} refers to something that cannot be directly observed, measured, or understood using current scientific or intellectual methods. This could be due to limitations in technology, the abstract or metaphysical nature of the entity, or inherent epistemological boundaries. Non-knowable entities might remain beyond the reach of human understanding either temporarily (until methods evolve) or permanently (due to their nature). Examples of non-knowable entities include the nature of consciousness, the origin of the universe, or the existence of a deity. The distinction between knowable and non-knowable entities is not always fixed, but varies depending on the clarity and precision of the area of interest and the evolving nature of scientific understanding in each field.

Scientific research encompasses both \emph{concrete entities}\index{Concrete entities} and \emph{abstract entities}\index{Abstract entities}. Concrete entities refer to physical objects or phenomena that can be directly observed, measured, or interacted with, such as stars, cells, or chemical compounds. These entities form the basis of empirical research, where data is collected through direct observation or experimentation. In contrast, abstract entities are conceptual and do not have a physical presence, such as numbers, algorithms, or theoretical models. Abstract entities play a crucial role in scientific research, particularly in fields like mathematics and theoretical physics, where they provide the framework for understanding and modeling concrete phenomena. While abstract entities cannot be observed directly, they can be known through indirect methods. The inclusion of these abstract entities in scientific inquiry raises important philosophical questions about their ontological status: Are these abstract entities real in the same way that physical objects are, or are they simply conceptual tools? This question remains an open and debated issue in the philosophy of science.

Finally, in scientific inquiry, there is also a distinction between \emph{observable entities}\index{Observable entities}, which can be directly perceived or measured, and \emph{non-observable entities}\index{Non-observable entities}, which are inferred from empirical evidence but cannot be directly observed. Observable entities include things like trees, planets, and bacteria—objects that can be seen or detected using scientific instruments. Non-observable, but still concrete, entities include things like subatomic particles and gravitational forces. These entities are often crucial for explaining observable phenomena but exist at a level beyond direct human perception. For example, we cannot observe an electron in the same way we observe a tree, but through scientific theory and experimentation, we infer its existence.

A central debate within the scope of science is the tension between \emph{reductionism}\index{Reductionism} and \emph{holism}\index{Holism}. Reductionism is the view that complex systems can be fully understood by breaking them down into their simplest, most fundamental parts. For example, a reductionist might argue that biological processes can be explained entirely by chemistry, and chemistry by physics. This approach assumes that understanding the smallest components of a system will provide a complete explanation of the whole. In contrast, holism argues that some phenomena cannot be fully understood by reducing them to their components. Instead, the whole system exhibits properties that cannot be predicted or explained by analyzing its parts in isolation. For example, in ecology, the interactions within an ecosystem can produce emergent properties that are not reducible to the behavior of individual species.

Finally, the scope of science is shaped by the debate between \emph{scientific realism}\index{Realism} and \emph{anti-realism}\index{Anti-realism}, which addresses the question of whether the entities posited by scientific theories are real or merely useful constructs. Scientific realism holds that the entities described by scientific theories—whether observable or not—exist independently of our knowledge of them. According to this view, successful scientific theories reveal truths about the world. In contrast, anti-realism (or \emph{instrumentalism}\index{Instrumentalism}) argues that scientific theories are useful tools for predicting and organizing observations, but we should not necessarily believe that unobservable entities like electrons or gravitational waves are real. For anti-realists, the purpose of science is not to describe an independent reality, but rather to provide models that help us navigate and predict phenomena. This ontological uncertainty leads to philosophical discussions about \emph{natural kinds}\index{Natural kinds}—whether the categories used in science reflect real, essential divisions in nature, or whether they are human-made constructs imposed to bring order to a complex and often ambiguous reality. 

%
% Scientific Representation
%

\section{Scientific Representation}
\label{sec:scientific_representation}

Science helps us understand the natural world by using different kinds of representations of research entities. These representations include measurements from scientific instruments, descriptions of observations, digital images like X-rays or MRI scans, and more. Scientific practice also often considers mathematical equations, models, and theoretical constructs as valid forms of representation. The challenge of \emph{scientific representation}\index{Scientific representation} is to identify the conditions that make a representation scientific and determine what makes an effective representation. The main issues discussed in this area of philosophy include:

\emph{Scientific Representation Problem}: The scientific representation problem is about figuring out the necessary and sufficient conditions that make a representation valid in science. It explores whether these conditions are the same across all scientific fields or if they vary depending on the discipline or research context. For example, what qualifies as a valid representation in physics, which often uses mathematical models, might be different from what is used in biology, where visual and descriptive representations are common. This problem also questions whether scientific representations need to be adapted to specific research goals to be considered valid.

\emph{Representational Demarcation Problem}\index{Representational demarcation}: The representational demarcation problem looks at whether scientific representations are fundamentally different from other kinds of representations, like those found in art or everyday life. It examines what makes scientific representations unique, focusing on their purpose, accuracy, and the methods used to create them. Unlike artistic representations, which may emphasize subjective interpretation or aesthetic value, scientific representations are generally held to standards of precision, reliability, and empirical adequacy. Understanding these differences helps clarify the specific role that scientific representations play in knowledge production.

\emph{Problem of Style}\index{Problem of style}: The problem of style addresses the fact that the same entity can be represented in different ways, depending on the goals and methods of the research. Different styles of representation—such as diagrams, mathematical equations, physical models, or computer simulations—each have unique characteristics, including the intended audience, level of abstraction, and type of information conveyed. This issue also asks whether these styles are fixed or if new styles can be invented to meet emerging scientific needs. The flexibility of representation styles is crucial because it allows for new insights and different ways of understanding scientific phenomena.

\emph{Standard of Accuracy}\index{Standard of accuracy}: The standard of accuracy problem is about determining what makes a scientific representation accurate. It involves figuring out how to distinguish between accurate and inaccurate representations by considering factors like how well the representation matches empirical data, captures important features of the phenomenon, and its ability to make predictions. This issue also explores whether accuracy should be seen as an objective standard or if it depends on the specific aims and context of the research. For example, a simplified model might still be considered accurate if it effectively serves its purpose, such as making predictions or providing explanations.

\emph{Problem of Ontology}\index{Problem of ontology}: The problem of ontology in scientific representation deals with the nature of the entities that can serve as representations. It asks whether representations need to be concrete, like physical models or graphs, or if they can also be abstract, like mathematical equations or theoretical constructs. This issue also questions whether representations must be realistic or if more abstract, idealized forms can still be effective in scientific inquiry. Understanding these ontological aspects helps define the types of entities that are allowed in scientific discourse and how they relate to the real-world phenomena they represent.

There are also five conditions of adequacy that a scientific representation should satisfy to be considered effective and reliable:

\emph{Requirement of Directionality}\index{Requirement of directionality}: The requirement of directionality examines the relationship between representations and the real world. Representations are meant to describe entities in the real world, but this condition raises the question of how, if at all, real-world entities might describe their representations. It challenges us to think about the direction of influence between the representation and the entity it aims to depict.

\emph{Surrogative Reasoning}\index{Surrogative reasoning}: Surrogative reasoning addresses how scientific representations allow researchers to generate hypotheses about the entities they represent. This condition explores how using a representation as a surrogate can lead to new insights or predictions about the target, effectively using the representation to stand in for the real-world entity during reasoning and analysis.

\emph{Applicability of Mathematics}\index{Applicability of mathematics}: The applicability of mathematics condition is concerned with how mathematical models can be used to represent the real world. It questions how abstract mathematical constructs can effectively describe complex physical systems and whether the success of mathematical representation depends on any special features of the target phenomena. This condition highlights the central role of mathematics in developing and understanding scientific theories.

\emph{Possibility of Misrepresentation}\index{Possibility of misrepresentation}: The possibility of misrepresentation addresses whether representations that are not fully accurate can still be considered valid scientific representations. It considers situations where simplifications or approximations are necessary and whether these less-than-perfect representations can still contribute valuable understanding of a phenomenon. This condition is important for understanding how idealizations and abstractions function in scientific practice.

\emph{Targetless Models}\index{Targetless models}: The targetless models condition explores whether we can allow representations that do not have a direct real-world counterpart. It questions if a model that does not represent any existing entity can still be useful in scientific inquiry, perhaps as a way to explore theoretical possibilities or to understand potential scenarios. This condition emphasizes the creative and exploratory aspects of scientific modeling.

There have been multiple proposals to formally define the concept of scientific representation. Unfortunately, none of these proposals can provide a convincing answer to the questions and conditions of adequacy described above. In the rest of this section, we describe some of these proposals, identifying their advantages and drawbacks. To compare these proposals, we will present them as: "A scientific model $M$ represents a target system $T$ if, and only if ...".

\emph{Stipulative Fiat}\index{Stipulative fiat}: The stipulative fiat proposal states that "a scientific model $M$ represents a target system $T$ if, and only if, a scientist stipulates that $M$ represents $T$." The main problem with this interpretation is that, since anything can be a representation if a scientist says so, it is difficult to guarantee the surrogative reasoning condition. If any model can be deemed a representation by simple stipulation, it becomes challenging to determine which representations are genuinely useful for making scientific inferences. Proponents of this theory acknowledge that while all representations may be stipulated, some are undeniably more useful than others.

\emph{Similarity Conception}\index{Similarity conception}: The similarity conception proposes that "a scientific model $M$ represents $T$ if, and only if, $M$ and $T$ are similar." This conception addresses the surrogative reasoning condition since similarity between the model and the target allows us to derive similar properties. However, it introduces new challenges, particularly regarding the problem of style. The concept of similarity is often vague: in what sense are $M$ and $T$ similar? This vagueness can lead to issues with directionality and accuracy, as different aspects of similarity may not always align with what is relevant for scientific representation.

\emph{Structuralist Conception}\index{Structuralist conception}: The structuralist conception is based on the idea of isomorphism. According to this view, a scientific model $M$ represents a target system $T$ if the structure of $M$ is isomorphic to the structure of $T$. In other words, there is a one-to-one correspondence between the elements and relationships in both $M$ and $T$. This approach justifies surrogative reasoning because having the same structure implies that properties and relations in the model correspond to those in the target. Furthermore, since mathematics is fundamentally concerned with the study of structures, this conception also supports the applicability of mathematics in representing natural systems.

\emph{Inferential Conception}\index{Inferential conception}: The inferential conception proposes that a model $M$ is an epistemic representation of a target $T$ if, and only if, the user adopts an interpretation of $M$ in terms of $T$. This view emphasizes the role of the user in giving meaning to the model, suggesting that representation is not an inherent property of the model itself but arises through its use in making inferences about the target system. This conception underscores the importance of context and interpretation in determining whether a model effectively represents its target.

\emph{Fiction View of Models}\index{Fiction view of models}: According to the fiction view of models, $M$ represents $T$ if and only if $M$ functions as a prop in a game of make-believe that prescribes imagining certain things about $T$. This view draws an analogy between scientific modeling and storytelling, where models are treated as fictional constructs that facilitate imaginative engagement with the target system. Although this approach highlights the creative aspects of modeling, it raises questions about how such fictional constructs can be rigorously linked to real-world entities.

\emph{Representation-As}\index{Representation-as}: The representation-as approach suggests that a scientific model represents a target system as something, emphasizing that representation involves highlighting certain features of the target while downplaying others. This conception focuses on the interpretive aspect of modeling, where the modeler selects specific attributes of the target to represent, depending on the research goals. This approach allows for a flexible understanding of representation that can accommodate different styles and purposes, but it also implies that the usefulness of a representation is contingent on how well the modeler captures the relevant aspects of the target.

Each proposal has its strengths and weaknesses, highlighting the complexity of what it means for a model to effectively represent a target in scientific inquiry. Understanding these various perspectives is crucial for advancing our comprehension of the role of representation in science.


%
% Laws, Models and Theories in Science
%

\section{Scientific Explaination}

{\color{red}

It is assumed that there is a single kind or form of explanation that is "scientific". The notion of "scientific explanation" suggest at least two contrasts -- first, a contrast between those "explainations" that are characteristics of "science" and those explanations that are not, and, second, a contrast between "explanation" and something else  (e.g. mere "descriptions") [...] It is possible for a set of claims to be true, accurate, supported by evidence, and so on and yet unexplanatory. [...] failure to capture all of these forms of "explanation" [...] their intended explicandum is, very roughly, explanations of why things happen, where the "things" in question can be either particular events or something more general [...] concepts lie "explanation", "law", "cause" and "support for counterfactuals" as part of an interrelated family of concepts that are "modal" in character [...] it would be "circular" to explain one concept from this family in terms of others from the same family [...] they must be explained in terms of other concepts from outside the modal family [...] A related issue has to do with whether all scientific explanations are causal and if not, what distinguishes casual from non-causal explanations.

}

The \emph{Deductive-Nomological model}\index{Deductive-Nomological model}, or DN model, emphasizes the importance of deductive reasoning and general laws. According to this model, a phenomenon is explained by demonstrating how it logically follows from a general law combined with specific initial conditions. This an explanation comprises two main components: the \emph{explanans}\index{explanans}, which includes the general laws and initial conditions, and the \emph{explanandum}\index{explanandum}, which is the phenomenon to be explained. For the explanation to be valid, the explanans must be true and logically entail the explanandum, meaning the phenomenon can be deduced from the laws and conditions provided. A DN model answers the question "Why did the explanandum occur?" by showing that the phenomenon resulted from specific circumstances $C_1, C_2, \ldots, C_i$, in conjunction with laws $L_1, L_2, \ldots, L_j$. For example, the motion of a pendulum can be explained by applying Newton's laws of motion (general laws) along with specific details such as the pendulum's length and initial displacement.

The \emph{Statistical-Relevance model}\index{Statistical-Relevance model}, or SR model focuses on explaining phenomena through statistical relationships rather than strict deductive reasoning. Unlike the Deductive-Nomological model, which requires logical entailment from general laws, the SR model emphasizes the identification of statistically relevant factors that significantly influence the likelihood of a phenomenon. In this model, given some class or population \( A \), an attribute \( C \) is \emph{statistically relevant}\index{Statistical relevance} to another attribute \( B \) if and only if \( P(B | A, C) \neq P(B | A) \). This means that \( C \) affects the probability of \( B \) within the context of \( A \). An explanation involves identifying such statistically relevant factors and evaluating their impact within a reference class (a group of events or entities sharing common characteristics). For example, in explaining the likelihood of developing a particular disease, an SR explanation might highlight factors such as age, genetic predisposition, or lifestyle choices, showing how these variables alter the probability of the disease occurring. By uncovering these statistical relationships, the SR model provides a method for explaining probabilistic phenomena that cannot be addressed deterministically.

Causality

{\color{red}

There is considerable disagreement among philosophers about whether all explanations in science and in ordinary life are causal and also disagreement about what the distinction (if any) between causal and non-causal explanations consists in. 

}

--

Models are one of the main tools we have today to do science. From an ontological point of view, there is a large variety of things that can be considered as models. Models can be physical objects, for example, scaled down (or scaled up) pieces made of wood or metal, a wood model of a car; they can be also fictional, that is, abstract ideas residing in the mind of scintists, like Borh model of the atom. Mathematical models, either set theoretic structures [... example ...] or equations, like the Black-Scholes partial differential equations to estimate the price of some financial derivative products like options. Finally, we could also as models descriptions, like for example, the ones included in scientific papers.

From a semantic point of view, models can be also representations of target systems, in the sense already covered in the previous section. In this sense, models have the same problems already covered. And they have to be considered for each type of model (physical, fictional, mathematicals, ...). An idealized model {\color{red} is a deliberate simplification of something complicated with the objective of making it more tractable [...] Aristotelian idealization amounts to 'stripping away' [...] all properties from a concete object that we believe are not relevant to the problem at hand [...] Galilean idealizations are ones that involve deliberate distortions [...] point masses [...].} In mathematical models we have also approximations.

{\color{red} [...] epistemology [...] how do we learn with models? [...] Models are vehicles for learning about the world [...] models allow for surrogative reasoning [...]}

{\color{red} Learning about a model happens at two places, in the construction and the manipulation of the model [...] There are no fixed rules or recipes for model building and so the very activity of figuring out what fits together and how affords and opportunity to learn about the model. Once the model is built, we do not learn about its properties by looking at it; we have to use and manipulate the model in order to elicit its secrets [...] by performing though experiment [...] An importatn class of models is of mathematical nature [...] solve equations analytically [...] making a computer simulation.}

{\color{red} Once we have knowledge about the model, this knowledge has to be 'translated' into knowledge about the target system [...] there do not seem to be any general accounts of how the transfer of knowledge from a model to its target is achieved}

{\color{red} [...] philosophy of science [...] how do models relate to theory? Models and other debates} 


%
% Scientific Discovery
%

\section{Scientific Discovery}

\begin{verbatim}
   - **Empiricism vs. Rationalism**  
     Exploring different approaches to acquiring knowledge in science.
\end{verbatim}

Scientific discovery refers to the process of conceiving new scientific ideas, hypotheses or novel explanations. Scientific discovery involves an "eureka moment" or "happy though" in which the new idea is sough, its formal articulation, and the validation process. In this section, we are interested in the fist part of this process, that is, the eureka moment. We are interested in the nature of this insightful moment, and in particular, if it can be analyzed, and if there exists rules, algorithms, guidelines, or heuristics, to generate these novel insights. We do not consider in this section if those hypotheses are worth articulating and testing.

{\color{red} explain that during this epoch, doing sience and meta-science was the same activity} During the 17th and 18th centuries, great philosophers, like Bacon, Descartes and Newton proposed methods to discover new knowledge. {\color{red} Briefly describe their ideas}

{\color{red} [...] the two pocesses of conception and validation of an idea or hypothesis became distinct, and the view that the merit of a new idea does not depend on the way in which it was arrived at became widely accepted.} 

The general agreement among philosophers is that the creative process of conceiving a new idea is a non-rational process that can not be formalized as a set of steps. Current philosophy of science focus on the formulation and justification of new ideas rather than finding them. {\color{red} because philosophy of science is intended to be normative}

{\color{red} Discovery as abduction [...] the act of discovery [...] follows a distinctive logical pattern, which is different from both inductive logic and the logic of hypothetico-deductive reasoning. The special logic of discovery is the logic of abductive or "retroductive" inferences [...] an inference beginning with [...] surprising or anomalous phenomena [...] discovery is primarily a process of explaining anomalies or surprising, astonishing phenomena. The scientists' reasoning proceeds abductively from an anomaly to an explanatory hypothesis in light of which the phenomena would no longer be surprising or anomalous [...] the schema of abductive reasoning does not explain the very act of conceiving a hypthesis or hypothesis-type.}

{\color{red} Heuristic programming [...] artificial intelligence at the intersection of philosophy of science and cignitive science [...] problem solving activity [...] whereby the systematic aspects of problem solving are studied within an information-processing framework. The aim is to clarify with the help of computational tools the nature of the methdos used to discover scientific hypothesis [...] searches for solutions [...] "problem space" in a certain domain [...] the basic idea behind computational heuristics is that rules can be identified that serve as guidelines for finding a solution to a given problem quickly and efficiently by avoiding undesired states of the problem space [...] the data from actual experiments the simulations cover only certain aspects of scientific discoveries [...] they do not design newe expeirments, instrumetns, or methods [...] the complex problem spaces for scientific problems are often ill defined}

{\color{red} In recent decades, philosophers have subsumed their interest in this eureka moment. However, the research is no only philosophy based, they borrow ideas and collaboate, with areas like cognitive science, neuroscience, computational research, and environmental and social psychology, philosophers have sought to demystify the cotnitive processes involved in the generation of new ideas}

{\color{red} [...] A discovery is not a simple act, but an extended, complex process, which culminates in paradigm changes. Paradigms are the symbolic generalizations, metaphysical commtments, values, and exemplars that are shared by a community of scientists and that guide the research of that community [...] A discovery begins with an anomaly, that is, with the recognition that the expectations induced by an established paradigm are being violated.}

Some authors have tried to define exaclty what we mean by being creative, proposin that it is novel, surprising and important.

{\color{red} [...] the role of analogy in the development of new knowledge, whereby analogy is understood as a process of bringing ideas that are well understood in one domain to bear on a new domain [...] the distinction between positive, negative and neutral analogies [...] the distinction between horizontal and vertial analogoies between domains.}

Model-based reasoning proposes that much of the human problem solving is based on mental models rather than the application the laws of logic to a collection of propositions. According to this theory, human mind uses model based representations to visualize how the world works, and to manipulate the structure of this models, using tools like analogy, or thought experiments. Unfortunately, the concept of model is too vague.

The formal methods of scientific discovery are covered in Section XXX.

{\color{red} Psychological studies of crative individuals' behavioral dispositions suggest that creative scientists share certain personality traits, including confidence, openness, dominance, independence, introversion, as well as arrogance and hostility [...] creative individuals usually have outsider status - they are socially deviant and diverge from the mainstream}


%
% The Scientific Method
%

\section{The Scientific Method}

\begin{verbatim}
3. **Scientific Method and Reasoning**  
   - **Induction and Deduction**  
     Examining the role of inductive and deductive reasoning in science.
   - **The Hypothetico-Deductive Method**  
     A closer look at how hypotheses are formulated and tested in scientific practice.
   - **Falsifiability and Confirmation**  
     Karl Popper’s theory of falsifiability and its significance in demarcating science from pseudoscience.
\end{verbatim}

{\color{red} The study of the scientific method is the attempt to discern the activities by which [science is an enormously success]. Among the activities often identified as characteristic of science are systematic observation and experimentation, inductive and deductive reasoning, and the formation and testing of hypotheses and theories [...] methods are the means by which [the goals of science] are achieved [...] methodological rules are proposed to govern method [...] method is distinct [...] from the detailed and contextual practices through which methods are implemented [...] how pluralist do we need to be about method? [...] how much can method be abstracted from practice? [...] Unificationists continue to hold out for one method essential to science; nihilism is a form of radical pluralism, shich considers the effectiveness of any methodolocial prescirption to be so context snsitive as to rendeer it not explanatory on its own.}

{\color{red} [...] scientific activity varies so much across disciplines, times, places, and scientists that any account which manages to unify it all will either consists of overwhelming descriptive detail, or trivial generalization [...] For most of the history of scientific methodology the assumption has been that the most important output of science is knowledge and so the aim of methodology should be to discover those methods by which scientific knowledge is generated [...] very few philosophers arguing any longer for a grand unified methodology of science}

{\color{red} On the hypothetico-deductive account, scientist work to come up with hypotheses from which true observational consequences can be deduced.}

{\color{red} A distinction in methodology was made between the contexts of discovery and of justification. The distinction could be used as a wedge between, o the one hand the particularities of where and how theories or hypotheses are arrived at and, on the other, the underlying reasoning scientiesi use [...] when assessing theories and judging their adequacy on the basis of the available evidence. By and large [...] philosophy of science focused on the second context.}

{\color{red} [...] the Hypothetico-Deductive (H-D) method [...] a theory [...] is confirmed by its true consequences}

{\color{red} Method may therefore be relative to discipline, time or place [...] by the close of the 20th century the search by philosophers for the scientific method was flagging.}

{\color{red} A problem with the distinction between the contexts of discovery and justification [...] is that no such distincition can be clearly seen in scientific activity [...] new scientific concepts are constructed as solutions to specific problems by systematic reasoning, and that of analogy, visual representation and though-experimentation are among the importatn reasoning practices employed [...] model-based reasoning consists of cycles of construction, simulation, evaluation and adaption of models that serve as interim interpretations of the target problem to be solved [...] this proess will lead to modifications or extensions, and a new cycle of simulation and evaluation [...] there is no logic of discovery [...] a large and integral part of scientific practice is [...] the creation of concepts through which to comprehend, structure, and communicate about physical phenomena [...] science as problem solving [...] scientific problem solving as a special case of problem-solving in general [...] the primary role of expeirments is to test theoretical hyptheses according to the H-D model [...] exploratory experimentation was introduced to describe experiments dirven by the desire to obtain empirical regularities and to develop concepts and classifications in which there regularities can be described [...] the development of high trhoughput instrumentation [...] has given rise to a special type of exploratory experimentation that collects and analyses very large amounts of data [...] data-driven research.}

{\color{red} [...] the ability of computers to process, in a reasonable amount of time, the guge quantities of data [...] computers allow for more elaborate experimentation [...] but also, through modelling and simulations, migh constitue a form of experimentation themselves [...] does the practice of using computers fundamentatlly change scientific method, or merely provide a more efficient means fo implementing standar methods? [...] Because computers [...] many of the steps involved in reaching a conclusion on the basis of experiment are nore made inside a "black box" [...] we ought to consider computer simulatin a "qualitatively different way of doing science" [...] simulation as a "third way" for scientific methodology (theoretical reasoning and experimental practice are the first two ways)}

{\color{red} [...] a fixed four or five step procedure starting from observations and description of a phenomenon and progressing over formulation of a hypothesis which explains the phenomenon, designing and conducting experiments to test the hypothesis, analyzing the results, and ending with drawing a conclusion [...] conclusion of recent philosophy of science that there is not any unique, easily described scientific method.}


%
% The limits of science
% 

\section{The Limits of Science}

\begin{verbatim}
   - **Reductionism vs. Holism**  
     Discussion on whether complex phenomena can be reduced to basic scientific laws or require holistic approaches.
   - **The Limits of Scientific Explanation**  
     Addressing the boundaries of scientific inquiry, such as metaphysical or moral questions.
   - **Scientific Realism vs. Anti-Realism**  
     Debate about whether science uncovers true aspects of the world or just useful models.
   - **The Problem of Objectivity**  
     Exploring whether science can be truly objective or is influenced by social and personal values.
   - ** The Demarcation Problem
     Discussing the philosophical challenge of clearly distinguishing between science and non-science (including pseudoscience), and why this boundary is often contested.

Limitations of Scientific Knowledge
Addressing areas where science may have limited access, such as consciousness, moral values, or metaphysical questions.

\end{verbatim}


%
% Section: References
%

\section*{References}

{\color{red} Add entry of Scientific Representation in the Stanford Encyclopedia of Philosophy}

{\color{red} Perhaps add entries for Lewis Carroll's Sylvie and Bruno and Borges' On Exactitude in Science}

\ref{pirsig1999zen} contains an interesting review of the concept of science, the scientific method, and the role that technology plays in our society. The author proposes that the goal of science should be quality, although the concept of quality is left undefined, and how to reconcile the rational and romantic points of view in science. The book also contains some advice about which is the right state of mind to pursue a scientific problem, and how to deal with the inevitable failures.


