%
% CHAPTER 7.- Interesting Questions
%

%
%  Section 1
%    - (major) We have to clarify which description function we are using
%  Section 4
%    - (major) Missing reference to combined nescience proposition
%

\chapterimage{thinker.pdf}

\chapter{Interesting Questions}
\label{chap:Interesting-Research-Questions}

\begin{quote}
\begin{flushright}
\emph{It is not the answer that enlightens,\\
but the question. \\}
Eugène Ionesco
\end{flushright}
\end{quote}
\bigskip

In this chapter, we propose a set of metrics for classifying research topics based on their potential as a source of interesting problems, along with a methodology for assisted discovery of new questions. The methodology can be used to identify new applications of existing tools to solve open problems and to discover new, previously unexplored research topics. The methodology is applicable to both intradisciplinary and interdisciplinary topics, but the most interesting outcomes are obtained in the latter case. In Chapters \ref{chap:philosophy-science} and \ref{chap:computational-creativity}, we demonstrate the application of the methodology in practice and suggest new questions and research topics.

Our primary assumption is that a research question is interesting if it meets the following three criteria\index{Criteria for Interesting Questions}:

\bigskip

\begin{description}
\item[C1] The question should be new and original, meaning that it has not been previously considered.
\item[C2] Upon its resolution, there should be a significant increase in our knowledge about one or more specific research topics.
\item[C3] It should have practical applications that could have a substantial (hopefully positive) impact on people's lives.
\end{description}

\bigskip

Some researchers may argue that the requirements presented may not be the most suitable. For instance, researchers from the so-called "hard sciences", such as pure mathematics and theoretical physics, might object that practical applications are not a crucial factor in pursuing an interesting open problem. In such situations, the metrics introduced can be redefined since they are simply mathematical abstractions, and the same methods can still be applied. The methodology described is universal and can be employed in various domains, not only in discovering new research questions. In fact, the metrics and methods outlined can be utilized in any field where there is a vast collection of interconnected describable objects, and the objective is to uncover new and previously unknown objects. The precise definition of concepts like relevance graph, applicability graph, or maturity will be contingent on the field in which the approach is being employed. Nevertheless, as in the case of Chapter \ref{chap:Nescience}, we prefer to present the methodology and new concepts in the specific context of scientific research because it aids in their comprehension.

It is worth mentioning the relationship between this new methodology and the areas of computational creativity and artificial intelligence. What is proposed in this chapter is an algebraic approach to the assisted discovery of potentially interesting questions. The intention is not for the computer to understand the meaning of the questions posed. Furthermore, not all the questions generated are necessarily relevant or even meaningful. It is the responsibility of the researchers to assess the proposed combinations of topics and determine if any of the questions are appropriate.

We have already explored two dimensions for classifying topics: miscoding (Chapter \ref{chap:Miscoding}) and mismodel (Chapter \ref{chap:Redundancy}). These metrics enable us to quantitatively assess our understanding of a topic, which we refer to as nescience. According to criterion \textbf{C2} of our list of requirements for questions, the higher the nescience of a topic, the greater its potential as a source of interesting research questions. In this section, we will introduce two additional metrics for characterizing topics: relevance and applicability. Relevance measures the impact a topic has on people's lives and serves as a complement to nescience. Applicability measures how frequently a topic has been applied in other areas and enables us to identify new applications of already existing technologies.

%
% Section: Combination of Topics
%

\section{Combination of Topics}

{\color{red} TODO: rewrite this section}

In this section we propose a theoretical model for the process of discovering new research topics by means of the (creative) combination of other, already known, topics. The proposed model is a highly simplified version of what the researchers do in practice. However, it allows us to study some of the properties of this process, and to automate it in practice.

\begin{definition}
Let $r$ and $s$ be two topics. A \emph{creative combination} of topics $r$ and $s$ is a topic $t \in \mathcal{T} \backslash \{r, s, \lambda\}$
\end{definition}

Let $r$ and $s$ be two topics, and let $CM$ the set of Turing machines that given as input $r$ and $s$ produce a new topic $t \in \mathcal{T}$. In order to be a \emph{creative combination}, we require that the topic t must be different from $r$, $s$ and $\lambda$, and the complexity of $K(t)$ must be greater than the length of the machine. Formally:
\[
CM = \{ T \in TM : T(r, s) = t, t \in \mathcal{T} \backslash \{r, s, \lambda\}, K(t) \geq l(T) \}
\]

Let $\mathcal{C_{r,s}}$ the creative set generated by $r$ and $s$.


\begin{definition}

\end{definition}



\begin{definition}
Let $r$ and $s$ be two topics, and let T be the set of creative Turing machines given $s$ and $t$. We define the creative combination of the topics $r$ and $s$, represented by $r \oplus s$ is defined as the topic $t$ produced by the shortest Turing machinge, that is:
\[
r \oplus s = \min_{l(T)} \{ t \in \mathcal{T} : t = T(r,s) \}
\]
\end{definition}

ddd

\begin{proposition}
The set of topics $\mathcal{T}$ with the operation of combination $\oplus$ form a commutative monoid. That is:
\begin{itemize}
\item For every two topics $s$ and $t$ we have that $s \oplus t = t \oplus s$.
\item  For every three topics $r$, $s$ and $t$ we have that $r \oplus \left( s \oplus t \right) = \left( r \oplus s \right) \oplus t$.
\end{itemize}
\end{proposition}
\begin{proof}
\end{proof}

%
% Section: Relevance
%

\section{Relevance}

Before to measure the impact of a research topic in people's life, we have to introduce the concept of \emph{relevance graph}\index{Relevance graph}. The relevance graph is a graph that describes which people is affected by which research topics.

\begin{definition}\index{Relevance Graph}
\label{def:relevance-graph}
We define the \emph{relevance graph}, denoted by $\mathbf{RG}$, as the bipartite graph $\mathbf{RG} = (\mathcal{T}, \mathcal{P}, E)$, where $\mathcal{T}$ is the set of topics, $\mathcal{P}$ the set of people, and $E\subseteq\left\{ \left(i,j\right):i\in \mathcal{T},j\in \mathcal{P} \right\}$ is the set of arcs between topics and people. An arc $(i, j)$ belong to $E$ if, and only if, person $j$ is affected by topic $i$.
\end{definition}

When we refer to the set of people $\mathcal{P}$, we are referring to all individuals in the world. An edge in the relevance graph indicates that someone is affected by a topic, rather than being interested in it. The exact meaning of "being affected by" is a complex epistemological concept that is beyond the scope of this book. Therefore, our definition of the relevance graph is a mathematical abstraction. In Section \ref{sec:Classification_Research_Topics}, we will explore how to approximate this quantity in the case of scientific research topics. In Chapter \ref{chap:Software-Engineering}, we will provide an alternative interpretation of the relevance graph in the context of measuring software quality.

\begin{example}
A man who is affected by ALS (amyotrophic lateral sclerosis) disease will be connected to the ALS topic in the graph. His spouse will also be connected because she is likely to be impacted by the consequences of the disease as well. However, a researcher who is interested in ALS as a research problem, rather than someone suffering the disease, will not be connected to the ALS node in the graph.
\end{example}

Optionally, we can add a weight $w_{ij}\in\left[0,1\right]$ to the edges of the graph to specify the degree in which a person $j$ is affected by a topic $i$. A weight of $1$ could represent a life-or-death dependence, and $0$ would mean that this person is not affected at all. In Figure \ref{fig:Relevance-Graph} it is depicted an example of relevance graph. 

\begin{figure}[h]
\centering\includegraphics[scale=0.7]{bipartite_graph}
\caption{\label{fig:Relevance-Graph}Relevance Graph}
\end{figure}

\begin{definition}\index{Relevance}
\label{def:relevance}
We define the \emph{relevance} of a topic $t \in \mathcal{T}$, denoted by $R(t)$, as the degree of the node $t$ in the relevance graph, that is, $R(t) = deg(t)$.
\end{definition}

Intuitively, the higher the relevance of a topic, the higher its potential as a source of interesting questions, since we will be working on a problem that affects many people.

Sometimes it is covenient to work with a normalized version of the relevance of a topic.

\begin{definition}\index{Normalized relevance}
\label{def:normalized_relevance}
We define the \emph{normalized relevance} of a topic $t \in \mathcal{T}$, denoted by $\bar{R}(t)$, as the normalized degree of the node $t$ in the relevance graph, that is, $\bar{R}(t) = deg(t) / d(E)$.
\end{definition}

We could have computed also $deg(p)$, that is, the number of edges that links to a person $p$ in the relevance graph, as a measure of the number of topics that affects a particular person. However, this quantity is not used in the theory of nescience. The relation between $deg(t)$ and $deg(p)$ is given by the degree sum formula:
\[
\sum_{t \in \mathcal{T}} deg(t) = \sum_{p \in \mathcal{P}} deg(p) = d(E)
\]

Next proposition proves that adding more topics to a research project can only increase its relevance. Of course, a research project dealing with "life, the universe and everything" would be a highly relevant one, but very impractical as well. How to properly combine research topics will be described in Section \ref{sec:New_Research_Topics}.

\begin{proposition}
\label{prop:nondecreasing_relevance}
Given any two topics $t_{1}, t_{2} \in \mathcal{T}$, we have that $R(t_{1}) + R(t_{2}) \geq R(t_{1})$.
\end{proposition}
\begin{proof}
Let $S_{1}$ the set of people connected to to topic $t_{1}$ in the relevance graph, and  $S_{2}$ the set of people connected to to topic $t_{2}$. Since $d \left( S_{1} \cup S_{2} \right) = d(S_{1}) + d(S_{2}) - d(S_{1} \cap S_{2})$, and $d(S_{2}) - d(S_{1} \cap S_{2}) \geq 0$ we have that $d \left( S_{1} \cup S_{2} \right) \geq d(S_{1})$ and thus $R(t_{1}) + R(t_{2}) \geq R(t_{1})$.
\end{proof}

{\color{red} TODO: Rewrite the rest of this section in terms of a multi-objective optimization problem. Detail its properties.}

Finally, we define the concept of interestingness of a topic as a source of interesting problems, that is, how likely is that the topic can be used in a new interesting research question, as a function of its relevance and nescience.

\begin{definition}\index{Interestingness of a topic as a problem}
Given a topic $t \in \mathcal{T}$, we define the \emph{interestingness of the topic as a problem}, denoted by $IP(t)$, as:
\[
IP(t) = \nu(t) R(t)
\]
\end{definition}

Intuitively, a topic is interesting as a problem worth investigating if it has a large relevance (it has high impact in people's life) and a large nescience (it is not very well understood). In this sense, we are borrowing ideas from Popper's falsificationism: the more risky is a conjecture, the higher the advance achieved in science given its confirmation.

\begin{example}
\label{ex:fixed_point}
The fixed point theorem has some relevance, since people life's can be indirectly affected by its implications, but since it is a very well understood theorem (our nescience is very low), it is not a very interesting research problem by itself.

World War I is a very relevant topic, because it had a huge impact on many people's life, and also it is not very well understood topic, since it takes hundreds of pages to explain its causes, and there is no general agreement among the specialists. So, according to our definition, it is a very interesting research problem.
\end{example}

%
% Applicability
%

\section{Applicability}

As we mention in Example \ref{ex:fixed_point}, the fixed point theorem is not very interesting as a research problem by itself. However, it is a very important theorem, since it can be applied to prove many other theorems. In this section we are going to introduce the concept of \emph{applicability}, a new measure that allows us to identify which topics are important because they can be used as a tools.

In order to formally define the concept of applicability, first we have to introduce a new graph that describes which topics have been applied as tools in other topics.

\begin{definition}\index{Applicability graph}
\label{def:applicability-graph}
We define the \emph{applicability graph}, denoted by $AG$, as the directed graph $AG = (\mathcal{T}, E)$, where $\mathcal{T}$ is the set of research topics, and $E\subseteq\left\{ (i,j):i,j\in \mathcal{T} \right\} $. An arc $(i, j)$ belong to $E$ if, and only if, topic $t_i$ has been applied to topic $t_j$.
\end{definition}

For example, there is a direct link between the topics "graph theory" and "recommendation engines", since graph theory has been successfully applied to the problem of how to recommend purchase items to customers over Internet.

{\color{red} TODO: Say something about how the applicability graph can be computed in practice.}

{\color{red} TODO: Include a picture of applicability graph.}

\begin{definition}\index{Applicability}
\label{def:applicability}
We define the \emph{applicability} of a topic $t\in T$, denoted by $A(t)$, as the outdegree of that node in the applicability graph, that is:
\[
A(t) = outdeg(t)
\]
\end{definition}

Intuitively, the higher the applicability of a topic, the higher is its potential as a tool that can be applied to solve new problems. If a tool has been already applied many times in the past to solve open problems, it is likely that it can be applied to solve other open problems as well.

{\color{red} TODO: We should provide a normalized definition of applicability.}

{\color{red} TODO: Mention here we can use other, alternative, centrality measures to compute applicability.}

Next proposition proves that the combination of two topics can only increase their applicability. That is, the more tools we have at our disposal, the more problems we could solve in principle.

\begin{proposition}
Given any two topics $t_1, t_2 \in \mathcal{T}$, we have that $A(t_1) + A(t_2) \geq A(t_1)$.
\end{proposition}
\begin{proof}
Use the same argument than in Proposition \ref{prop:nondecreasing_relevance}.
\end{proof}

to provide a new metric to measure how \emph{mature} is a topic.

\begin{definition}\index{Maturity}
Given a topic $t \in \mathcal{T}$, we define the \emph{maturity} of topic $t$, denoted as $M(t)$, as the inverse of nescience, that is:
\[
M(t) = \nu(t)^{-1}
\]
\end{definition}

Intuitively, the more mature is a topic the higher is it potential applicability to solve other open problems. Highly immature topics should not be applied to solve open problems, since they could provide wrong solutions.

{\color{red} TODO: We should provide a normalized definition of relevance.}

\begin{example}
{\color{red} TODO: Find an illuminating example.}
\end{example}

Finally, we can define the concept of interestingness of a topic as a source of interesting tools, that is, how likely is that the topic can be used to solve a new problem, as a function of its maturity and applicability.

\begin{definition}\index{Interestingness of a topic as a tool}
Let $t \in \mathcal{T}$ a topic. We define the \emph{interestingness of the topic as a tool}, denoted by $IT(t)$, as:
\[
IT(t) = A(t) M(t)
\]
\end{definition}

Intuitively, a topic is interesting as a tool if it has been already applied to many other problems, and it is very well understood topic.

{\color{red} TODO: Again, mention why the product.}

\begin{example}
{\color{red} TODO}
\end{example}

%
% Section: Interesting Questions
%

\section{Interesting Questions}

The set of topics $\mathcal{T}$ is composed by all possible topics, including those topics we are not yet aware of its existence. However, in order to pose interesting questions, we have to restrict our search to only those topics we already know. How to discover new interesting research topics will be described in Section XX.

\begin{definition}\index{Set of known research topics}
The set of \emph{known research topics}, denoted by $T'$, is the subset of $T$ composed by all those topics that have been the subject of a research activity.
\end{definition}

A questions is a pair of a topics $t_{1}, t_{2} \in T'$. Intuitively, the question would be something like "\emph{can we apply the tool described by topic $t_{1}$ to solve the problem described by topic $t_{2}$?}".

\begin{definition}\index{Question}
Given two topics $t_{1}, t_{2} \in T'$, a \emph{question}, denoted by $Q_{t_{1}\rightarrow t_{2}}$, is the ordered pair $\left(t_{1},t_{2}\right)$.
\end{definition}

The most interesting questions will arise when topic $t_{1}$ has a high interestingness as a tool, and a topic $t_{2}$ has a high interestingness as a problem. The interestingness of the new question is defined as a function of the interestingness of the topics involved.

\begin{definition}\index{Interestingness of a question}
The \emph{interestingness} of the question $Q_{t_{1}\rightarrow t_{2}}$, denoted by $IQ_{t_{1} \rightarrow t_{2}}$, is given by:
\[
IQ_{t_{1} \rightarrow t_{2}}=A_{t_{1}}R_{t_{2}}+M_{t_{1}}N_{t_{2}}
\]
\end{definition}

In practice, what we have to do is to compute the all the possible combination of those topics with very large $IT$ with those other topics with very large $IP$, and select those combinations with higher $IQ$. Of course, most of the questions posed with this approach will be meaningless, but the same happens when researchers organize brainstorming sessions to identify new tools to solve difficult problems.

We can easily generalize the above procedure to multiple tools and, perhaps, multiple problems. In this way we came up with the application of two tools to a given problem ($T+T \rightarrow P$), the application of a single tool to the combination of two problems ($T \rightarrow P + P$), and so on. The exact meaning of those combinations of tools and problems depends on the topics themselves, and so, it is left to the creative interpretation of the researcher.

From a practical point of view, sometimes it is convenient to restrict ourselves to certain areas to find interesting questions, or even to specific topics. For example, if a researcher is specialist in area $A$, he might be interested in finding interesting problems where he can apply his knowledge. In the same way, a researcher specialist in problem $t$ could be interested in finding new tools to solve the problem.

\begin{definition}\index{Intradisciplinary question} \index{Interdisciplinary question}
Let $A \subset T'$ a research area, and $t_{1}, t_{2} \in T'$ two topics. If both topics belongs to the same area, that is $t_{1}, t_{2} \in A$, we say that the question $Q_{t_{1} \rightarrow t_{2}}$ is \emph{intradisciplinary}, otherwise, we say that the question is \emph{interdisciplinary}.
\end{definition}

In principle, the most innovative questions would be the interdisciplinary questions, because the probability that somebody has though about them is lower, since it requires specialists in both research areas working together to come up with that particular question. Interdisciplinary questions would address my requirement \textbf{A1} of what makes a question interesting, that is, the question is new and original.

%
% Section: New Research Topics
%

\section{New Research Topics}
\label{sec:New_Research_Topics}

In the previous section my focus was in how to find new interesting research questions. In this section I will go one step beyond, and I will show how to identify new, previously unconsidered, research topics.

\begin{definition}\index{Unknown frontier}
In the two-dimensional space defined by relevance and nescience, the \textit{unknown frontier}, denoted by $\mathbb{F}$, is defined as the following arc:
\[
\mathbb{F} = \left\{(x,y) \mid x^{2}+y^{2}=max(\{N^2_{t} + R^2_{t}, t \in T'\}),x>0,y>0\right\} 
\]
\end{definition}

If we plot all the known research topics according to their relevance and nescience, the unknown frontier will cover them. Intuitively the \emph{unknown frontier} marks the frontier between what we do not know and we are aware that we do not know (we do not fully understand those topics), and what we do not known and we are not yet aware that we do not known those topics. This intuitive property is in general terms, since it may happen that some unknown topics lie under the unknown frontier as well.

\begin{definition}\index{New topics area}
Lets $T'$ the set of known research topics. The \emph{new topics area}, denoted by $\mathbb{S}$, is defined by:
\[
\mathbb{S} = \left\{(x,y) \mid x^{2}+y^{2}>max(\{N^2_{t} + R^2_{t}, t \in T'\}),x>0,y>0\right\} 
\]
\end{definition}

The new topics area contains all those unknown topics that we are not aware we do not know them (unknown unknown). The big issue is how to reach this new topics area if we do not know anything about the topics included in that area.

\begin{proposition}
\label{prop:new-topics-area}
Let $r \in T$ be a topic, if  $N^2_{r} + R^2_{r} > max(\{N^2_{t} + R^2_{t}, t \in T'\}$ then $t \in T \setminus T'$.
\end{proposition}
\begin{proof}
Let $N^2_{r} + R^2_{r} > max(\{N^2_{t} + R^2_{t}, t \in T'\}$ and suppose that $r \in T'$, then have that $max(\{N^2_{t} + R^2_{t}, t \in T'\}) \geq N^2_{r} + R^2_{r}$ that it is a contradiction, and so $t \in T \setminus T'$.
\end{proof}

Proposition \ref{prop:new-topics-area} together with the fact that the combined nescience of two topics is higher than the nescience of any of them isolated (Proposition {\color{red} XXX}), and that their combined relevance is higher that the relevance of any of them (Proposition \ref{prop:nondecreasing_relevance}), a possible approach to identify new topics could be by means of combining already existing interesting problems. In Figure \ref{fig:New-Topics} is depicted graphically the idea.

\begin{figure}[h]
\centering\includegraphics[scale=0.4]{NewTopics}
\caption{\label{fig:New-Topics}The discovery of new research topics}
\end{figure}

\begin{definition}\index{New topic}
Given two topics $t_{1}, t_{2} \in T'$, a \emph{new topic}, denoted by $S_{\left\{ t_{1},t_{2}\right\}}$, is the unordered pair $\left\{ t_{1},t_{2}\right\}$.
\end{definition}

The exact meaning of the new topic that results as the combination of topics $t_{1}$ and $t_{2}$ is left to the creative interpretation of the researcher.

\begin{definition}\index{Interestingness of a new topic}
The \emph{interestingness} of the new topic, denoted by $IS_{\left\{ t_{1},t_{2}\right\} }$, is given by:
\[
IS_{\left\{ t_{1},t_{2}\right\} } = R_{t_{1}}R_{t_{2}}+N_{t_{1}}N_{t_{2}}
\]
\end{definition}

In practice, what we have to do is to compute all possible combination of those topics with very large interestingness as problems $IP_{t}$ with themselves, and select the combinations with higher $IS$. Of course, some of the combinations generated would be totally meaningless. Advanced techniques from the area of natural language processing or machine learning could be used to try filter out those nonsense combinations.

\begin{definition}\index{Intradisciplinary new topic} \index{Interdisciplinary new topic}
Let $A \subset T$ a research area, and $t_{1}, t_{2} \in T$ two topics. If both topics belongs to the same area, that is $t_{1}, t_{2} \in A$, we say that the new topic $S_{\left\{ t_{1},t_{2}\right\} }$ is \emph{intradisciplinary}, otherwise, we say that the topic is \emph{interdisciplinary}.
\end{definition}

Again, the most innovative new topics would be by the combination of interdisciplinary topics, because the probability that somebody has already though about them is lower.

%
% Section: Classification of Research Areas
%

\section{Classification of Research Areas}

In the same way we studied the nescience of research areas (see Section \ref{sec:nescience_areas}), we could also study the interestingness of research areas.

\begin{definition}\index{Average interestingness of an area}
Given a research area $A\subset T$, we define the \emph{average interestingness
of the area as a source of interesting tools} by
\[
IT_{A}=\frac{1}{n}\sum_{t\in A}IT_{t}
\]
and the \emph{average interestingness of the area as a source of interesting
problems} by
\[
IP_{A}=\frac{1}{n}\sum_{t\in A}IP_{t}
\]
where $n$ is the cardinality of $A$.
\end{definition}

In this way we could compute the interestingness of mathematics, physics, biology, social sciences, and other disciplines as a source of interesting tools and problems. Other alternative measures of centrality and dispersion could be used for the characterization of research areas as well.

As I will show in Chapter \ref{chap:The-Scientific-Method}, the interestingness of mathematics as a source of tools is higher than the interest of social sciences, since mathematics is composed of topics with a high applicability that are very well understood, and that is not the case, in general, for social sciences. On the other hand, the interestingness of social sciences as a source of problems is higher than the interest of mathematics, since the topics studied by the social sciences are more relevant to humankind \footnote{Please mind that I am not saying that the topics addressed by mathematics are not relevant to humankind, what I am saying is that, in relative terms, the problems addressed by social sciences have a higher relevance.} and, in general, not very well understood.

\begin{example}\index{Areas in decay}
We could use the interestingness of an area to identify research areas in decay. A knowledge area is in decay (from the research point of view) if it has no enough interesting research problems. For example, although the aerodynamics of zeppelins is not fully understood (still some nescience), it is not longer useful (low relevance), since people does not use zeppelins to travel anymore, and so, the average interestingness is very low. Another example of area in decay is classical geometry: although it is relevant, our understanding of this subject is nearly perfect, since there are almost no unsolved problems, and so, its average nescience is very low. However, on the contrary to what happens in case of the aerodynamics of zeppelins, classical geometry is still very interesting as a source of tools.
\end{example}

It is worth to mention that we could add other metrics to provide a finer, or even alternative, characterization of the unknown unknown area. For example, we could add to nescience and relevance a third dimension with the probability that a topic description is true. However, these extended or alternative characterizations will be not considered in this book. Fortunately, the idea of how to reach the unknown unknown area is the same, regardless of the number and the metrics used (as long as these metrics satisfy some minimal mathematical properties, described in Chapters \ref{chap:Nescience} and \ref{chap:Interesting-Research-Questions}).

%
% References
%

\section*{References}

Popper and falsificationism

